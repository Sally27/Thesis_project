% $Id: introduction.tex 54223 2014-05-16 14:12:08Z pkoppenb $
%%%%%%%%%%%%%%%%%%%%%%%%%%%%%%%%%%%%%%%%%%%%%%%%%%%%%%%%%%%%%%%%%%%%%%%%%%%%%%%%%%%%%%%%%%%%%%

\chapter{Introduction}
The Standard Model (\Gls{SM}) is an effective theory which describes fundamental particles and their interactions to an impressive precision. An example of this precision can be seen in the prediction of the electron magnetic moment using Quantum Electrodynamics (\Gls{QED}). The electron magnetic moment is characterised by the dimensionless quantity, $g$, which, at tree level,  is predicted to be $g = 2$. The deviation due to higher order effects is expressed as $a = \frac{g - 2}{2}$.  This has been predicted to $10^{\mathrm{th}}$ order~\cite{g_2_theory} to be
\begin{equation}
  a_{\mathrm{theory}}=(1.159 652 181 78\pm 0.00000000077)\times10^{-12}
\end{equation}
and has been experimentally measured~\cite{g_2_exp} to be

\begin{equation}
  a_{\mathrm{experiment}}=(1.159652180730\pm000000000028)\times 10^{-12}
\end{equation}
where the numbers in brackets indicate the uncertainties on the last two digits. As can be seen there is agreement between the two values to 12 significant figures. %%%%%%%%%%%%%%%%%%%%%%%%%%%%%%%%%%%%%%%%%%%%%%%%%%%%%%%, within error.

Despite the success of the SM, there are areas where it fails. For example, there is substantial astronomical evidence for Dark Matter\cite{2010arXiv1001.0316R} but the \Gls{SM} offers no Dark Matter candidate. There is also a problem with hierarchy, that is, why the Higgs boson should be so much lighter than the Planck mass. This could be explained by a cancellation between the quadratic radiative corrections and the bare mass but such fine-tuning is thought to be unnatural. %This hierarchy is apparent also in the strength of the weak force compared to gravity, the latter being $10^{32}$ times weaker than the former.

The SM has no way of incorporating gravity and it can be safely left-out due its to comparatively weak strength but this further highlights that the SM is by no means a complete theory. Moreover, although the SM allows for a  small degree of asymmetry between matter and anti-matter there is still no explanation for the much larger amount of asymmetry between matter and anti-matter\footnote{the level of asymmetry between matter and anti-matter allowed for in the  SM falls short of explaining the observed level by several orders of magnitude \cite{book}} observed in the universe. 

These failings motivate theories introducing beyond the standard model (\Gls{BSM})  physics, the existence of which could be inferred from the observation of BSM particles. For example, there are many BSM theories which predict new weakly interacting massive particles or light supersymmetric particles as solutions to the Dark Matter problem~\cite{dark2014}.

Previous measurements in flavour physics suggest that unless it is assumed that the flavour structure of new physics (\Gls{NP}) is near-identical to that of the SM, the masses of these new particles are so high as to be out of the reach of direct searches at current accelerators\cite{kaonmix}. Thus this motivates searches for NP particles via indirect means.

One way of searching for new particles indirectly is by looking for their appearance as virtual BSM particles in rare decays. Given that rare decays are more suppressed in the SM, they could be more sensitive to BSM effects. The existence of BSM particles could be inferred by measuring a difference between the observed value and that predicted in the SM for the branching fraction of a decay. %the branching fractions of decays which are highly suppressed in the \Gls{SM}. This suppression means that these decays may be more sensitive to BSM affects. In the case of rare decays, the existence of BSM particles could be inferred if the measured branching fraction differed significantly from the SM model prediction.

The analysis presented in this thesis uses data taken at the Large Hadron Collider (\Gls{LHC}) using the \Gls{LHCb} detector. The LHCb detector exploits the large forward $b\overline{b}$ production from $pp$ collisions and the  boost of $b$-hadrons along the direction of the beam pipe in order to make precision measurements of $b$-hadron decays. The LHCb detector has been used to study extensively the decay modes and properties of \B-mesons but the decays of $b$-baryons are still relatively unexplored. This motivates measurements of heavily suppressed \Lb decays, which may be sensitive to the existence of new particles.

In addition, while particles with lifetimes up to $\sim$ $10^{-12}$ seconds,  characteristic of \B-decays, have been widely studied at LHCb, longer-lived particles are also of interest for new physics searches. The efficiency to reconstruct such long-lived particles is comparatively poorly known and this motivates the use of new methods to establish the reconstruction efficiency as a function of position in the detector.

%% WILL OUTLINE SPECIFIC CONTENT OF EACH CHAPTER HERE WHEN HAVE WRITTEN ALL CHAPTERS

%% The work in Chapter 4 of this thesis discuss the novel method developed to provide improved precision in the efficiency measurements of long-lived decays. The search for the decay \Lbpi is presented in Chapters 6 and 7.



\newpage




%% The \Lb is a heavy baryon containing a bottom quark. The decay mode \Lbpi has not been yet observed, its Feynman diagram can be shown in 
%% Fig. \ref{Fig:FD} alongside the already observed hadronic decay mode \Lbpijpsi. As \Lbpi is a rare process occurring via a loop diagram it may be sensitive to new physic effects not seen in the resonant case of \Lbpijpsi. In this analysis \Lbpijpsi will be used as the normalization mode.


%% The decay mode \LbK  can be formed by replacing the $d$ quark in Fig. \ref{FD:1} with an $s$ quark.  Thus the ratio $\frac{\BF(\Lbpi)}{\BF(\LbK)}$ is proportional to the ratio of the CKM matrix elements $(\frac{V^{*}_{td}}{V^{*}_{ts}})^{2}$.  If the ratio of form factors can be calculated this ratio provides a test for the Minimal Flavour Violation (MFV) hypothesis - whereby MFV states that this ratio should be consitent with the SM.  The ratio $\frac{\BF(\Lbpi)}{\BF(\LbK)}$ is comparable to that of $\frac{\BF(\Bd\to\pip\mumu)}{\BF(\Bd\to\Kp\mumu}$, as both are comparing \bquark\to\dquark to \bquark\to\squark transitions. The value for $\frac{V^{*}_{td}}{V^{*}_{ts}}$ measured at LHCb, extracted using the ratio $\frac{\BF(\Bd\to\pip\mumu)}{\BF(\Bd\to\Kp\mumu}$,  in the paper in Ref. ~\cite{LHCb-PAPER-2012-020} is consistent with the Standard Model. %If one navievly ignores form factors and phase space one thus naively expects that 
%%the ratio between the decay rates of the two processes, \LbPi, \LbK, to scaleslike $\left(\frac{|V_{cd}|}{|V_{cs}|}\right)^2$ \sim 25.

%% %%%%%%%%%%%%%%%%%%%%%%%%%%%%%%%%%%%%%%%%%%%%%%%%%%%%%%%%%%%%%%%%%%%%%%%%%%%%%%%%%%%%%%%%%%%%%%
%% \begin{figure}[!t]\def\nh{0.5\textwidth}
%% \centering
%% \subfloat[]{\includegraphics[clip=true, trim =30mm 0mm 0mm 0mm, scale = 0.7]{feynmann/mumuargh.pdf}\label{FD:1}} \\
%% \subfloat[]{\includegraphics[clip =true, trim = 30mm 0mm 0mm 0mm, scale = 0.7]{feynmann/jpsi_update_type2.pdf}\label{FD:2}}\hskip 0.04\textwidth
	
%% 	%% \subfloat[]{\includegraphics[height=\nh]{figs/Lb_jpsipK.png}\label{FD:3}}\hskip 0.04\textwidth
%% 	%% \subfloat[]{\includegraphics[height=\nh]{figs/Lb_jpsipK_penguin.png}\label{FD:4}} 
%% 	\caption{Feynman diagrams for \protect\subref{FD:1} the signal channel \Lbpi,  and \protect\subref{FD:2}, the resonant decay \Lbpijpsi. 
%% }
%% \label{Fig:FD}
%% \end{figure}
%% %\subsection{Expected Ratio and yield}\label{Sec:Expect}
%% By comparing the ratio of branching fractions of other decays whose ratio's give $\frac{q_{i}\to q_{j}\mumu}{q_{i}\to q_{j}\jpsi (\to \mumu)}$, one can make a rough estimate for $\frac{\BF(\Lbpi)}{\BF(\Lbpijpsi)}$. This comes out at $\sim$ 0.01. Given that $\sim$ 2000 \Lbpijpsi events were observed with 3\invfb of data in Ref. ~\cite{LHCb-PAPER-2014-020}, and that efficiency selection will be slightly less for \Lbpi one would expect $\sim$ 15-20 signal events. The measurement of \LbK is still work in progress but naively ignoring phase space and form factors the rate for \LbK should be  $(\frac{V^{*}_{td}}{V^{*}_{ts}})^{2}$ ($\sim$ 25)  times more than \Lbpi.

%(such as $\frac{\Bd\to\Kstar\mumu}{\Bd\to\Kstar\jpsi\times \BF(\jpsi\to\mumu)}$, $\frac{\Bu\to\Kp\mumu}{\Bu\to\Kp\jpsi\times \BF(\jpsi\to\mumu)}$ or $\frac{\Bd\to\piz\mumu}{\Bd\to\piz\jpsi\times \BF(\jpsi\to\mumu)}$)

%%%%%%%%%%%%%%%%%%%%%%%%%%%%%%%%%%%%%%%%%%%%%%%%%%%%%%%%%%%%%%%%%%%%%%%%%%%%%%%%%%%%%%%%%%%%%%
%% Another aspect worth studying is the proton-pion system. One can notice that besides the 
%% $c\bar{c}$ pair of $J/\psi$ there is an $udd$ quark system, which needs an additional gluon to 
%% be radiated and decayed into an $u\bar{u}$ pair in order for the $p\pi$ pair to be produced. By 
%% studying the invariant mass distribution of the $p\pi$ system, one can tell whether any extra 
%% resonances are produced within the $udd$ quark system before the final decay. 

%% While searches for \CP-violation have been successful in most mesons, 
%% no evidence of \CP-violation has been found in baryonic decays.
%% This includes both decays of baryons and decays of mesons to baryons
%% (like \Bz\to\proton{}\antiproton).

%% Already in the early 1990es theorists proposed 
%% to search for \CP-violation in beautiful 
%% baryons~
%%\cite{Dunietz:1992ti,*Fridman:1993mn} at the LHC, 
%% but this remained marginal
%% compared to the literature on meson decays. 
%% This low theory output was matching the 
%% experimental capabilities: previous experiments
%% had no (\PB factories) or too little (Tevatron) data to perform 
%% precision physics with heavy baryons.
%% As a consequence most of the theory
%% calculations focus on a limited range of topics of heavy baryons,
%% like semileptonic or rare electroweak penguin 
%% decays\cite{Mannel:2011xg,*Aliev:2010uy,*Lu:2009cm,*Wang:2009hra,*Ball:2008fw,*Wang:2003it,*Bagan:1993ii,*Grozin:1992td,*Hussain:1990uu,*Isgur:1990pm}.

%% This state explains why baryons are also not high on the LHCb physics plate:
%% there are no precise theory predictions to test. There is for instance
%% no mention of them in the original Technical Proposal~\cite{TP}.
%% This situation has not prevented the LHCb collaboration to publish 
%% papers and conference results involving 
%% heavy baryons~\cite{LHCb-PAPER-2011-018,LHCb-PAPER-2012-002,LHCb-PAPER-2012-012,LHCb-CONF-2012-031,LHCb-CONF-2011-007}
%% but the activity is limited and not focusing on precision 
%% measurement of \CP-violation.
%% There has been more activity at the Tevatron, in particular at 
%% the CDF experiment, but the small data samples prevent any conclusive message. 

%% Although \CP-violation effects in baryon decays are expected to occur
%% through similar mechanisms as in meson decays,  we need to keep our eyes open to any 
%% unexpected result. The recent past has shown us
%% beyond the SM asymmetries are not showing up where we expected them.
%%  \Lb\to\jpsi{}\proton{}\Km  and \Lb\to\jpsi{}\proton{}\pim are potential
%% candidates to search for unexpected \CP asymmetries.
%% These 
%% are \bquark\to\cquark{}\cquarkbar{}\squark and \bquark\to\cquark{}\cquarkbar{}\dquark
%% transitions at tree level (Fig.~\ref{Fig:FD}), 
%% similar to \Bd\to\jpsi{}\KS and \Bs\to\jpsi{}\KS~\cite{DeBruyn:2010hh}.

%% In these decays the colour-suppressed tree diagram competes with a penguin diagram with a comparable amplitude.
%% While no \CP-violation is expected from this interference in the Standard Model, potential
%% new particles or couplings could change this picture. 

%% Given no deviation from the SM have been observed so far, such new couplings are bound to be small. 
%% Their contribution to a decay rate 
%% can thus only be observed if they are not overwhelmed by a large
%% Standard Model contribution. In other words, the lower the SM contribution
%% to the decay rate of a given decay is the higher the chances are that such
%% unexpected \CP asymmetries arise. 

%% %#############################################################
%% \subsection{Expected Ratio}\label{Sec:Expect}
%% Very naively assuming phase-space decays, the
%% ratio of branching fractions can be written as
%% \begin{equation}
%% \frac{{\cal B}(\Lbpi)}{{\cal B}(\LbK)}=
%% \frac{|{\cal A}(\Lbpi)|^2}{|{\cal A}(\LbK)|^2}\times
%% \frac{{\Phi}(\Lbpi)}{{\Phi}(\LbK)}.
%% \end{equation}
%% Neglecting any form factors, the ratio of amplitudes for the tree 
%% diagrams is $|\Vcd|^2/|\Vcs|^2=0.0536\pm0.0003$ and for penguin $|\Vtd|^2/|\Vts|^2=0.046\pm0.004$~\cite{PDG2012}.
%% %%%%%%%%%%%%%%%%%%%%%%%%%%%%%%%%%%%%%%%%%%%%%%%%%%%%%%%%%%%%%%
%% \begin{figure}\centering
%% \includegraphics[width=0.8\textwidth]{figs/both.png}
%% \caption{Phase space of \Lbpi (blue) and \LbK (red).}\label{Fig:Both}
%% \end{figure}
%% %%%%%%%%%%%%%%%%%%%%%%%%%%%%%%%%%%%%%%%%%%%%%%%%%%%%%%%%%%%%%%
%% The phase-space factors are calculated using the kinematics section
%% of the PDG~\cite{PDG2012}. The Dalitz planes for the two decays
%% are shown in Fig.~\ref{Fig:Both} and have an area of
%% $43.5\:\gev^4/c^8$ (\Lbpi) and $30.2\:\gev^4/c^8$ (\LbK),\footnote{I
%% never imaged I would once use $c^{-8}$.} for a factor of 0.69.
%% Overall this gives an expected ratio between $7.7\%$ (pure tree)
%% and $6.6\%$ (pure penguin).

%% Interestingly, the recently observed decays \decay{\Lb}{\Lc\Ds} and 
%% \decay{\Lb}{\Lc\Dp} contain exactly the same quark lines as
%% \LbK and \Lbpi, respectively. The measured ratio of BFs is~\cite{LHCb-PAPER-2014-002}
%% \begin{equation*}
%% \frac{{\cal B}(\decay{\Lb}{\Lc\Ds})}{{\cal B}(\decay{\Lb}{\Lc\Dp})} = 0.042\pm0.003\stat\pm0.003\syst.
%% \end{equation*}
%% This is compatible with the expections from $|\Vcd|^2/|\Vcs|^2$ and $f_\Dp/f_\Ds=1/1.26$~\cite{PDG2012}.

