\chapter{Selection and backgrounds in the \boldmath{\Lbpi} analysis}
\label{chap:sel}
The \Lb is a heavy baryon containing a bottom quark. The decay mode \Lbpi is a $b\to d$ \Gls{FCNC} transition and can therefore only occur at loop level in the SM. In new physics models, new heavy particles can contribute and can significantly change the branching fraction of this process.

In this chapter, an outline of the selections used to increase the fraction of signal events with respect to the background is presented, along with the types of backgrounds considered. In the next chapter, the multivariate analysis techniques used to further reduce background are presented. In \autoref{chap:mass} there will be a discussion of the fit models used and the efficiencies of the selections applied and finally the results of this analysis are presented in \autoref{Sec:Results}.

In this chapter in \autoref{sec:overview}, a brief reiteration of the motivation for searching for \Lbpi is presented, along with an outline of the analysis method. This is followed by a summary of the data and simulation samples used in this analysis in \autoref{sec:datasiminto}, and then the details of the initial selection applied to increase the signal fraction of events with respect to the background in \autoref{Sec:Pre}. The agreement between data and simulation samples is discussed in \autoref{sec:datasim}. The selections used to remove decays from $B$ reflections, i.e. decays with a mis-identified particle that can accumulate at a certain $B$ mass, and background from the Cabibbo-favoured channel \LbK, are discussed in \autoref{Sec:backgrounds}. Finally, there is a discussion of partially reconstructed backgrounds in \autoref{Sec:partreco}. %various multi-variant analysis techniques usedt o reduced background in X, Y.

This analysis was performed blind, meaning that events within the signal region of the \Lbpi mass spectrum were removed from the dataset during all selection processes outlined in this chapter and during the optimisation process outlined in \autoref{chap:bdt}. Therefore, all plots of the \Lbpi mass spectrum in chapters \ref{chap:sel}--\ref{chap:mass} do not show the signal region.

\section{Overview}
\label{sec:overview}
The aim of the analysis presented in the next four chapters is to measure the \Lbpi branching fraction. The \Lbpi decay mode has never been observed and neither has any other baryon decay mediated by a $b\to d$ transition. The \Lbpi branching fraction is measured relative to the resonant decay mode \Lb\to\proton\pim\jpsi(\to\mumu), which has already been observed \cite{LHCb-PAPER-2014-020}. It is desirable to measure the branching fraction relative to the branching fraction of another decay with the same final state, as many sources of systematic uncertainty cancel. For example the absolute number of \Lb baryons produced within LHCb does not affect the ratio of \BF(\Lbpi)/\BF(\Lb\to\proton\pim\jpsi(\to\mumu)). The \Lb\to\proton\pim\jpsi(\to\mumu) channel is hereafter denoted simply \Lbpijpsi. % is an ideal candidate as it has an identical final state as that of \Lbpi.

%% The Feynman diagrams for the \Lbpi decay were shown in \autoref{fig:boxpeng}\protect\subref{FD:1}, \protect\subref{FD:3} in \autoref{sec:bdll} in \autoref{chap:theory}. The \Lbpi penguin diagram is shown again in \autoref{Fig:FD}, alongside the diagram for the hadronic decay mode \Lbpijpsi.  For both the \Lbpi and \Lbpijpsi decay modes, the proton and pion come from strongly decaying $N^{*}$ resonances.
The Feynman diagram for the \Lbpi decay is shown in \autoref{fig:boxpeng}\protect\subref{FD:1} and the diagram for the hadronic decay mode \Lbpijpsi is shown in \autoref{Fig:FD} .  For both the \Lbpi and \Lbpijpsi decay modes, the proton and pion can come from strongly decaying $N^{*}$ resonances.

\begin{figure}[!ht]%\def\nh{0.5\textwidth}
  \centering
 %% \hspace*{-2.4cm}
 %%  \subfloat[]{\includegraphics[clip=true, trim =10mm 150mm 0mm 0mm, scale = 0.47]{figs/Lb_peng_Huge}\label{FD:1}}
%  \subfloat[]{
    \includegraphics[clip =true, trim = 10mm 170mm 0mm 0mm, scale = 0.57]{figs/Lb_jpsi_HUGE}%}\label{FD:2}
  %% \caption{Feynman diagrams for the signal channel \Lbpi \protect\subref{FD:1},  and the resonant decay \Lbpijpsi \protect\subref{FD:2}.}
    \caption{Feynman diagram for the decay \Lbpijpsi.}
  \label{Fig:FD}
\end{figure}

%%%%%%%%%%%%%%%%%%%%%%%%%%%%%%%%%%%%%%%%%%%%%%%%%%%%%%%%%%%%%%%%%%%%%%%%%%
%% The diagram for the Cabibbo favoured decay mode \LbK can be formed by replacing the $d$ quark in \autoref{Fig:FD}\protect\subref{FD:1} with an $s$ quark.  Thus the ratio of branching fractions, $\BF(\Lbpi)/\BF(\LbK)$, is proportional to the ratio of the CKM matrix elements ${|\frac{\Vtd}{\Vts}|}^{2}$. Together with the $\LbK$ and $\Lbpi$ form factors, a measurement of this branching fraction ratio will therefore allow ${|\frac{\Vtd}{\Vts}|}^{2}$ to be computed and hence a test of the Minimal Flavour Violation (MFV) hypothesis to be made.  This hypothesis predicts that the ratio ${|\frac{\Vtd}{\Vts}|}^{2}$ should be the same in NP models as in the SM. Further measurements of ${|\frac{\Vtd}{\Vts}|}^{2}$ are of interest given the hints of some tension in the CKM unitarity condition that is observed following recent improvements in lattice calculations \cite{vtdvts}. This tension is shown in \autoref{fig:vtdvts}. One of the values of the ratio $|\frac{\Vtd}{\Vts}|$ shown in \autoref{fig:vtdvts} is calculated using the branching fraction ratio $\BF(\Bu\to\pip\mumu)/\BF(\Bu\to\Kp\mumu)$, which was measured using LHCb data in Ref.\cite{LHCb-PAPER-2012-020}.linded


%% \begin{figure}
%%   \centering
%%   \includegraphics[scale = 0.5]{figs/vtdvts.png}
%%   \caption{The values of \Vtd, \Vts and $|\Vtd/\Vts|$ given by tree-level and different loop-level measurements.}
%%   \label{fig:vtdvts}
%% \end{figure}

%%%%%%%%%%%%%%%%%%%%%%%%%%%%%%%%%%%%%%%%%%%%%%%%%%%%%%%%%%%%%%%%%%%%%%%%%%%%%%%%%%%%%%%%%%%%%%


\autoref{tab:bfcomp} shows the comparison of the branching fractions of various decays of the form $\frac{H_{b}\to X_{q}\mu\mu}{H_{b} \to X_{q} \jpsi (\to\mu\mu)}$, where $H_b$ is a hadron containing a $b$-quark and $X_{q}$ is any hadron. The majority of the ratios, $\frac{H_{b} \to X_{q} \jpsi (\to\mu\mu)}{H_{b}\to X_{q}\mu\mu}$, in \autoref{tab:bfcomp}, are of order 100. Taking the ratio between the \Lbpijpsi and \Lbpi branching fractions  to also be of order 100, the 2000 \Lbpijpsi decays observed in 3\invfb of LHCb data in Ref.~\cite{LHCb-PAPER-2014-020} would then lead to an expectation of $\sim$ 10--20 \Lbpi signal events, depending on the relative efficiencies between the \Lbpi and \Lbpijpsi channels. If the ratio of the \Lb decay modes, $\frac{\Lb\to\PLambda\mup\mun}{\Lb\to\PLambda\jpsi}$  is used however, an event yield of 30--60 is expected. In either case, given the small number of signal events expected, the primary challenge for this analysis is the reduction of backgrounds. The branching fractions for the decays \Lbpijpsi and \LbKjpsi are also shown in \autoref{tab:bfcomp}. There is currently no branching fraction measurement for the decay \LbK, although it is an ongoing LHCb analysis at the time of writing. Once the \LbK analysis at LHCb is completed, the \LbK and \Lbpi branching fractions will be combined to deduce the ratio, $\mathcal{R}$, between them. Note that there is little to gain by deducing the \LbK and \Lbpi branching fractions in a combined analysis (which would allow for the cancellation of some systematic uncertainties when calculating $\mathcal{R}$) as the error on $\mathcal{R}$ will be statistically dominated due to the limited statistics available, particularly in the \Lbpi channel. As such, the work outlined in this thesis only measures the branching fraction measurement for \Lbpi and does not attempt to calculate $\mathcal{R}$.  % The branching fractions for the \Lbpijpsi decay, and the Cabibbo-favoured decay \LbKjpsi, are also shown


\begin{table}

  \centering
  \begin{tabular}{l|c|c}
    %% \hline
    %% \BdToJPsiKst  &  (1.32$\pm$ 0.06) $\times$ $10^{-3}$\\
    %% \BdToKstmm & (1.05 $\pm$ 0.10) $\times$ $10^{-7}$\\
    \hline
    Decay& Branching Fraction & Ratio of branching fractions\\
    &&     $\jpsi\to\mumu/\mu\mu$\\
    \hline
    $\Bd\to\Kstarz \mup\mun$ & (1.05 $\pm$ 0.10) $\times$ $10^{-6}$ & \multirow{2}{*}{74.9$\pm$7.9}\\
    $\Bd\to\Kstarz \jpsi$ & (1.32 $\pm$ 0.06) $\times$ $10^{-3}$ &\\
    \hline
    $\Bu\to\Kp \mup\mun$ & (4.43$\pm$0.24)$\times 10^{-7}$ & \multirow{2}{*}{137.2$\pm$8.6}\\
    $\Bu\to\Kp \jpsi$ & (1.02$\pm$0.03)$\times 10^{-3}$ & \\
    \hline
    $\Bd\to\pip\mup\mun$ & (2.3$\pm$0.6)$\times 10^{-8}$ &  \multirow{2}{*}{106 $\pm$ 30}\\
    $\Bd\to\pip\jpsi$ & (4.1$\pm$0.4)$\times 10^{-5}$ &\\
    \hline
    $\Lb\to\Lz\mup\mun$ & (1.1$\pm$0.3) $\times 10^{-6}$& \multirow{2}{*}{34$\pm$12}\\
    $\Lb\to\Lz\jpsi$ & (6.2$\pm$1.4) $\times 10^{-4}$&\\
    \hline
    $\Lbpijpsi$ & $(2.6\pm^{+0.5}_{-0.4}) \times 10^{-5}$& ---\\
    \hline
    $\LbKjpsi$ & $(3.0\pm^{+0.6}_{-0.4}) \times 10^{-4}$& ---\\
    \hline
    
  \end{tabular}
  \caption{Comparison of branching fractions between the $\mumu$ and $\jpsi$ channels for various decays. The ratio between the resonant and non-resonant channels is estimated assuming that $\BF(\jpsi\to\mumu) = 0.0596 \pm 0.0003$. All values for the branching fractions are taken from Ref.\cite{pdg}, with the exception of the \Lbpijpsi and \LbKjpsi branching fraction measurements, which are taken from Refs~\cite{LHCb-PAPER-2014-020} and \cite{LbKjpsi} respectively.}
  \label{tab:bfcomp}
  
\end{table}

 %%%%%%%%%%%%%%%%%%%%%%%%%%%%%%%%%%%%%%%%%%%%%%%%%%%%%%%%%%%%%%%%%%%%%%%%%%
\cleardoublepage
%% METHOD
%%%%%%%%%%%%%%%%%%%%%%%%%%%%%%%%%%%%%%%%%%%%%%%%%%%%%%%%%%%%%%%%%%%%%%%%%%
\subsection{Analysis strategy}
\label{sec:ana}
%\footnote{A more detailed explanation of BDT-based multi-variant analysis methods can be found in Appendix \ref{app:bdt}.}
The overall idea of this analysis is to fit the \Lb mass distribution for signal and background components after applying background reduction criteria in the form of specific vetoes and a BDT. The remaining backgrounds, after these specific vetoes and the BDT have been applied, are combinatorial background and partially reconstructed (\gls{partreco}) background. Combinatorial background arises from cases when tracks coming from different mother particles are combined to form a candidate. Part-reco background refers to cases when a decay has been mis-identified as a signal candidate and not all the final state particles of the mis-identified decay are included in the final reconstruction. In this analysis, the part-reco background predominantly comes from semi-leptonic  cascades of the form $\alpha\to\beta(\to X)\mu\nu$, where the decay products $X$ contain a hadron that has been mis-identified as a muon. The \Lbpijpsi and $\Lb\to\psi(2S)(\to\mumu)\proton\pim$ channels are vetoed during the \Lbpi selection by applying cuts on the $q^{2}$ spectrum.

Using \Lbpijpsi as the normalisation channel,  the \Lbpi branching fraction can be determined from the expression

\begin{equation}
  \BF(\Lbpi) = \frac{N_{\Lbpi}}{N_{\Lbpijpsi}}\times\frac{\epsilon_{\Lbpijpsi}}{\epsilon_{\Lbpi}}\times \BF(\Lbpijpsi)\BF(\jpsi\to\mumu),
  \label{eq:norm}
  \end{equation}
where $\epsilon_{X}$ refers to the total selection efficiency of channel $X$. The terms $N_{X}$ in \autoref{eq:norm} are extracted by fitting the \Lb mass distributions in data and $\epsilon_{X}$ are calculated using simulation, with the exception of the BDT efficiency. The order of selection applied in the analysis is outlined as follows:
\begin{itemize}
  \item Application of an initial loose selection and removal of the $\Lb\to\psi(2S)\proton\pim$ channel by placing vetoes in $q^{2}$.
  \item Application of a selection dedicated to the removal of \gls{reflections}, i.e. decays with a mis-identified particle that can accumulate at a certain $B$ mass.
  \item Application of a BDT to further reduce combinatorial and reflection backgrounds.
    \item Selection of either the \Lbpi or the \Lbpijpsi channel by applying vetoes or selections respectively in $q^{2}$.
\end{itemize}

\section{Data samples and simulation}\label{Sec:Exp}\label{Sec:Selection}
\label{sec:datasiminto}
The data-set analysed during this study represents an integrated luminosity of $3\:\invfb$ of
$pp$-collisions recorded over the 2011 and 2012 data-taking periods. 

%%%%%%%%%%%%%%%%%%%%%%%%%%%%%%%%%%%%%%%%%%%%%%%%%%%%%%%%%%%%%%%%%%%%%%%%%%
\subsection{Data and simulation samples}\label{Sec:Samples}
Signal and background simulated events were generated with PYTHIA 6.4 and 
8.1~\cite{pythia6,pythia8}, parameterised as 
specified in Ref.\cite{LHCb-PROC-2010-056}. Unstable particles were decayed with EVTGEN \cite{Lange:2001uf} and the detector response was simulated with 
GEANT4 \cite{Agostinelli:2002hh,LHCb-PROC-2011-006,*Allison:2006ve}. All candidates are required to be in the detector acceptance. All simulation samples used in this analysis are generated using phase space only models. In phase space simulation, the decay rate for the simulated process is calculated disregarding the form factors arising from the presence of hadronic matrix elements in the complete expression for the decay rate, which have not yet been predicted. The distributions generated in phase space simulation are dependent only on the kinematics of the decay, meaning that resonances and other QCD effects will not be modelled. 

\section{Preselection}
\label{Sec:Pre}
The term preselection refers to the initial selection placed on data candidates to increase the fraction of signal events with respect to the background.
This section discusses all the preselection criteria applied. %The section is followed by a discussion on the removal of so-called reflection backgrounds in \autoref{Sec:backgrounds}.  The application of the BDT is discussed in \autoref{chap:bdt} respectively.
\subsection{Stripping cuts}\label{Sec:Strip}
As previously discussed in \autoref{chap:ks}, after the data output by the LHCb HLT2 trigger has been stored, an initial selection is placed on the data, referred to as the stripping selection. The stripping line used for the \Lbpi analysis is the {\tt B2XMuMu} line. This line features the selections shown in \autoref{tab:strip}. In \autoref{tab:strip}, the variables $m_{B}$, $m_{\mumu}$ refer to the invariant mass of $\Kp\pim\mumu$ and of the dimuon system, respectively. The $\B$ \gls{bm} refers to the vertex fit for the combination of the daughter tracks of the $\B$, where $B$ refers to the mother particles, which in the stripping line is assumed to be a $B$ meson. All other variables in \autoref{tab:strip} have already been defined in \autoref{sec:hlttrig}.% in \autoref{chap:dec}.

%The {\tt B2XMuMu} line selects \Lbpi candidates with the requirements listed 
\begin{table}[h!]
      \centering
  \begin{tabular}{ l | c }
    \hline
    Selection & Criteria \\
    \hline
    $m_{B}$ & $4800 < m_{\Kp\pim\mu\mu} < 7100\mevcc$ \\
    $m_{\mup\mun}$ & $< 7100\mevcc$ \\
    %       track \pt & $> 0.8\gevc$ \\
    $\mu$ \gls{ipchi2}& $> 9$ \\
    $h$ \gls{ipchi2}& $> 6$ \\
    %       clone distance & $>5000$ \\
    %       \mup nShared & $< 3$ \\
    \B \gls{ipchi2} & $< 16$ \\
    %       \Bp \pt & $> 2.5\gevc$ \\
    %\mup \mun vertex $\chisq/{\mathrm {ndof}}$ & $< 9$ \\
    \B vertex $\chisq/{\mathrm {ndof}}$ & $< 8$ \\
    %\B $\theta$ & $< 8\mrad$ \\
    \B \gls{fdchi2}& $>$ 121\\
    %       $h^{+}$ $p$ & $>$ $4 \gevc$ \\
    %       \Kp (\pip) DLL$_{K\pi}$ & $>1$ ($<-1$) \\
    %       \pip \dllkpi & $< -1$ \\
    %       \Kp (\pip) DLL$_{K\pi}$ & $>1 (<-1)$ \\
    $\mu$ \gls{dllmupi} & $> -3$ \\
    $\mu$ \gls{isMuon} & True \\
         
    %      $h^{+}$ isMuonLoose & False \\
    %      $h^{+}$ inMuonAcceptance & True \\
    \hline
  \end{tabular}
  \caption{The selection criteria used in the stripping line.}
  \label{tab:strip}
  \end{table}
  
%\footnote{More details are available at 
%\href{http://cern.ch/LHCb-release-area/DOC/stripping/config/stripping20/dimuon/strippingbetaslambdab2jpsippidetachedline.html}{http://cern.ch/LHCb-release-area/DOC/stripping/config/stripping20/dimuon/\-strippingbetaslambdab2jpsippidetachedline.html}.}
The stripping line does not apply hadron PID requirements and events are selected using the stripping line criteria in \autoref{tab:strip} with the mass hypothesis of the daughters being taken as $\Kp\pim\mumu$. Once selected, the mass hypothesis of the kaon is substituted with that of a proton. This means that the cut at 4.8\gevcc on the $\Kp\pim\mumu$ mass in the stripping line is not a clean cut off under the \mumuppi mass hypothesis but rather gives a shoulder rising up from 4.8\gevcc. Due to this rise, when fitting the \Lb mass distribution, only events above 5100\mevcc are considered.


%% , as shown in \autoref{fig:rise}. \autoref{fig:rise} shows the lower mass range of the data with the preselection and stripping requirements applied, under both the \mumuppi mass hypothesis and the \Kp\pim\mumu mass hypothesis.



%% \begin{figure}
%% \centering
%% \includegraphics[height = 5.5cm]{figs/rise.pdf}
%% \caption{
%% The lower mass range of the data with the preselection and stripping requirements applied, under both the \proton\pim\mumu mass hypothesis and the \Kp\pim\mumu mass hypothesis. }
%%  \label{fig:rise}
%% \end{figure}


%%%%%%%%%%%%%%%%%%%%%%%%%%%%%%%%%%%%%%%%%%%%%%%%%%%%%%%%%%%%%%%%%%%%%%%%%%

  \subsection{Trigger requirements}\label{Sec:Trigger}
 The trigger requirements placed on all data and simulation samples are listed in \autoref{tab:trig}. At least one line from each of the L0, HLT1 and HLT2 triggers must be passed. All trigger lines in \autoref{tab:trig} have been previously discussed in sections \ref{sec:trig} and \ref{sec:hlttrig} in \autoref{chap:dec}.
\begin{table}[h!]
  %\hline
  \label{tab:trig}
  \centering
  \begin{tabular}{ l }
    
    \hline
    Trigger Decisions\\
    \hline
    \tt LambdabL0MuonDecisionTOS \\
   \hline
   \tt LambdabHlt1TrackAllL0DecisionTOS \\
   \tt LambdabHlt1TrackMuonDecisionTOS\\
   \hline
   \tt LambdabHlt2TopoMu2BodyBBDTDecisionTOS     \\
   \tt LambdabHlt2TopoMu3BodyBBDTDecisionTOS     \\
   \tt LambdabHlt2Topo2BodyBBDTDecisionTOS       \\
   \tt LambdabHlt2DiMuonDetachedDecisionTOS      \\
   \tt LambdabHlt2DiMuonDetachedHeavyDecisionTOS\\
   \hline
  \end{tabular}
  
  \caption{Trigger requirements.}
          \label{tab:trig}  
    \end{table}

\subsection{Preselection criteria}\label{Sec:Presel}


Further preselection cuts have been applied as outlined in \autoref{tab:presel}.  These cuts are placed to reduce combinatorial background, and to eliminate some backgrounds arising from the mis-identification of one or more daughter particles, i.e. reflection backgrounds. These reflection backgrounds are discussed in full in \autoref{Sec:backgrounds}, along with the additional selections used to further reduce them. Similarly, the hard cut made on the proton momentum is imposed because of the increased mis-identification rate of kaons as protons at low momentum when applying the PID \dllpk requirement, where \dllpk = \dllppi - \dllkpi. Similarly, the hard cut made on \dllpk is because of the large background from \Bd\to\Kstarz(\to\Kp\pim)\mumu decays. The lower bound cut on the dihadron mass, $m_{p\pi}$, is to ensure that there is no contribution from \Lb\to\Lz(\to\proton\pim)\mumu decays. This cut is 99.99\% efficient on the signal channel, according to simulation. There is no upper bound placed on $m_{p\pi}$, where values go up to just under 5000\mevcc according to simulation. The branching fraction measurement performed in this analysis is therefore defined as being valid within the region of $1120\mevcc<m_{p\pi}<5000\mevcc$. The $q^{2}$ range used in the \Lbpi channel is from 0 to 20 $\gev^{2}/c^{4}$, where the upper limit represents approximately the phase space limit. Additional vetoes are placed on the $q^{2}$ spectrum to suppress background, as discussed in \autoref{sec:q2}.

%% The initial requirement on the \dllpk variable in this analysis is $\dllpk>8$. The mis-id rate for the similar requirement of $\dllpk>5$, as a function of the momentum of the proton, is demonstrated in \autoref{fig:pid3}, taken from Ref.\cite{LHCb-DP-2012-003}. Although the cut of 7.5\gevc on the proton momentum is soft compared to other analyses, for example the analysis in Ref~\cite{LHCb-PAPER-2014-020}, as the proton PID is in the BDT, the BDT effectively applies a harder cut on the momentum through its exploitation of the correlations between momentum and PID performance. This is demonstrated in \autoref{fig:protonjpsi}, which shows the momentum distribution of the proton from  sWeighted \Lbpijpsi data, after the BDT has been applied. 

%% \begin{figure}[h!]
%%   \centering
%%   \includegraphics[scale = 0.5]{figs/pplussjpsippi.pdf}
%%     \caption{The momentum distribution of the proton for sWeighted \Lbpijpsi data after the BDT cut has been applied. }
%%         \label{fig:protonjpsi}           
%% \end{figure}                       


%% The possibility of using the $m_{p\pi}$ variable as a discriminator against background, by placing an upper cut on this variable, was also considered. It is known from \Lbpijpsi data that the signal will peak at lower values of $m_{p\pi}$, due to the dominant lower mass $N^{*}$ resonances, as discussed further in \autoref{sec:ppisys}. However, as shown in \autoref{Fig:mppijpsi} , the combinatorial background, here taken from the upper mass side band of \Lbpi data, also has similar peaking behaviour at low $m_{p\pi}$, and therefore is not used as a discriminate.

%% \begin{figure}[!t]\def\nh{0.33\textwidth}
%% \centering
%% \includegraphics[scale = 0.8]{figs/ppimumu_umsb_ppi.pdf}
%%         \caption{Mass fit to \LbKjpsi data.}%, \protect\subref{phasespaceLc:3} $\Lb_\pt$, \protect\subref{phasespaceLc:4}}

%%   \label{Fig:mppijpsi}
%%    \end{figure}



The efficiency of the total preselection on signal is (67.7$\pm$0.5)\%, with the largest efficiency loss being due to the PID selection.








      \begin{table}[!ht]
          \centering
            \begin{tabular}{c|c}
                 \hline
                     Quantity & Selection                 \\
                     
                     
    \hline           
        $\proton$ \pt  & $> 400$\mevc         \\
            $\pion$ \pt & $> 400$\mevc          \\
                $\proton$ momenta  & $> 7500$\mevc    \\
                    $\pion$ momenta  & $> 2000$\mevc    \\
                        \Lb vertex \chisq & $< 4$                  \\
                            \Lb \Gls{DIRA} & $> 0.9999$            \\
                                                        $m_{p\pi}$& $1120<m_{p\pi}< 5000$\mevcc            \\       

    \hline
        $\proton$ $\dllppi$  & $> 0$           \\
            \proton ($\dllppi$-$\dllkpi$)  & $> 8$            \\
                $\pim$ $\dllkpi$  & $< -5 $            \\
            $\pim$ isMuon & False \\
            $\proton$ isMuon & False \\
                    \hline

  \end{tabular}
    \caption{Preselection applied.}
      \label{tab:presel}
      \end{table}
 \subsection[Selections in $q^{2}$]{Selections in $\mathbold{q^{2}}$}\label{sec:q2}
The majority of the selection process during this analysis is performed on the combined \Lbpijpsi and \Lbpi data sets, with the only selection on $q^{2}$ initially being the removal of the \psitwos region. In the final stages of the analysis  additional cuts are placed on the $q^{2}$ distributions to select either the \Lbpi or \Lbpijpsi candidates, with the criteria shown in \autoref{tab:q2}. 






\begin{table}[ht]

  \centering
    %\caption{Vetos in $q^{2}$}

  \begin{tabular}{l| c }
      \hline
          Cut & $q^{2}$ veto / $(\gev^{2}/c^{4})$ \\
              \hline
                  $\jpsi$ veto  &8.0--11.0 \\
                      $\psi$(2S) veto  & 12.5--15.0 \\
                          \hline
                              Cut used to select the \Lbpijpsi channel & 9.0--10.0\\
                                  %% \p \Km & 5279.5 \pm 36    & 0.00 \\
                                      %% \Kp \pm & 5279.5 \pm 18    & 0.00 \\
                                          %% \Kp \pm & 5279.5 \pm 18    & 0.28 \\
                                              \hline

  \end{tabular}

  \caption{Vetoes in $q^{2}$ for the \Lbpi channel and the selection in $q^{2}$ for the \Lbpijpsi normalisation channel.}
    \label{tab:q2}
    \end{table}
    
\section{Data and simulation agreement}
\label{sec:datasim}
Simulation is used to estimate the selection efficiency, however some variables, such as the PID likelihoods, are known to be poorly described in the simulation. This section describes the techniques used to improve the agreement between data and simulation.

There are two techniques used: the reweighting of the simulation, discussed in \autoref{sec:reweight}, and the resampling of the simulation, discussed in \autoref{sec:resample}.

\subsection{Simulation reweighting}
\label{sec:reweight}
The simulation is reweighted in order to better reproduce the distributions of variables used in the selection. These variables are the track multiplicity ($n_{t}$), momentum ($p$), transverse momentum (\pt) and lifetime ($t$) of the \Lb. The \Lbpijpsi decay is used as a signal proxy. The weights are calculated by comparing \Lbpijpsi data to \Lbpijpsi simulation. The \Lbpijpsi data candidates have all selection criteria applied. These weights are then applied to both \Lbpijpsi simulation and \Lbpi simulation.

%The \Lb $p$ and $p_{T}$ variables are reweighted simultaneously, due to the strong correlations between these two variables. The track multiplicity and lifetime are reweighted independently, one after the other. 

The distributions of $\Lb$ $p, p_{T}, t$ and $n_{t}$ for \Lbpijpsi data along with \Lbpijpsi simulation before and after reweighting are shown in \autoref{Fig:jpsijpsiw}.  The equivalent distributions for \Lbpi simulation, again shown against \Lbpijpsi data, are shown in \autoref{Fig:mumjpsiw}. %in The results can be seen for the case where the weigh

Both the \Lbpi and \Lbpijpsi simulation agree with \Lbpijpsi data within errors after reweighting. The effect of the simulation reweighting on the final efficiency value is discussed in more detail as a source of systematic uncertainty in \autoref{sec:rew}.
%The selection efficiency is calculated using both weights from \Lbpijpsi data and \LbKjpsi data, with the efficiency calculated using weights from \Lbpijpsi data being taken as the nominal value. 



    \begin{figure}[!t]\def\nh{0.23\textwidth}
  \centering
  \hspace*{-2.3cm}
        \subfloat[]{\includegraphics[scale = 0.5]{figs/mc_data_PT__rw_sim_jpsisimjpsippidatanoreflcuts}\label{phasespaceLc:1}} 
\subfloat[]{\includegraphics[scale = 0.5]{figs/mc_data_P__rw_sim_jpsisimjpsippidatanoreflcuts}\label{phasespaceLc:2}}\\\hskip 0.04\textwidth
\hspace*{-2.3cm}                           
            \subfloat[]{\includegraphics[scale = 0.5]{figs/mc_data_NT__rw_sim_jpsisimjpsippidatanoreflcuts.pdf}\label{phasespaceLc:3}} 
    \subfloat[]{\includegraphics[scale = 0.5]{figs/mc_data_TAU__rw_sim_jpsisimjpsippidatanoreflcuts.pdf}\label{phasespaceLc:4}}\\\hskip 0.04\textwidth
            \caption{\Lbpijpsi simulation distributions, before and after reweighting compared to sWeighted \Lbpijpsi data for $\Lb$ $p_{T}$ \protect\subref{phasespaceLc:1},  $\Lb$ $p$ \protect\subref{phasespaceLc:2}, track multiplicity \protect\subref{phasespaceLc:3},  $\Lb$ $\tau$ \protect\subref{phasespaceLc:4}.}%, \protect\subref{phasespaceLc:3} $\Lb_\pt$, \protect\subref{phasespaceLc:4}}
                \label{Fig:jpsijpsiw}
   \end{figure}
    \begin{figure}[!t]\def\nh{0.23\textwidth}
  \centering
  \hspace*{-2.3cm}
        \subfloat[]{\includegraphics[scale = 0.5]{figs/mc_data_PT__rw_sim_mumusimjpsippidatanoreflcuts}\label{phasespaceLc:1}} 
\subfloat[]{\includegraphics[scale = 0.5]{figs/mc_data_P__rw_sim_mumusimjpsippidatanoreflcuts}\label{phasespaceLc:2}}\\\hskip 0.04\textwidth
\hspace*{-2.3cm}                           
            \subfloat[]{\includegraphics[scale = 0.5]{figs/mc_data_NT__rw_sim_mumusimjpsippidatanoreflcuts.pdf}\label{phasespaceLc:3}} 
    \subfloat[]{\includegraphics[scale = 0.5]{figs/mc_data_TAU__rw_sim_mumusimjpsippidatanoreflcuts.pdf}\label{phasespaceLc:4}}\\\hskip 0.04\textwidth
            \caption{\Lbpi simulation distributions, before and  after reweighting compared to sWeighted \Lbpijpsi data for $\Lb$ $p_{T}$ \protect\subref{phasespaceLc:1},  $\Lb$ $p$ \protect\subref{phasespaceLc:2}, track multiplicity \protect\subref{phasespaceLc:3},  $\Lb$ $\tau$ \protect\subref{phasespaceLc:4}.}%, \protect\subref{phasespaceLc:3} $\Lb_\pt$, \protect\subref{phasespaceLc:4}}
                \label{Fig:mumjpsiw}
    \end{figure}

    
    
\FloatBarrier
\subsection{Resampling the simulation}
\label{sec:resample}

The PID variables, $\mathrm{DLL}_{X\pi}$, express the difference in an event's total likelihood when the hypothesis of the track in question is changed from being that of a pion to particle X. The distributions of these variables in simulation are resampled, meaning that the PID value for an event is replaced with a value picked at random from the PID distribution for a given track multiplicity, pseudo rapidity ($\eta$) and momentum taken from the data\footnote{The advantage of using this technique is that the PID distributions are not correlated after resampling, meaning that they can still be used to train BDTs (although this is not the case in this analysis)}. The data samples used to produce these distributions are obtained from large calibration samples where different species of particles have been selected without using any PID information.   The PID variables of interest in the present analysis are $\dllppi$, $\dllkpi$ and $\dllpk = \dllppi - \dllkpi$. The calibration samples used to produce the relevant PID distributions in data are shown in \autoref{tab:ur}. The data calibration samples are provided by a dedicated LHCb software package.

\begin{table}[!ht]

  \centering
  
  \begin{tabular}{l c c}
      \hline
      Particle & Decays used&PID variables\\
    \hline
    \kaon&$\D^{+*}$\to\Dz(\to \Km$\pip$)\pip& \dllkpi\\
    $\pi$&$\D^{+*}$\to\Dz(\to \Km$\pip$)\pip& \dllkpi\\
    \proton&\Lz\to\proton$\pim$& \dllppi \\
    \hline
 \end{tabular}
 \caption{Decay types used to produce the calibration samples used for resampling.}
\label{tab:ur}

\end{table}

\subsubsection{Verifying the resampled simulation against data}

  Resampled simulation samples are checked against \LbKjpsi and \Lbpijpsi sWeighted data. Comparisons for the pion \dllkpi and the proton \dllppi PID variables between simulation and data are shown in \autoref{Fig:resample}. The simulation has been reweighted in these plots as outlined in \autoref{sec:reweight}. The agreement between data and simulation is good for both PID variables. The discrepancy at high values for the proton \dllppi is due to an upper cut of 50 in the data calibration samples. The effect that this cut has on the final relative efficiency between the signal and normalisation channel is discussed in \autoref{sec:cutpid}. % in \autoref{Sec:Results}.

More generally, any discrepancies between data and simulation are not considered problematic, as the same PID variables are featured in the signal and normalisation channels and the simulation is only used to estimate the relative efficiency between the two. %not used to train the BDT or to estimate its efficiency. 

  \begin{figure}[!ht]\def\nh{0.3\textwidth}
  \centering
    \subfloat[]{\includegraphics[height = 5.5cm]{figs/p_PIDp_resampled_edit_th}\label{resample:2}}\\
      \subfloat[]{\includegraphics[height = 5.5cm]{figs/KpiPIDredone_th}\label{resample:3}}\\
  \caption{The distribution of the proton \dllppi variable for resampled \LbKjpsi simulation and \LbKjpsi sWeighted data \protect\subref{resample:2}, and the distribution of the pion \dllkpi variable for resampled \Lbpijpsi simulation and \Lbpijpsi sWeighted data \protect\subref{resample:3}.}
  \label{Fig:resample}
  \end{figure}



\begin{figure}[!t]\def\nh{0.3\textwidth}
           \centering   
       \includegraphics[height = 5.5cm]{figs/pidpk_mc_data_edited_th}   %\label{q2:3}}                                                                 
        \caption{The proton $\dllppi$-$\dllkpi$  variable for \LbKjpsi sWeighted data and \LbKjpsi resampled simulation.}%\protect\subref{q2:2}, \LbKjpsi resampled simulation \protect\subref{q2:1} and proton $\dllppi$-$\dllkpi$ \protect\subref{q2:3} for both MC and data}
 \label{Fig:pidpkcorr}       
\end{figure}
The proton \dllpk distribution, as seen in \autoref{Fig:pidpkcorr}, is poorly replicated in simulation. This is due to the fact that binning the PID variables in the quantities $n_{t}$, $\eta$ and $p$ preserves only the correlation in these binned quantities after resampling. This presents a problem with the variable $\dllpk$ as in the data samples provided by LHCb software it is expressed as the difference between $\dllkpi$ and $\dllppi$, meaning it is not possible to resample the variable \dllpk directly. These two variables, $\dllkpi$ and $\dllppi$, are correlated: if a particle, with a given value of $n_{t}$, $\eta$ and $p$, creates a ring in the RICH detectors which appears more pion-like when considering \dllppi,  it is also more likely to be classified as more pion-like when considering \dllkpi.


Due to this poor replication,  the \dllpk variable cannot be modelled with resampled simulation. Where the variable is used in data, the selection efficiency is computed directly using the data calibration samples shown in \autoref{tab:ur}.

%% The efficienies of the selections on the \dllppi and \dllkpi variables are calculated using resampled simualtion as, due to the complex mass-dependent selections on the PID variables, which will be discussed in detail in \autoref{Sec:backgrounds}, the use of resampled simulation to calculate the \dllppi and \dllkpi PID efficiencies greatly simplifies the task

%% The poor replication of the \dllpk variable in resampled simulation also means that simulation cannot be used to train the BDT discussed in detail in \autoref{chap:BDT}. However, the efficiency calculations for selections involving the \dllkpi and \dllppi variables can still be computed using resampled simulation, as the \dllkpi and \dllppi variables are correctly modelled in resampled simulation.  Due to the complex mass-dependent selections on the PID variables, which will be discussed in detail in \autoref{Sec:backgrounds}, being able to use resampled simulation to calculate the \dllppi and \dllkpi PID efficiencies greatly simplifies the task.  A summary of how the efficiency on signal is calculated, for each PID variable, is shown in \autoref{tab:PIDmethods}.

%% \begin{table}[!ht]

%%   \centering
  
%%   \begin{tabular}{l c c}
%%       \hline
%%     PID variable & Efficiency calculated using:\\
%%     \hline
%%     \dllppi&Resampled simulation\\
%%        \dllkpi & Resampled simulation\\   
%%     \dllpk&PIDCalib efficiency values (using sim. to get the kinematic dist.)\\
%%     \hline
%%  \end{tabular}
%%  \caption{Approach used to calculate the PID efficiency, as a function of PID variable.}
%% \label{tab:PIDmethods}

%% \end{table}



\FloatBarrier
%%%%%%%%%%%%%%%%%%%%%%%%%%%%%%%%%%%%%%%%%%%%%%%%%%%%%%%%%%%%%%%%%%%%%%%%%%%%%%%%%%%%%%%%%%%%%%%%%%%%%%%%%
\section{Reflection backgrounds}
\label{Sec:backgrounds}
A reflection, or peaking, background is a decay with a mis-identified daughter. Despite the daughter mis-identification, such decays can accumulate in mass, causing a peaking structure. However, owing to baryon number conservation, for $B$ meson decays to form peaking backgrounds under the \Lbpi mass hypothesis, a kaon or pion must be mis-identified as a proton. The difference in mass between the true and false hypotheses is so large as to wash-out much of the peaking structure.

%% A reflection, or peaking, background is a decay with a mis-identified daughter which therefore peaks at the mass of its mother when the correct mass is assigned to the relevant daughter, but which appears within the mass region of interest for the signal channel when under the incorrect mass hypothesis. If the mis-identified daughter is given a mass hypothesis which is not greatly dissimilar from its true mass (e.g. swapping a kaon mass for a pion mass) then the background will peak even under the wrong mass hypothesis. This is not the case for any reflections coming from $B$ meson decays considered in this analysis. This is because the $B$ meson decays require one of the kaon or pion daughters to be mis-ided as a proton, rendering the difference in mass between the true and false hypothesis large enough such as to lose much of the peaking structure when the reflection background is reconstructed under the incorrect mass hypothesis. This is demonstrated in \autoref{chap:mass} in \autoref{Fig:jpsibd}.


In this section, the removal of $\B$ reflections, that is, decays of type $\B\to hh\mu\mu$ or $\B\to hh\jpsi$, where $h$ represents hadrons, is discussed in \autoref{Sec:refl} and the number of reflections remaining after these vetoes are applied is estimated. This is followed by an examination of the removal of \LbK reflections in \autoref{subsec:LbK}. Finally, there is a discussion of the removal of double mis-identified peaking backgrounds in \autoref{sec:double} and double mis-identified charmonium resonances in \autoref{sec:mismuon}.

To reduce the amount of background as much as possible and to make the background shape easier to model, all of the major reflection channels are vetoed. All cuts are applied to both the signal channel \Lbpi and the normalisation channel \Lbpijpsi identically.


\subsection[$B$ reflection vetoes]{$\mathbold{B}$ reflection vetoes}
\label{Sec:refl}
The $B$ reflections considered are:


\begin{itemize}
\item $\Bd\to(\Kp\to\proton)\pim\mumu$
\item $\Bd\to(\pip\to\proton)(\Km\to\pim)\mumu$
\item $\Bd\to(\pip\to\proton)\pim\mumu$
\item $\Bs\to(\pip\to\proton)\pim\mumu$
\item $\Bs\to(\Kp\to\proton)(\Km\to\pim)\mumu$

\end{itemize}
where, in each case, the dimuon pair are either non-resonant or from a \jpsi and the notation (X \to Y) denotes X as being the true ID of the particle and Y as the PID hypothesis used to reconstruct the event.% particle has been mis-identified as, resulting in its misclassification as a \Lbpi (\Lbpijpsi) signal candidate. 


All reflections are vetoed by changing the track's mass hypothesis and re-evaluating the relevant invariant mass. If an event is close to the relevant mother particle mass a harder PID cut is then applied to the event. For example, the mass of the proton is changed to the mass of a kaon for $\Bd\to\Kstarz(\to \Kp\pim)\mumu$ events, and events with mass, $m_{K\pi\mu\mu}$, close to that of the $\Bd$ mass, have a harder PID cut applied. These cuts to the daughter hadrons are referred to as mass-dependent PID cuts. The strength of the cut applied is dependent on the size of the contribution of the reflection background in question. For the case of the decay $\Bs\to\phi(\to\Kp\Km)\mumu$, a veto is placed around the $\phi$ mass. The regions in $B$ mass where harder PID cuts are applied are shown in \autoref{Fig:refl}, as indicated by the shaded regions.  These regions are detailed in \autoref{tab:refl}.
\begin{figure}[h!]
  \def\nh{0.7\textwidth}
  \centering
  \subfloat[]{\includegraphics[scale = 0.57]{figs/data_refl}\label{refl:1}} \\
  \subfloat[]{\includegraphics[scale = 0.57]{figs/MC_refl}\label{refl:3}}\hskip 0.04\textwidth
  \caption{Shape of all data candidates, prior to any PID selection and within 60\mevcc of the \Lb mass, under different mass hypotheses \protect\subref{refl:1}, and \Lbpi signal simulation under these same hypotheses, \protect\subref{refl:3}. The shaded regions indicated the areas within which a tighter PID selection is applied. The lower shaded region in mass corresponds to events falling near the \Bd mass and the higher region corresponds to events falling near the \Bs mass. The widths of these shaded regions are detailed in \autoref{tab:refl}.}
  \label{Fig:refl}
\end{figure}
\autoref{Fig:refl}\protect\subref{refl:1} shows data within $\pm$ 60\mevcc of the \Lb mass, under different mass hypotheses with only preselection cuts applied, but no PID criteria. \autoref{Fig:refl}\protect\subref{refl:3} shows \Lbpi signal simulation under these same mass hypotheses. The dominance of the \Bd\to\Kstarz\mumu channel\footnote{Where for the case of \autoref{Fig:refl}\protect\subref{refl:1} the dimuon pair can either be decaying via a \jpsi resonance or non-resonantly} is apparent in \autoref{Fig:refl}\protect\subref{refl:1}. The exact vetoes used for each reflection background are shown in \autoref{tab:refl}, along with the fraction of the relevant background that is removed by these cuts.

The effect of these mass-dependent PID cuts on the signal mass distribution when the daughters are given the correct mass hypothesis of \proton\pim\mumu is shown in \autoref{Fig:reflsig}. In \autoref{Fig:reflsig}\protect\subref{refl:2} the signal mass distribution is shown with and without the mass-dependent cuts, as listed in \autoref{tab:refl}, applied. \autoref{Fig:reflsig}\protect\subref{refl:3} shows the difference between the two histograms in \autoref{Fig:reflsig}\protect\subref{refl:2}, where each histogram has been normalised before taking the difference between them. The PID cuts in \autoref{tab:refl} are dependent on the mass of the daughters when one or more of the daughters are given an alternative mass hypothesis. Due to the large difference in mass between the proton and the mass of a pion or kaon, the shaded regions indicated in \autoref{Fig:refl}\protect\subref{refl:3}, when plotted under the signal mass hypothesis, are much wider. As a result, there is no strong dependence on the signal mass of the PID cuts in \autoref{tab:refl}. 
\begin{figure}[h!]
  \def\nh{0.7\textwidth}
  \centering
    \subfloat[]{\includegraphics[scale = 0.57]{figs/bMPID.pdf}\label{refl:2}}\hskip 0.04\textwidth\\
  \subfloat[]{\includegraphics[scale = 0.57]{figs/Bdiff.pdf}\label{refl:3}}\hskip 0.04\textwidth
  \caption{The signal mass distribution with and without the mass-dependent PID cuts, as listed in \autoref{tab:refl}\protect\subref{refl:2}, applied, \protect\subref{refl:2}. The difference between the two histograms in \protect\subref{refl:2} (when normalised), \protect\subref{refl:3}.} 
  \label{Fig:reflsig}
\end{figure}

\begin{table}[ht]
  \centering
 % \hspace*{-1cm}
  \begin{tabular}{l c c c c}
    \hline
    Daughters & Window/\mevcc & Cut & Rejection  & Efficiency for \\
    &&&& \Lbpi sim.\\
    \hline
    $ \Kp$ $\pim\mup\mun$ & 5246 -- 5330    & \proton $\dllpk>$ 17 & 0.97 & 0.90\\
    \hline
    \multirow{2}{*}{$\pip$ $\pim\mup\mun$} & 5247 -- 5329& \multirow{2}{*}{$\proton$ $\dllppi$ $> 5$}& \multirow{2}{*}{0.60} & \multirow{2}{*}{0.98}\\& 5348 -- 5406 &&&\\
    %$\Bs \pip$ $\pim\mup\mun$&   & $\proton$ $\dllppi$ $> 5$  &0.60 & 0.98\\

    \hline
    $\Kp$ $\Km\mup\mun$ & 5348 -- 5406 & $\pion$ $\dllkpi$ $<$-10 &  \multirow{2}{*}{0.99} & \multirow{2}{*}{0.96} \\
    %\hline
    \Kp\Km  &  $<$ 1025    & removed & & \\
    \hline

  \end{tabular}
  \caption{Vetoes used to reject reflections and their rejection rate,  calculated using \B\to\jpsi X simulation, along side their efficiency on \Lbpi simulation, relative to the initial PID cuts already placed.}\label{tab:refl}
\end{table}
The width of the windows within which harder PID cuts are applied were chosen by rerunning the whole analysis with different window widths and optimising using the Figure of Merit (\Gls{FOM}) $\frac{S}{\sqrt{S+B}}$, where $S$ is the expected number of signal events and $B$ is the expected number of background events.
%% The value of $S$ is estimated using \Lbpijpsi data, assuming that the branching fraction of \Lbpi is 100 times less than that of \Lbpijpsi. The value of $B$ is estimated by fitting the background either side of the signal region and extrapolating into the signal region, using a single exponential to model the background shape. The remaining background events after these mass-dependent PID vetoes have been applied, are largely combinatorial in nature. The residual reflection backgrounds are discussed in \autoref{sec:subsubref}.
Once the cuts in \autoref{tab:refl} have been applied, the number of remaining events in each reflection channel, after all selection has been applied, is estimated. This selection includes the mass-dependant PID cuts, the preselection and the application of the BDT to reduce background, as discussed in \autoref{chap:bdt}. The BDT cut used during the optimisation process for the reflection background selection criteria is, however, not the same as the nominal BDT cut used throughout the rest of this analysis. This is due to the BDT cut being optimised again after the decisions on the reflection selection criteria had been made. The BDT cut used during the optimisation of the reflection background selection criteria is 0.2 and, as such, is not dissimilar for the nominal BDT cut value.

To estimate the number of remaining events in each channel, a data-driven approach is used. PID cuts on the data are reversed, so as to favour a particular reflection background channel. All other cuts are left the same. For example, to deduce the yield of \Bd\to\Kp\pim\mup\mun the PID cut on the proton would be reversed to make it more kaon like. These cuts are referred to as reversed-PID cuts.

The reversed-PID cuts allow the background peaks to be fitted with the correct mass hypothesis. The initial signal yield is then taken from these fits and resampled simulation is used to calculate the efficiency that results from unfolding the effect of the reversed-PID cuts and applying the standard PID selection.

In \autoref{chap:bdt}, there is a discussion of the application of a BDT to data, the purpose of which is to reduce both combinatorial and reflection backgrounds. To estimate the number of remaining reflection backgrounds after the BDT has been applied, the BDT efficiency is deduced by applying the BDT to the relevant reflection background with the reverse-PID cut still applied. By comparing the yield from the fit to the relevant reflection background, before and after the BDT cut has been applied, the value of the BDT efficiency on the given reflection background can be estimated. Combining all this information allows an estimation of the remaining number of events for each reflection background, after all selection has been applied. This estimation process is carried out for the \jpsi reflection channels, where a large sample is available in the data. The ratio of branching fractions between the \jpsi mode and the equivalent \mumu mode is used to estimate the remaining number of events in the equivalent \mumu channel.
The resulting yields for the \jpsi reflection channels are insignificant for all background channels considered with the exception of the \BdToKpi\jpsi channel, the yield for which is $69\pm9$ events. %Here, the dominating error comes from deducing the efficiency of the BDT on \BdToKpi\jpsi data.

Assuming that the ratio of PID and selection efficiencies for the \BdToKpi\jpsi and \Bd\to\Kp\pim\mup\mun channels are the same as those between the \Lbpijpsi and \Lbpi channels and taking the ratio between the relevant branching fractions (as shown in \autoref{tab:bfcomp}) gives an expected yield for the \BdToKpi\mumu background in the \Lbpi channel of $\sim$0.2 events, which is considered negligibly small. 

In conclusion, all reflection channels are negligible after all selection has been applied, with the exception of $\Bd\to \Kp\pim\jpsi(\to\mumu)$. The $\Bd\to \Kp\pim\jpsi(\to\mumu)$ reflection background is initially added as a fit component to the final fit, with its yield constrained to the expected yield of $69\pm9$, as discussed in \autoref{Sec:MassFit}.% in \autoref{chap:mass}.



%% The number of events in each reflection channel present after the preselection are estimated by fitting to data under the relative mass hypothesis.

%% These numbers are shown in \autoref{tab:reflno}.
%% Fits to data are used to  
%% The number of events in each reflection channel present after the preselection are estimated by fitting to data under the relative mass hypothesis. These numbers are shown in \autoref{tab:reflno}.

%% \begin{table}[ht]
%%   \centering
%%   \hspace*{-1cm}
%%   \begin{tabular}{l c c c c}
%%     \hline
%%     Channel & Events remaining\\
%%     \hline
%%     $ \Bd\to\Kp$ $\pim\mup\mun$& 4543 $\pm$ 67\\
%%     $ \Bd\to\pip$ $\pim\mup\mun$& 225$\pm$23\\
%%     $ \Bs\to\Kp$ $\Km\mup\mun$& 2103 $\pm$ 45\\
%%     $ \Bs\to\pip$ $\pim\mup\mun$& 1251$\pm$ 36\\
%%     \hline
%%   \end{tabular}
%%   \caption{The estimated number of events in each reflection channel after preselection has been applied.}\label{tab:reflno}
%% \end{table}

%% The simulation used to calculate the efficiencies is phase space, and therefore the dihadron mass distribution,  $m_{p\pi}$, will not have any resonance structures from $N^{*}$ states. To quantify how much this affects the estimation of the efficiency of the $ \Kp\Km    < 1025$\mevcc  cut, the  $m_{p\pi}$ distribution in simulation is reweighted from that of \Lbpijpsi sWeighted data. The difference between the efficiency of the $ \Kp\Km    < 1025$\mevcc  veto, with and without $m_{p\pi}$ weighting applied, is 1.1\% of the unweighted efficiency value, which is negligibly small for the purpose of this exercise. The similarity between weighted and unweighted efficiencies is unsurprising as the lightest $N^{*}$ resonance lies at 1440\mevcc, and most events lying below 1025\mevcc in $m_{KK}$ space also lie below 1440\mevcc in $m_{p\pi}$ space. The effect that the mis-modelling of the dihadron mass spectrum in phase space simulation has on the total relative efficiency is discussed as a systematic in \autoref{sec:ppisys}.

%% The width of the windows were chosen by rerunning the whole analysis with different window widths and optimising using the FOM $\frac{S}{\sqrt{S+B}}$, where S is the number of signal events and B is the number of background events. The value of $S$ is estimated using \Lbpijpsi data, assuming that the branching fraction of \Lbpi is 100 times less than that of \Lbpijpsi. The value of $B$ is estimated by fitting the background either side of the signal region and extrapolating into the signal region, using a single exponential to model the background shape. The remaining background events after these mass-dependent PID vetoes have been applied, are largely combinatorial in nature. The residual reflection backgrounds are discussed in \autoref{sec:subsubref}.

%The cut strength was decided by comparing efficiency and rejection of signal and background using simulation.

%% The final regions in $B$ mass selected are shown in \autoref{Fig:refl}. \autoref{Fig:refl}\protect\subref{refl:1} shows data within $\pm$ 60\mevcc of the \Lb mass, under different mass hypotheses with only preselection cuts applied, but no PID criteria. \autoref{Fig:refl}\protect\subref{refl:2} shows \Lbpi signal simulation under these same mass hypotheses.
%% \hspace*{-4cm}
%% \begin{figure}[h!]
%%   \def\nh{0.7\textwidth}
%%   \centering
%%   \subfloat[]{\includegraphics[scale = 0.57]{figs/data_refl}\label{refl:1}} \\
%%   \subfloat[]{\includegraphics[scale = 0.57]{figs/MC_refl}\label{refl:2}}\hskip 0.04\textwidth
%%   \caption{Shape of all data candidates, prior to any PID selection and within 60\mevcc of the \Lb mass, under different mass hypotheses \protect\subref{refl:1} and \Lbpi signal simulation under these same hypotheses in \protect\subref{refl:2}.}
%%   \label{Fig:refl}
%% \end{figure}

%% The efficiencies of these cuts in data are discussed in \autoref{Sec:Eff}.
%% \FloatBarrier                               
%% \subsection{data-driven estimates of remaining $\mathbold{B}$ reflections}\label{sec:subsubref}
                               
%% To check how many reflections are left after all selection requirements, including the BDT, have been applied, a data-driven approach is followed. Firstly PID cuts on the data are reversed, so as to favour a particular background type. All other cuts are left the same. For example, to deduce the yield of \Bd\to\Kp\pim\mup\mun the PID cut on the proton would be reversed to make it more kaon like. These cuts are referred to as reversed-PID cuts.

%% The reversed-PID cuts allow the background peaks to be fitted with the correct mass hypothesis. The initial signal yield is then taken from these fits and resampled simulation is used to calculate the efficiency that results from unfolding the effect of the reversed-PID cuts and applying the standard PID selection. The BDT efficiency is deduced by applying the BDT to the relevant reflection background with the reverse-PID cut still applied. By comparing the yield from the fit to the relevant reflection background, before and after the BDT cut has been applied, the value of the BDT efficiency on the given reflection background can be estimated. Combining all this information allows an estimation of the remaining number of events for each reflection background, after all selection has been applied, to be made.


%% The mass fits to various background for the \jpsi channels can be seen in \autoref{fig:refljpsi}.

%% The resulting yields are insignificant for all background channels considered with the exception of the \BdToKpi\jpsi channel, the yield for which is $69\pm4$ events. Here, the dominating error comes from deducing the efficiency of the BDT on \BdToKpi\jpsi data.

%% Assuming that the PID and selection efficiencies for the \BdToKpi\jpsi and \Bd\to\Kp\pim\mup\mun channels are the same to first order (although the $q^{2}$ vetoes on the \mumu channel means the \mumu selection efficiency will be worse than in the \jpsi channel) and taking the ratio between the relevant branching fractions (as shown in \autoref{tab:bfcomp}) gives an expected \BdToKpi\mumu background in the \Lbpi channel of less than one event. 

%% The $\Bd\to\Kstarz(\to \Kp\pim)\mumu$ reflection background is added as a fit component to the final fit, with its yield allowed to float, as discussed in \autoref{Sec:MassFit}.

%% \begin{figure}[!t]\def\nh{0.33\textwidth}
  
%%   \centering
%%   \hspace*{-3cm}
%%   \subfloat[]{\includegraphics[height=\nh]{figs/refl_fits/nopull/KpluspiminusJpsi_tpad_e.pdf}\label{mumufits:1}} 
%%   \subfloat[]{\includegraphics[height=\nh]{figs/refl_fits/nopull/BdBs_pipiJpsi_tpad_e}\label{mumufits:2}}\\\hskip 0.04\textwidth
%%   \hspace*{-3cm}
%%   \subfloat[]{\includegraphics[height=\nh]{figs/refl_fits/nopull/KminuspiplusJpsi_tpad_e}\label{mumufits:3}}
%%   \subfloat[]{\includegraphics[height=\nh]{figs/refl_fits/nopull/KminusKplusJpsi_tpad_e}\label{mumufits:4}}\\\hskip 0.04\textwidth 
  
%%   \caption{Fits to data under reverse PID cuts for \BdToKpi\jpsi \protect\subref{mumufits:1}, \Bs\to\pip\pim\jpsi and \Bd\to\pip\pim\jpsi,\protect\subref{mumufits:2}, \Bd\to\pip\Km\jpsi \protect\subref{mumufits:3}, \BsToKK\jpsi \protect\subref{mumufits:4}.}
%%   \label{fig:refljpsi}
%% \end{figure}





\FloatBarrier
\subsection[Background from $\LbK$ and $\LbKjpsi$ decays]{Background from $\mathbold{\LbK}$ and $\mathbold{\LbKjpsi}$ decays}\label{subsec:LbK}

The Cabibbo-favoured modes \LbK and \LbKjpsi will form backgrounds for the signal and normalisation channels respectively. As for the $B$ reflection channels, the background from \LbK and \LbKjpsi is reduced by the application of mass-dependent PID cuts. The mass region in which a tighter PID cut of $\dllkpi<-15$ is applied (in addition to the nominal PID cut of $\dllkpi<-5$ applied to all data) is shown in \autoref{Fig:widerpk}, along with the signal simulation under the \LbK hypothesis. 

\begin{figure}[!h]\def\nh{0.3\textwidth}
  \centering
  \includegraphics[height = 5.5cm]{figs/wide_pk}
  \caption{\LbK simulation with the region in which tighter PID cuts are imposed indicated by the shading and \Lbpi simulation under the \LbK mass hypothesis.}
  \label{Fig:widerpk}
\end{figure}
\begin{figure}[!h]\def\nh{0.3\textwidth}
  \centering
  \includegraphics[height = 5.5cm]{figs/pkMDPIDC.pdf}
  \caption{The mass distribution of \Lbpi simulation with and without the mass-dependent PID cut to reduce \LbK contributions applied.}
  \label{Fig:widerpksig}
\end{figure}
In \autoref{Fig:widerpksig} the effect of the mass-dependent cuts applied to reduce the Cabibbo-favoured mode \LbK on the signal channel is shown via the comparison of the \Lbpi mass distribution in simulation with and without the relevant mass-dependent PID cut applied.

The number of remaining \LbK events, after the mass-dependent vetoes and all other selections have been applied, is again estimated using a data-driven method. The number of \LbK events present with a reverse-PID cut on the meson of \dllkpi$>$5, which favours the kaon hypothesis, is of order 300 events. The calibrated simulation is then used to deduce the number of \LbK events remaining once this cut is removed and the mass-dependent PID cut has been re-applied. This gives 1$\pm$1 residual \LbK events and retains $\sim$ 75\% of signal events. Given the low expected \LbK yield, the \LbK component is not added to the final signal fit, although the effect of not adding it is evaluated as systematic uncertainty in \autoref{Sec:Results}.%This mass-dependent PID cut is 74\% efficient on the \Lbpi signal mode.

%% The cut \dllkpi$<-15$ on \LbK resampled simulation is 0.5\% efficient, and the cut \dllkpi$<-5$ on \LbK resampled simulation is 1.2\% efficient. The simulation indicates that 98.5\% of \LbK events fall in the mass window indicated in \autoref{Fig:widerpk}. Combining these efficiencies gives 1$\pm$ 1 expected events in the \mumu channel, given the number of \LbK events observed in \autoref{Fig:optpointfits} in \autoref{Sec:BDT} (plot remains blinded). %The error on this number is dominated by the systematic associated with the calculation of the PID efficiencies, as discussed in \autoref{sec:pidsys}.

%% %pk number was 1.4\pm8

%% For the \Lbpi channel, the cut at $\dllkpi<-5$ is 77\% efficient and the cut at $\dllkpi<-15$ is 55\% efficient. Given that 13\% of \Lbpi events have the tighter PID cuts applied, this renders the total efficiency for the \dllkpi variable requirements on \Lbpi t<o be equal to 74\%.



The estimation of the number of remaining \LbKjpsi events is also approached using a data-driven method. To estimate the number of \LbKjpsi events, the known \LbKjpsi and \Lbpijpsi branching fractions are used, as shown in \autoref{tab:bfcomp}, along with the efficiencies of the \dllkpi selections on the \LbKjpsi and \Lbpijpsi channels (taken from calibrated simulation) and the number of observed \Lbpijpsi events in \autoref{Fig:optpointfits}. This gives the expected number of \LbKjpsi events as being 84~$\pm$~10. A \LbKjpsi component is therefore added to the \Lbpijpsi fit, as discussed in \autoref{Sec:MassFit}. % in \autoref{chap:mass}.



%The reason for again having a mass-dependent selection, as opposed to placing the requirements in \autoref{tab:muon} on all events, is to lower the number of signal events rejected.
%% \subsubsection{Estimating the amount of background from swaps between muons, pions and protons}
%% To estimate the yield expected from this mis-identification of charmonium resonances after the selection in \autoref{sec:mismuon} has been applied, the number of \Lbpijpsi events observed in data, along with \Lbpijpsi simulation, is used. The efficiency depends on the kinematics of the events in question. The simulation is used to deduce the kinematics of the decay and data calibration samples are used to estimate the efficiency on the \Lbpijpsi channel, on an event by event basis, when the pion (proton) has the isMuon==1 and $\dllmupi>-3$ requirements applied and the muon with the same (opposite) sign of the pion has the isMuon==0 and inMuon==1 requirements applied.  he average efficiency is 0.02\%, with a variation between $\sim$ 0\% and 0.08\%.

%% Given that there are 1051$\pm$32 \Lbpijpsi events observed in data, and assuming that the relative efficiency between the \Lbpi channel and the \Lbpijpsi channel is the same as the relative efficiency between the double mis-ided \Lbpijpsi channel and the \Lbpijpsi channel under the correct hypothesis, gives an expected yield for the double mis-ided \Lbpijpsi channel of 0.1 events.

%% The number of events from the $\psi(2S)$ channel are estimated by looking at the relative branching fraction between \Bd\to\Kstarz\jpsi(\mumu) and \Bd\to\Kstarz$\psi(\mumu)$, and assuming this will be similar for the relative branching fractions \Lb\to\proton\pim\jpsi(\mumu) and \Lb\to\proton\pim$\psi(\mumu)$. Based on these estimations the \jpsi channels will be $\sim$ 15-20 times larger than the $\psi(2S)$ channel and thus the contributions from the $\psi(2S)$ channel are considered negligible compared to the \jpsi channel.

%% Given that the \Lbpijpsi branching fraction is of order $10^{-6}$, an efficiency of $10^{-4}$ gives an effective branching fraction of $10^{-10}$. A conservative estimate for the \Lbpi branching fraction, assuming that it is 100 times smaller than the \Lbpijpsi branching fraction, is $\sim 10^{-8}$. This is 100 times larger than the expected contribution from the double mis-ided \Lbpijpsi channel. Consequently, no fit component is required for the double mis-ided \Lbpijpsi channel. As an additional post-unblinding check, the double swap contamination is further investigated by vetoing the $m_{\mup(\pim\to\mun)}$ spectrum around the \jpsi mass.  No significant effect is observed.


%%%%%%%%%%%%%%%%%%%%%%%%%%%%%%%%%%%%%%%%%%%%%%%%%%%%%%%%%%%%%%%%%%%%%%%%%%%%%%%%%%%%%%%%%%%%%%%%%%%%%%%%%
\subsection{Double mis-identification of muons as pions}
\label{sec:double}
A further source of background, with a double mis-id of the pions as muons, is from events of the form \Lb\to\proton\pim\pip\pim. The Feynman diagram for this process is shown in \autoref{fig:lbpipipi}. Due to the small mass difference between muons and pions, this background would peak around the \Lb mass under the signal mass hypothesis. 

\begin{figure}[!h]
  \centering
  %\includegraphics[scale = 0.3, trim = 0mm 180mm 0mm 300mm]{figs/scrppipi_br.png}
  \includegraphics[scale = 0.3, clip = true, trim = 0mm 190mm 0mm 0mm]{figs/scrppipi_br.png}
  \caption{The Feynman diagram for the decay \Lb\to\proton\pim\pip\pim.}
  \label{fig:lbpipipi}
  \end{figure}


In order to deduce the number of \Lb\to\proton\pim\pip\pim present after selection, a data-driven method is used to estimate the efficiency of the muon selection when placed on a true pion. This selection is namely $\mathrm{\gls{isMuon}==True}$ and $\dllmupi>-3$. The combined efficiency to mis-identify both pions as muons is found to be $\sim$ 0.01\%. There is no branching fraction measurement for \Lb\to\proton\pim\pip\pim. Instead, the $\Lb\to\proton\pi$ branching fraction is used as a proxy. This branching fraction is a suitable proxy for \Lb\to\proton\pim\pip\pim as $\Lb\to\proton\pi$ has a related Feynman diagram.  The $\Lb\to\proton\pi$ branching fraction is of order $10^{-6}$. Applying the muon selection efficiency gives an effective branching fraction of order $10^{-10}$.

%The efficiency of the selection will depend on the kinematics of the event and as such \Lbpi data events are used as a proxy for $\Lb\to\proton\pim\pip\pim$ events for the purposes of sampling the relevant kinematic region. 

Given that the \Lbpijpsi branching fraction is of order $10^{-6}$, the \Lbpi branching fraction, assuming that it is 100 times smaller than the \Lbpijpsi branching fraction, will be  $\sim 10^{-8}$. This is 100 times larger than the expected contribution from $\Lb\to\proton\pim\pip\pim$ after the muon selection criteria have been applied. Consequently, no fit component is required for this \Lb\to\proton\pim\pip\pim contribution.

More recently, the number of $\Lb\to\proton\pim\pip\pim$ decays in 3\invfb of LHCb data has been measured in Ref.\cite{Aaij:2016cla} and the observed number is compatible with the estimate of the branching fraction outlined above.

As an additional post-unblinding check, the double mis-id contamination is further investigated by checking the effect of tightening the selection on the $\dllmupi$ variable on the signal sample. No significant effect is observed.


%%%%%%%%%%%%%%%%%%%%%%%%%%%
\subsection{Mis-identification from swaps between muons and hadrons}
\label{sec:mismuon}

Charmonium resonances can have double mis-identification such that the muon with the same sign as the pion is misidentified as the pion, and the pion is misidentified as the muon. The same process can occur between the proton and the muon with the same sign as the proton. The possibility of these backgrounds is dealt with by tightening the selection on events where the invariant mass of the pion (proton) candidate and the muon of opposite sign to the pion (proton), form a mass close to the nominal \jpsi or \psitwos masses. The additional requirements placed on \Lbpi candidates to reduce swap backgrounds are shown in \autoref{tab:muon}. The extra \gls{inMuon} requirement reduces the rate of mis-identification, in order for the muon to have been mis-identified as a pion (proton) it has to fulfil the isMuon==False requirement, which it would do if it fell outside of the muon acceptance.  All pion and proton tracks are already required to have isMuon == False. The residual background from all swaps after these selections have been applied is found to be negligible at $\sim$ 0.1 events.

%% The largest swap background will be from \Lbpijpsi decays, as the \Lbpijpsi branching fractions is estimated to be $\sim$15-20 times larger than \Lb\to\psitwos\proton\pim. In addition, the rate from pion to muon swaps will be higher than for proton to pion swaps, as the similar muon and pion masses makes it more difficult to distinguish the two particle types. 

%% The additional requirements placed on \Lbpi candidates to reduce swap backgrounds are shown in \autoref{tab:muon}. The extra \gls{inMuon} requirement reduces the rate of mis-identification, in order for the muon to have been mis-identified as a pion (proton) it has to fulfil the isMuon==False requirement, which it would do if it fell outside of the muon acceptance (inMuon == False).  All pion and proton tracks are already required to have isMuon == False. The removal of pion misidentified events falling in the centre of the \jpsi mass peak is due to the higher background coming from pion and muon \Lbpijpsi swaps. The total efficiency on signal of the selection in \autoref{tab:muon} is (96.4$\pm$0.9)\%. The estimated level of background from all swaps after these selections have been applied is negligible.

\begin{table}
  \centering
  \begin{tabular}{ l | c }
    \hline
    Selection / ($\gev^{2}/c^{4}$) & Criteria \\
    \hline
    %$\;\,$9.0$<m^{2}_{\mup(\pim\to\mun)}<$10.0 & removed\\
    $\;\,$8.0$<m^{2}_{\mup(\pim\to\mun)}<$11.0 & $\pi$ inMuon$:$ True\\
    12.5$<m^{2}_{\mup(\pim\to\mun)}<$15.0 & $\pi$ inMuon$:$ True\\
    $\;\,$8.0$<m^{2}_{(\proton\to\mup)\mun}<$11.0 & $\proton$ inMuon$:$ True\\
    12.5$<m^{2}_{(\proton\to\mup)\mun}<$15.0 & $\proton$ inMuon$:$ True\\
    % $3535<m_{\pi , \mup}<3872$ & \pi inMuon: True\\

    \hline
  \end{tabular}
  \caption{The veto requirements used to reduce the background from charmonium resonances. The notation $m_{X(Y\to Z)}$ implies the combined invariant mass of the $X$ and $Z$ particle, where the $Y$ particle is given the mass of $Z$.}
  \label{tab:muon}
\end{table}
%%%%%%%%%%%%%%%%%%%%%%%%%%%%%%%%%%%%%%%%%%%%%%%%%%%%%%%%%%%%%%%%%%%%%%%%%%%%%%%%%%%%%%%%%%%%%%%%%%%%%%%%%%%%%%%%%%%%%%%%%%%%%%%%%%%%%%%%%%%%%%%%%%%%%%%%%%%%%%%%
\section[Partially reconstructed backgrounds in the $\Lbpi$ channel]{Partially reconstructed backgrounds in the $\mathbold{\Lbpi}$ channel}
\label{Sec:partreco}
The background in the \Lbpijpsi channel is largely combinatorial and as such the \Lbpijpsi mass distribution can be well-modelled by a single exponential. However, this is not the case for the \Lbpi channel, which has an additional contribution from \gls{partreco} backgrounds, as shown in \autoref{Fig:partreco}, alongside a fit to the mass distribution from the \LbK channel. All fits in this analysis are unbinned extended maximum-likelihood fits. The mass shapes are discussed in detail in \autoref{chap:mass}.
\begin{figure}[!ht]\def\nh{0.3\textwidth}
  \centering
%  \hspace*{-2cm}
  \subfloat[]{\includegraphics[height = 5.5cm]{figs/realblind.png}\label{optfit:1}}
  \subfloat[]{\includegraphics[height = 5.5cm]{figs/pkmumu_th.png}\label{optfit:3}} %pk_5100_7000_fit_blinded_edit.png
    \caption{Fitted mass distributions for blinded \Lbpi data \protect\subref{optfit:1} and \LbK data \protect\subref{optfit:3}. }
  \label{Fig:partreco}
\end{figure}



The background in the \Lbpi channel must originate from muons coming from different vertices, otherwise it would also be present in the \Lbpijpsi case. In addition, the distribution of the momentum of the positive muon with respect to the direction of the dimuon system is highly asymmetric in the lower mass side band of \Lbpi data, suggesting that the part-reco background component is dominated by semi-leptonic processes. %This argument will be discussed in more detail in \autoref{subsec:id}.
%% Semi-leptonic cascade decays which contain a pion mis-identified as a muon are discussed in \autoref{subsec:id} and decays where two pions are mis-id'd as muons are discussed in \autoref{sec:double}.

In this section the composition of the backgrounds in both the \Lbpi data and the \LbK data are considered. The \LbK background is studied because the backgrounds for both the \Lbpi and \LbK datasets will be similar but there are more statistics in the \LbK channel. The \LbK candidates are selected in the same way as the \Lbpi candidates but with the \dllkpi requirement on the kaon candidate changed from $\dllkpi<-5$ to $\dllkpi>5$ and the mass-dependent PID veto used to remove \LbK candidates not applied. %placed on \Lbpi data to remove \LbK candidates is lifted. 
%% \subsection{Mis-Identification of the proton or pion candidates}
%% \label{subsec:mish}


%% The good momentum resolution of the LHCb detector means that partially reconstructed events without any mis-id will have a mass less than the nominal \Lb mass. However, those with a lighter particle mis-reconstructed as the proton, get an increased invariant mass and might then contribute to the signal region. 

%% The possibility that the proton and pion candidates are other hadrons which are mis-identified is investigated by comparing the PID distribution of events with
%% $m_{\proton\pim\mumu}<5500\mevcc$ (referred to as the lower mass side band, lmsb) with that of a sample of well-identified protons  and pions. The proton and pion samples are taken from sWeighted \Lbpijpsi data and therefore show  \dllppi, \dllkpi and \dllpk distributions similar to those excepted for \Lbpi signal events. In addition, the same \dllkpi, \dllppi and \dllpk distributions for well identified kaon samples are compared. Mis-Identified kaons are a likely source of background. Therefore, the comparison of the lmsb PID distributions with those from kaon samples helps quantify how similar the PID distributions for \Lbpi events in the lmsb are to PID distributions from potential background events. The kaon samples are taken from sWeighted \BdToJPsiKst data for the \dllppi and \dllpk variables and \LbK simulation for the \dllkpi variable. All of the PID distributions discussed in this section are shown in \autoref{Fig:lmsbpid}. There is no strong evidence of any significant contamination of either proton or pion candidates. 



%% \begin{figure}[!t]\def\nh{0.33\textwidth}
%%   \centering
%%   \subfloat[]{\includegraphics[height=\nh]{figs/piminusPIDK_lmsb.pdf}\label{lmsbpid:1}}
%%   \subfloat[]{\includegraphics[height=\nh]{figs/pplusPIDpk_lmsb}\label{lmsbpid:2}}\hskip 0.04\textwidth\\
%%   \subfloat[]{\includegraphics[height=\nh]{figs/PIDp_lmsb}\label{lmsbpid:3}}
%%   \subfloat[]{\includegraphics[height=\nh]{figs/ANA_kaonPIDK.pdf}\label{lmsbpid:4}}\\
%%  \hspace*{-0.6cm}
%%   \subfloat[]{\includegraphics[scale = 0.45, trim = 0mm 0mm 9mm 0mm, clip = true]{figs/ANA_kaonPIDpK_data_fromKstjpsi.pdf}\label{lmsbpid:5}}
%%   \subfloat[]{\includegraphics[height=\nh]{figs/ANA_kaonPIDp_data_fromKstjpsi.pdf}\label{lmsbpid:6}}


  
%%   \caption{Comparison of PID distributions calculated with either \Lbpijpsi signal (sWeighted) or the lower mass side-band (indicated with lmsb in legend) of \Lbpi data, \protect\subref{lmsbpid:1}, \protect\subref{lmsbpid:2} and \protect\subref{lmsbpid:3}. The same PID variables as in \protect\subref{lmsbpid:1} - \protect\subref{lmsbpid:3} are shown in \protect\subref{lmsbpid:4} - \protect\subref{lmsbpid:6} for the case of the kaon. The datasets used to produce the distributions in \protect\subref{lmsbpid:4} - \protect\subref{lmsbpid:6} are shown in the legend.}
%%    \label{Fig:lmsbpid}
%% \end{figure}
%% \FloatBarrier
%%%%%%%%%%%%%%%%%%%%%%%%%%%%%%%%%%%%%%%%%%%%%%%%%%
%\subsection{Backgrounds from partially reconstructed semi-leptonic decays}
%\label{subsec:id}
To check for potential background arising from mis-identified particles, alternative mass hypotheses are tried in combinations of pairs and triplets for the muon, proton and pion (kaon) candidates in the region $m_{p\pi(\kaon)\mu\mu}<5500\mevcc$ in the \Lbpi (\LbK) datasets. The only narrow mass peak that is formed by trying all of the available combinations corresponds to cascade decays involving a \Lc. These decays are most clearly visible with the BDT selection relaxed, as shown in \autoref{Fig:Lcpeak}. The \Lc mass peaks observed in the \Lbpi and \LbK datasets come from the decays $\Lb \to \Lc (\to \proton \pip X) Y$ where $X = \pim$ or $\Km$, $Y=\mun\nu$ or $\pim$ and the pion with the same sign as the proton is mis-identified as a muon. 

\begin{figure}[!t]\def\nh{0.33\textwidth}

  \centering
  
  \subfloat[]{\includegraphics[height = 5.5cm]{figs/ppipi_th.pdf}\label{Lcpeak:1}}\hskip 0.04\textwidth\\
  \subfloat[]{\includegraphics[height = 5.5cm]{figs/lcpkpi}\label{Lcpeak:2}}\hskip 0.04\textwidth  

  
  \caption{ Combined mass of ($p\pim(\mup\to\pip)$) for \Lbpi lower mass side band data where the muon with the same sign as the proton has been given the mass of a pion \protect\subref{Lcpeak:1}, combined mass of $p\Km(\mup\to\pip)$ for \LbK lower mass side band data where the muon with the same sign as the proton has been given that of a pion \protect\subref{Lcpeak:2}. The peaks at $\sim$ 2300\mevcc are due to the $\Lc$ baryon appearing via $\Lb\to\Lc(p\Km(\pim)\pip)X$ decays.}
  \label{Fig:Lcpeak}  

\end{figure}
The background from decays with Y=$\mun\nu$ rather than Y= $\pim$ are expected to be dominant, as only one mis-id is required. This is verified by looking at the  mass, $m_{\proton\pi(\mu\to\pi)}$, of the \Lc candidates (now with the nominal BDT cut in place) against the mass, $m_{\proton\pi\mu\mu}$, of the \Lb candidates. These distributions are shown in \autoref{Fig:LcVsLb} for both $m_{\proton\pi(\mu\to\pi)}$ and $m_{\proton\kaon(\mu\to\pi)}$. All the \Lc candidates have $m_{\proton\pi\pi}$ significantly less than $m_{\Lb}$, suggestive of the large mass difference associated with the Y= $\mun\nu$ hypothesis rather than the Y= $\pim$ hypothesis. In the latter case, the small mass difference between pions and muons would give a value of $m_{\proton\pi\pi\pi}$ much closer to the \Lb mass. The full range in $m_{p\pi\mu\mu}$ is shown in the $x$ axis of \autoref{Fig:LcVsLb}\protect\subref{Lclbmass:2}. Note that this does not unblind the \Lbpi dataset, as the range in $m_{\proton\pi\pi}$ is restricted to less than 2850\mevcc. % to avoid unblinding \Lbpi signal events. %In these plots in \autoref{Fig:LcVsLb}, the opt
As a post-unblinding check against $\Lb \to \Lc (\to \proton \pip \pim) \pim$ decays, which would peak around the \Lb mass under the $m_{p\pi\mu\mu}$ hypothesis, the events with a value of $m_{p\pi(\mu\to\pi)}$ which fall around the nominal \Lc mass are removed and the effect on the \Lbpi signal peak evaluated. There is no significant effect observed. %There is no change to the  number of events taken from the fit to the \Lbpi signal after this veto is applied.


\begin{figure}[!t]\def\nh{0.33\textwidth}
  \centering
  \subfloat[]{\includegraphics[height = 5.5cm]{figs/pkpi_pkmumu_edit.pdf}\label{Lclbmass:1}}\hskip 0.04\textwidth
\subfloat[]{\includegraphics[height = 5.5cm]{figs/ppipi_ppimumu_edit.pdf}\label{Lclbmass:2}}
 
\caption{The mass of the combination $(p\kaon(\mup\to\pip))$ against the mass of the combination of $(p\kaon\mun\mup)$ for the $\LbK$ data set after the complete selection~\protect\subref{Lclbmass:1} and the mass of the combination $(p\pim(\mup\to\pip))$ against the mass of the combination of $(p\pim\mun\mup)$ for the $\Lbpi$ data set after the complete selection \protect\subref{Lclbmass:2}. The red lines in the $x$ and $y$ direction indicate the nominal \Lc and \Lb masses respectively.}
\label{Fig:LcVsLb}
\end{figure}


The assumption that the part-reco is semi-leptonic in nature is also supported by the fact that the $\cos(\theta_{ll})$ distribution, where $\theta_{ll}$ is defined as the angle between the momentum of the lepton with the same sign as the proton and the sum of the lepton momenta, both in the frame of the \Lb mother, is asymmetric. This is a signature for semileptonic cascade decays where one lepton is much harder than the other.  This is demonstrated for \LbK candidates in \autoref{Fig:Lclbmass}, which shows the $\cos(\theta_{ll})$ distribution, along with a comparison of the mass distribution with and without a cut on $\cos(\theta_{ll})$. Again, the \LbK data is used to demonstrate these features given the larger sample size. 


\begin{figure}[!t]\def\nh{0.33\textwidth}
  \centering
 
  \subfloat[]{\includegraphics[height=\nh]{figs/helicity_th.pdf}\label{Lclbmass:1}}\hskip 0.04\textwidth\\ %helicitypkmostrecent
  \subfloat[]{\includegraphics[height=\nh]{figs/pkcosthetall}\label{Lclbmass:2}}\hskip 0.04\textwidth  

  
  \caption{\LbK data with and without a cut on cos($\theta_{ll}$), where $\theta_{ll}$ is defined as the angle between the momentum of the lepton with the same sign as the proton and the sum of the lepton momenta, both in the frame of the \Lb mother, \protect\subref{Lclbmass:1}, the distribution of cos($\theta_{ll}$) for \LbK data \protect\subref{Lclbmass:2}.}
  \label{Fig:Lclbmass}
\end{figure}

In order to model the partially reconstructed backgrounds, the similarity between the \Lbpi and \LbK channels is exploited and the \LbK candidates are used to define the part-reco background shape for the fit to \Lbpi data. A comparison between the lower mass side band distributions for the \Lbpi and \LbK datasets is shown in \autoref{Fig:pipkvalidation}.

In summary, there is a background component in the lower mass side band of both \LbK and \Lbpi data, that is not present in the \Lbpijpsi data. The exact composition of this background is not deduced, although there are signs both that it is semi-leptonic and that it contains \Lc cascade decays. Instead, the background is fitted in the \Lbpi channel by exploiting the similarity between the \Lbpi and \LbK mass distributions. The effect of this assumption on the observed signal yield taken from the fit is considered as a systematic uncertainty, as discussed in \autoref{Sec:Results}.

%% In addition, the \dllkpi variable of the combined \Lbpi and \LbK datasets, which is used to distinguish \LbK events from \Lbpi events, is plotted against the $\Lb$ mass in \autoref{Fig:pipkvalidation}\protect\subref{pipkvalidation:2}, which shows that there is no strong correlation between this separating variable and the \Lb mass value.
\begin{figure}[!h]\def\nh{0.3\textwidth}
  
  \centering
  
  \includegraphics[height=\nh]{figs/pkmumupimumu_lmsb_comp}
  %% \subfloat[]{\includegraphics[height=\nh]{figs/PIDlbmumu_edit.pdf}\label{pipkvalidation:2}}\hskip 0.04\textwidth\\

  \caption{Comparison of the separately normalised \Lb mass distribution in the lower-mass side band for \LbK and \Lbpi data.}
\label{Fig:pipkvalidation}
\end{figure}
\FloatBarrier

%The fit shapes themselves are discussed in Section~\ref{Sec:MassFit}


\clearpage




 
