\chapter{Conclusions and outlook}

This thesis presents the first observation of a $b\to d$ transition in the baryon sector, via the decay \Lbpi. This comes 29 years after the first observation of a $b\to d$ transition in the meson sector by the ARGUS experiment which measured $\Bd-\overline{\Bd}$ mixing in 1987\cite{ARGUS}. The observation of \Lbpi opens up the possibility of using baryonic decays to investigate the modest tension observed \cite{vtdvts} between the value of the CKM element ratio $|\Vtd/\Vts|$ measured via either tree- or loop-level processes.

In the future, by combining the measurement of $\BF(\Lbpi)$ with that of $\BF(\LbK)$\footnote{The \LbK branching fraction is also being measured by LHCb but is not a public result at the time of writing.}, the value of $|\Vtd/\Vts|$ will be extracted using
\begin{equation}
  \frac{\BF(\Lbpi)}{\BF(\LbK)} = |^{}\frac{\Vtd}{\Vts}|^{2}f^{2},
  \label{eq:R}
\end{equation}
where $f^{2}$ is the ratio of the relevant form factors and Wilson coefficients, integrated over the relevant phase space. In order to extract the value of $|\Vtd/\Vts|$, the value of $f$ must be calculated by theorists. New measurements of $|\Vtd/\Vts|$ via different channels are important in resolving the the modest tension observed between tree and loop measurements of $|\Vtd/\Vts|$ as the value calculated from neutral $B$ mixing is currently theory-limited. %Therefore, experimental efforts to increase precision can be achieved through measuring $|\Vtd/\Vts|$ via alternative channels.

The \Lbpi decay was observed with a significance of $\SIG\sigma$ using 3 \invfb of LHCb data. The value of the \Lbpi branching fraction is found to be
\begin{equation}
  %\begin{split}
    \BF(\Lbpi)  = \BFV, %25 \times (1/0.505) \times (1/1051) \times \BF(\Lbpijpsi)\times\BF(\jpsi\to\mumu) \\
    %& = (7.3 \pm 1.8 \pm 1.2^{+1.39}_{-1.10})\times 10^{-8}\\
    %  &  = [ 25 \times 3.0  \pm \sqrt25\times3.0 \pm (0.165 25\times3.0)^{+0.19 25\times3.0}_{-0.15 25\times3.0}] \times 10^{-9},\\
  %\end{split}
\end{equation}
where the first error is the statistical uncertainty, the second is the systematic uncertainty and the third is the uncertainty on \BF(\Lbpijpsi). The fitted \Lbpi yield is $\Syield$.

There is no theory prediction for the \Lbpi branching fraction. However, assuming that the \Lb\to\proton\pim\jpsi(\to\mumu) branching fraction is $\sim$ 100 times larger than the \Lbpi branching fraction, as discussed in \autoref{sec:overview}, a value of \BF~(\Lbpi) of order $10^{-8}$ is expected.

%% The expected significance for the decay was deduced by fitting  a series of pseudo-experiments, generated with an expected signal yield of $n_{sig} = 9$. This value of $n_{sig}$ was calculated by fitting for the \LbK yield and assuming that the ratio of the branching fractions \BF(\Lbpi)/\BF(\LbK) is equal to $|\Vtd/\Vts|$ and that the reconstruction and selection efficiencies for both the \LbK and \Lbpi channels are identical. The expected significance was 2.8$\pm$1.2. This is consistent with the observed significance of 6.0$\sigma$ at the 2.6$\sigma$ level. Similarly the observed \Lbpi yield of 25$\pm$6 is consistent with 9 events at the 2.6$\sigma$ level.

This thesis also presents a novel technique to measure the downstream tracking efficiency of the process $\KS\to\pip\pim$ in both data and simulation. This technique has been important for analyses which feature $\KS$ decays to help calibrate the difference between simulation and data. Given this, the same technique can also be applied to decays from other long-lived particles such as $\Lz$ baryons.

%These results have been presented to the LHCb collaboration via both an internal note and Twiki page and the results have been used in a number of analyses. %The next stage of this efficiency work is to redo the analysis for both 2015 and 2016 data and to apply the same technique to \Lz\to\proton\pim decays.



%There were \Lbpi 25$\pm$6 events observed, which is consistent with the expected value of 9 at 2.6$\sigma$. Similary the expected significance 


%\Lbpijpsi(\to\mumu) branching fraction, which is 1.6\times10^{-6}, place the 




