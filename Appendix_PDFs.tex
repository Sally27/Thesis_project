\label{sec:PDFS}
\section{Crystal Ball Function}
\label{CB}
Crystal Ball (\Gls{CB}) function \cite{Skwarnicki:1986xj}
is usually used for fitting of signal mass peaks in the invariant mass distributions. The \gls{CB} function consists of Gaussian function (which usually describes mass peak) with a power-law tail below a certain threshold. Its PDF is defined as
\begin{equation}
  f(x; \alpha,n,\overline{\mu},\sigma) = N \cdot
  \begin{dcases}
    e^{-\frac{(x-\overline{\mu})^{2}}{2\sigma^{2}}},& \text{if } \frac{(x-\overline{\mu})}{\sigma}  >\alpha\\
    A \cdot\Big(B - \frac{(x-\overline{\mu})}{\sigma}\Big)^{-n}, & \text{otherwise}
  \end{dcases}
  %% $$where$$
  %% \mathcal{C} = e^{-\frac{1}{2}(\frac{\alpha-\overline{\mu}}{k})^{2}}e^{-\beta \alpha};  \beta = \frac{\overline{\mu} - \alpha}{k^{2}}.
  \label{Eq:CB}
\end{equation}
where $A, B$ and $N$ are all constants that depend on ${\alpha,n,\overline{\mu},\sigma}$ ensuring correct normalisation and continuity of the first derivative. Thus, if $\alpha$ is positive, the tail, $A\cdot \Big(B - \frac{(x-\overline{\mu})}{2\sigma}\Big)^{-n}$, will start below the mean, usually arising from the photon-radiating decay products (left tail) and vice versa for the case where $\alpha$ is negative, arising from non-Gaussian resolution effects (right tail).

If one has to deal with different per-event uncertainties on the mass, one way is to model this by a sum of two Crystal Ball functions, where then each uncertainty on the event, would correspond to sum of two delta functions. Hence, double-sided Crystall Ball is defined as a linear combination of $f(x; \alpha,n,\overline{\mu},\sigma)$:
\begin{equation}
	g(x; \alpha,n,\overline{\mu},\sigma, f_{cb}) = f_{cb} \cdot f(x; \alpha,n,\overline{\mu},\sigma) + (1-f_{cb})\cdot f(x; \alpha,n,\overline{\mu},\sigma).
\end{equation}


\section{Double-sided Ipatia Function}
\label{IP}
Generalisation of (double-sided) Crystal Ball function where per-event uncertainty is taken into account, known as (double-sided) Ipatia function,\cite{Santos:2013gra}. 
Hence it has the same number of parameters and is usually denoted as $I(m,\mu_{IP},\sigma_{IP},\lambda,\zeta,\beta,a_{1},n_{1},a_{2},n_{2})$.
%, is defined in the following way:

\section{\texttt{Rookeys} Function from \texttt{ROOFIT} Package}
\label{RK}
A non-parametric function that is composed of superposition of Gaussians with equal surface, but with different widths $\sigma$, which are established by data at a given point.
%\begin{equation}
%I(x,\mu_{IP},\sigma_{IP},\lambda,\zeta,\beta,a_{1},n_{1},a_{2},n_{2})
%\end{equation}



