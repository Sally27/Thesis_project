
\newglossaryentry{ID}
{
	name=ID,
		description={Probability of correctly identifying particle, given PID requirement}
}

\newglossaryentry{misID}
{
	name=misID,
		description={Probability of incorrectly identifying particle given PID requirement}
}

\newglossaryentry{LS1}
{
	name=LS1,
		description={Long Shutdown 1}
}


\newglossaryentry{ghost}
{
	name=ghost track,
		text=ghost,
		description={A ghost track has less than 70\% of its hits originating from a single particle} 
}
\newglossaryentry{DTF}
{
	name=DTF,
		description={Decay Tree Fitter} 
}
\newglossaryentry{bm}
{
	name=vertex $\mathbold{\chisq}$/ndof,
		text=vertex $\chisq/$ndof,
		description={The $\chi^{2}$ per degree of freedom for a vertex fit for the combination of a set of daughter tracks} 
}
\newglossaryentry{zval}
{
	name=$\mathbold{z_{\mathrm{valid}}}$,
		text=$z_{\mathrm{valid}}$,
		description={The region along the $z$-axis within which the long track efficiency values are deemed to be valid} 
}
\newglossaryentry{zref}
{
	name=$\mathbold{z_{\mathrm{ref}}}$,
		text=$z_{\mathrm{ref}}$,
		description={A bin in $z$, with a 10mm width, lying in the region of \gls{zval}} 
}
\newglossaryentry{zks}
{
	name=$\mathbold{z_{\KS}}$,
		text=$z_{\KS}$,
		description={The position along the $z$-axis past which no new \KS particles are created} 
}
\newglossaryentry{n}
{
	name=$\mathbold{N_{0}}$,
		text=$N_{0}$,
		description={The number of $\KS$ decays in a certain \gls{zref} bin after efficiency corrections have been applied}}
		\newglossaryentry{LHCb}
{
	name=LHCb,
		description={The Large Hadron Collider beauty experiment}
}
\newglossaryentry{SPD}
{
	name=SPD,
		description={Scintillator Pad Detectors}
}
\newglossaryentry{FOM}
{
	name=FOM,
		description={Figure of Merit}
}                                    
\newglossaryentry{PRS}
{
	name=PRS,
		description={pre-shower}
}                                    
\newglossaryentry{PSB}
{
	name=PSB,
		description={Proton Synchotron Booster}
}
\newglossaryentry{SPS}
{
	name=SPS,
		description={Super Proton Synchotron}
}
\newglossaryentry{PS}
{
	name=PS,
		description={Proton Synchotron}
}
\newglossaryentry{TIS}
{
	name=TIS,
		description={Events which are Triggered Independent of Signal}
}
\newglossaryentry{ECAL}
{
	name=ECAL,
		description={Electromagnetic calorimeter}
}
\newglossaryentry{HCAL}
{
	name=HCAL,
		description={hadronic calorimeter}
}                                    
\newglossaryentry{TISTOS}
{
	name=TISTOS,
		description={Events which require both the presence of signal and the rest of the event to fire the trigger}
}                                    
\newglossaryentry{TOS}
{
	name=TOS,
		description={Events which are Triggered On Signal}
}                                    
\newglossaryentry{GP}
{
	name=GP,
		description={General Purpose}                                 }                  
		\newglossaryentry{PID}
{
	name=PID,
		description={Particle IDentification} 
}                 
\newglossaryentry{q}
{
	name=$\mathbold{q^{2}}$,
		text=$q^{2}$,
		description={The four-momenta of the dimuon system}
}
\newglossaryentry{muonstation}
{
	name=M1-5,
		%name=muon station,
		description={The five muon stations}
}
\newglossaryentry{reflections}
{
	name=reflections,
		description={Decays coming from a $b$-hadron with a mis-identified particle that can accumulate at a certain $B$ or \Lb mass}
}

\newglossaryentry{MWPCs}
{
	name=MWPCs,
		description={multi-wire proportional chambers}
}


\newglossaryentry{FOI}
{
	name=FOI,
		description={Field of Interest}
}


\newglossaryentry{dllkpi}
{
	name=$\mathbold{DLL_{\kaon\pi}}$,
		text=$\dllkpi$,
		description={The difference in an event’s total likelihood, given the distribution of hits in the RICH, when the hypothesis of the track in question is changed from being that of a pion to a kaon
		}}
\newglossaryentry{dllpk}
{
	name=$\mathbold{DLL_{\proton\kaon}}$,
		text=$\dllkpi$,
		description={\dllppi-\dllkpi}}
		\newglossaryentry{isMuon}
{
	name=isMuon,
		description={A binary selection of muon candidates based on the penetration of the muon candidates through the calorimeters and iron filters}
}
\newglossaryentry{inMuon}
{
	name=inMuon,
		description={A binary selection of muon candidates based on whether or not the candidate fell inside the acceptance of the muon stations}
}                                        
\newglossaryentry{dllppi}
{
	name=$\mathbold{DLL_{\proton\pi}}$,
		text=$\dllppi$,
		description={The difference in an event’s total likelihood, given the distribution of hits in the RICH, when the hypothesis of the track in question is changed from being that of a pion to a proton}                
}
\newglossaryentry{dllmupi}
{
	name=$\mathbold{DLL_{\mu\pi}}$,
		text=$\dllmupi$,
		description={The difference in an event’s total likelihood, given the distribution of hits in the RICH, when the hypothesis of the track in question is changed from being that of a pion to a muon 
		}}
\newglossaryentry{partreco}
{
	name=part-reco,
		text=part-reco,
		description={Partially reconstructed background, backgrounds where a decay has been mis-identified as a signal candidate and not all the final states of the mis-identified decay are included in the final reconstruction  
		}}
\newglossaryentry{mppi}
{
	name=$\mathbold{m_{p\pi}}$,
		text=$m_{p\pi}$,
		description={The combined mass of the proton and the pion}
}
\newglossaryentry{mcorr}
{
	name=$\mathbold{m_{corr}}$,
		description={corrected mass, a function of the visible mass and the missing transverse-momentum for a decay which features unreconstructed track(s)}
}                 
\newglossaryentry{CB}
{
	name=CB,
		description={Crystal Ball function}
}                 
\newglossaryentry{DIRA}
{
	        name=Cos$(\mathbold{\theta_{B}})$,
                text=cos$({\theta_{B}})$,
		description={The cosine of the angle between the momentum vector of the $B^{+}$ meson and the direction of the flight of the $B^{+}$ meson from its primary vertex to its secondary vertex }
}
\newglossaryentry{PV}
{
	name=PV,
		description={Primary Vertex, the $pp$ interaction vertex}
}
\newglossaryentry{SV}
{
	name=SV,
		description={Secondary Vertex}
}
\newglossaryentry{FD}
{
	name=FD,
		description={Flight Distance, how far a particle flies before decaying}
}                                                                      
\newglossaryentry{DOCA}
{
	name=DOCA,
		description={The Distance of Closest Approach between two tracks} 
}
\newglossaryentry{OT}
{
	name=OT,
		description={Outer trackers, the outer section of the T stations} 
}
\newglossaryentry{IT}
{
	name=IT,
		description={Inner trackers, the inner section of the T stations} 
}

\newglossaryentry{pgh2}
{
	name=P$\mathbold{_{ghost}}$,%$\mathbold_{ghost}$,
		text=P$_{ghost}$,
		description={Ghost Probability is probability of misreconstruction of the track, where for each track 0 is most signal-like and 1 is most ghost-like. A charged particle is not considered to be a ghost if 70$\%$ of the hits match between the reconstructed and simulated true tracks. Similarly, neutral particles are ghosts if simulated particle contributes less than 50$\%$ of the reconstructed cluster energy from calorimeter} 
}                
\newglossaryentry{vertexchi2ndof}
{
	        name={Vertex $\mathbold{\chisq/\rm{ndof}}$},
		text={vertex ${\chisq/\rm{ndof}}$},
		description={The vertex $\chi^{2}$ per degree of freedom in a vertex fit} %sally}
		}
\newglossaryentry{trackchi2ndof}
{
	        name={Track $\mathbold{\chisq/\rm{ndof}}$},
		text={track $\chisq/\rm{ndof}$},
		description={The track $\chi^{2}$ pre degree of freedom is the minimal difference in fit $\chi^{2}$ (quality of the fit) to the primary vertex between fit with this track added  and removed} %sally}
		}
\newglossaryentry{minipchi2}
{
	        name={Min $\mathbold{\mathrm{IP}\chisq}$},
		text={min $\mathrm{IP}\chisq$},
		description={The minimum impact parameter $\chi^{2}$ is the minimal difference in fit $\chi^{2}$ (quality of the fit) to the primary vertex between fit with this track added  and removed} %sally}
		}
\newglossaryentry{ipchi2}
{
	name=$\mathbold{IP\chisq}$,
		text=$IP\chisq$,
		description={The $IP\chisq$ is the difference in the $\chisq$ of the fit to the primary vertex, when the track whose $IP\chisq$ is being measured is added and then removed}
} 
\newglossaryentry{fdchi2}
{
	        name=FD $\mathbold{\chisq}$,
		text=FD $\chisq$,
		description={The \Gls{FD} $\chisq$ is defined as the increase in $\chisq$ when the primary and secondary vertex are fitted separately as compared to single vertex fit}
}                
\newglossaryentry{DD}
{
	name=DD,
		description={Events where both daughter tracks are downstream track types}
}
\newglossaryentry{LL}
{
	name=LL,
		description={Events where both daughter tracks are long track types}
}
\newglossaryentry{SU}
{
	name=SU,
		description={Special Unitary group}
}                 
\newglossaryentry{EM}
{
	name=EM,
		description={Electromagnetism}
}                 
\newglossaryentry{NP}
{
	name=NP,
		description={New Physics}
}
\newglossaryentry{prompt}
{
	name=prompt decays,
		text=prompt,
		description={prompt decays are the decays of particles that were produced at the \Gls{PV}}
}

\newglossaryentry{MFV}
{
	name=MFV,
		description={Minimal Flavour Violation}
}
\newglossaryentry{FCNC}
{
	name=FCNC,
		description={Flavour Changing Neutral Currents. In the \Gls{SM} these are denoted $\Delta F = 2$, referring to the two internal $W$ boson vertices required}}
		%% \newglossaryentry{mcorr}
		%%                  {
			%%                    name=$m_{corr}$,
				%%                    description={corrected mass, a function of the visible mass and the missing transverse momentum for a decay which features unreconstructed track(s)}}
				\newglossaryentry{BBDT}
{
	name=BBDT,
		description={Bonsai BDT, a BDT used in the topological trigger lines which takes discrete input variables}}
		\newglossaryentry{BDT}
{
	name=BDT,
		description={Boosted Decision Tree, a BDT employs multivariate analysis techniques to combine a set of weakly discrimating variables into a single discrimating variable}}                   
		\newglossaryentry{HLT}
{
	name=HLT,
		description={High Level Trigger. The HLT is the software trigger which is applied after the \Gls{L0} trigger}}
		\newglossaryentry{L0}
{
	name=L0,
		description={Level-0 trigger. The L0 is the first trigger to be applied and uses hardware to make decisions on events  }}                   
		\newglossaryentry{IP}
{
	name=IP,
		description={Impact Parameter. The IP is defined as the distance between a track and the \Gls{PV} at the track's closest point of approach }}                                                                        
		\newglossaryentry{RICH}
{
	name=RICH,
		description={Ring Imaging Cherenkov detectors, provide particle identification by using Cherenkov radiation}                         
}

		\newglossaryentry{RICH2}
{
	name=RICH2,
		description={Ring Imaging Cherenkov providing high momentum \gls{PID} by using Čerenkov radiation}                         
}
		\newglossaryentry{RICH1}
{
	name=RICH1,
		description={Ring Imaging Cherenkov providing low momentum \gls{PID} by using Čerenkov radiation}                         
}
\newglossaryentry{HPD}
{
	name=HPD,
		description={Photomultiplier tubes that collect Čerenkov light}
}

\newglossaryentry{ATLAS}
{
	name=ATLAS,
		description={A Toroidal LHC ApparatuS}
}
\newglossaryentry{CMS}
{
	name=CMS,
		description={Compact Muon Solenoid}
}
\newglossaryentry{ALICE}
{
	name=ALICE,
		description={A Large Ion Collider Experiment}
}

\newglossaryentry{LHC}
{
	name=LHC,
		description={Large Hadron Collider}
}
\newglossaryentry{VELO}
{
	name=VELO,
		description={VErtex LOcator. Subdetector of LHCb, placed around the $pp$ interaction point, used to realise the precise measurements of vertices and tracks}
}

\newglossaryentry{TT}
{
	name=TT,
		description={The tracking station upstream of the magnet composed of silicon micro-strips.}
}


\newglossaryentry{tstation}
{
	name=t1{,} t2 and t3,
		description={trackers downstream of the magnet composed of silicon micro-strips strips in the inner section and straw tubes in the outer section.}
}

\newglossaryentry{HLT1}
{
	name=HLT1,
		description={First stage of high level trigger}
}

\newglossaryentry{HLT2}
{
	name=HLT2,
		description={Second stage of high level trigger}
}
\newglossaryentry{Tstation}
{
	name=T1{,} T2 and T3,
		description={Trackers downstream of the magnet composed of silicon micro-strips strips in the inner section and straw tubes in the outer section.}
}

\newglossaryentry{MC}
{
	name=MC,
		description={Monte Carlo Simulation}
}

\newglossaryentry{longtrack}
{
	name=long track ,
		description={Long track is track category which classifies tracks that have hits in the VELO and the T stations. Hits in the TT stations are optional}
}
\newglossaryentry{downstreamtracks}
{
	name=downstream track ,
		description={Downstream tracks have no VELO track segment. They are reconstructed using hits in the T and TT stations}
}
\newglossaryentry{SM}
{      
	name = SM,
	     description = {Standard Model}     
}
\newglossaryentry{QED}
{      
	name = QED,
	     description = {Quantum Electrodynamics}     
}
\newglossaryentry{QCD}
{      
	name = QCD,
	     description = {Quantum Chromodynamics}     
}
\newglossaryentry{BSM}
{      
	name = BSM,
	     description = {Beyond the Standard Model}     
}
