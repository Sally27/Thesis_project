\chapter{Introduction}
%The Standard Model is without question the most powerful and tested theory of particle physics. It describes and predicts many phenomena very well even though the theory is not including any explanation for the nature of dark matter and it doesn't make any attempt to describe gravity in a quantum field theory framework.

%Furthermore, fine-tuning of some parameters in the Standard Model such as the Higgs mass, where parameters get exactly the right value to produce required behaviour, beg questions if there is some symmetry in the model building that is missing. In this chapter, the theoretical basis of the Standard Model is discussed and is then followed by experimental and theoretical consideration of fully leptonic decays.


The field of particle physics aims to describe the universe we see today by decomposing everything into fundamental building blocks, which then exhibit certain behaviour according to a given set of rules. So far, the best theoretical formulation that describes the universe around us in form of these building blocks, the Standard Model (\Gls{SM}), was conceived last century. Some achievements of the \gls{SM} do really leave us breathless, with agreement between theoretical and experimental results of ten parts in a billion. 

This theory is, however, incomplete as it fails to address several issues. The theory does not include any explanation for the nature of dark matter and it doesn't make any attempt to describe gravity in a quantum field theory framework. Furthermore, fine-tuning of some parameters in the \gls{SM} such as the Higgs mass, where parameters get exactly the right value to produce required behaviour, beg questions if there is some symmetry in the model building that is missing. Lastly, as with any model, the \gls{SM} operates with many free parameters that need to be plugged in so that predictions can be made. So why are there exactly so many?

This thesis describes a search for a decay which can help to shine light on some of these parameters and is organised as follows. In~\autoref{stheory} the \gls{SM} of particle physics is discussed together with the theoretical and experimental motivation for fully leptonic decays, especially for the \Bmumumu decay. In~\autoref{chap:dec} the tool to search for \Bmumumu decays, the LHCb detector, is detailed. Discussion about how a trimuon signature behaves in the detector is covered in~\autoref{chap:trimuon}. The analysis of \Bmumumu, the central theme for the thesis, is then described in three chapters:~\autoref{chap:sel}, where the selection for signal and normalisation are given;~\autoref{chap:back}, where backgrounds to \Bmumumu are considered; and finally~\autoref{chap:masandef}, where the efficiencies and mass fits are discussed. The result, along with its implications, will then close this thesis in~\autoref{chap:Results}.

