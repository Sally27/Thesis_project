%%%%%%%%%%%%%%%%%%%%%%%%%%%%%%%%%%%%%%%%%%%%%%%%%%%%%%%%%%%%%%%%%%%%%%%%%%%%%%%%%%%%%%%%%%%%%
\section{Fit with the \Xib included}\label{Sec:Xib}
%%%%%%%%%%%%%%%%%%%%%%%%%%%%%%%%%%%%%%%%%%%%%%%%%%%%%%%%%%%%%%%%%%%%%%%%%%%%%%%%%%%%%%%%%%%%%%
\begin{figure}[t]\twoplots
\subfloat[\PsipK]{\includegraphics[width=\nw]{figs/TC_PsipKFit-Xib.png}\label{Fig:Fit:Xib}} 
\caption{\PsipK fit with the \Xib component included.}
\label{Fig:XibFit}
% PsipKFit : plotted PsippiMass with colour 2
% PsipKFit : plotted PsipiKMass with colour 3
% PsipKFit : plotted PsiKpiMass with colour 4
% PsipKFit : plotted PsiKKMass with colour 6
% PsipKFit : plotted PsiKpMass with colour 7
\end{figure}
%%%%%%%%%%%%%%%%%%%%%%%%%%%%%%%%%%%%%%%%%%%%%%%%%%%%%%%%%%%%%%%%%%%%%%%%%%%%%%%%%%%%%%%%%%%%%%
Fig.~\ref{Fig:Fit:Xib} shows the \PsipK fit with an extended mass range including the \Xib.
The peak is not obviously significant ($27\pm13$ events) but could be improved so by
vetoing the peaking background instead of modelling them and by re-optimising the NN cut
for a \Xib\to\PsipK observation. The \LbK yield becomes $24765 \pm 231$ in this fit
(i.e. 104 more candidates). 
