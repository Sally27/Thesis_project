\chapter*{Abstract}
\noindent
This thesis presents a first search for the fully leptonic decay \Bmumumu in any experiment. This search is performed using
proton-proton collision data at LHCb corresponding to an integrated luminosity of $4.7$ fb$^{-1}$. The search is carried out in the
region where the minimum of the two $\mumu$ mass combinations is below $980$\mevcc. The measurement of the branching fraction of this decay is even more interesting given that the recent theoretical prediction \cite{Danilina:2018uzr} of the branching fraction for \Bmumumu of $1.3 \times 10^{-7}$ is high. Moreover, this decay is sensitive to the magnitude of the coupling strength between $b$ and $u$ quarks, which is of great interest given that there are some tensions in measurements of this magnitude.

The data are consistent with the background only hypothesis and a limit of $1.4 \times 10^{-8}$ at 95\% confidence level is set on the branching fraction in the stated kinematic region. This is therefore not consistent with the theoretical prediction made in Ref. \cite{Danilina:2018uzr}.

This thesis also presents a study of the response of the detector if three muons pass through it. This study shows that correlations induced by a trimuon system in the detector are substantial and they need to be addressed properly.


%reports the branching fraction measurement of the rare Cabibbo-suppressed decay \Lbpi. The decay is observed for the first time with a $\SIG\sigma$ deviation from the background-only hypothesis. This is the first observation of a $b$\to$d$ quark transition in the baryon sector. The dataset used for the measurement corresponds to 3\:\invfb of \proton{}\proton collisions collected at the LHCb experiment at CERN. The branching fraction is measured using \Lb\to\jpsi(\to\mumu)\proton\pim as a normalisation channel and is measured as
%\begin{equation*}
%  \BF(\Lbpi) = \BFV,
%\end{equation*}
%where the first error is the statistical uncertainty, the second is the systematic uncertainty and the third is the uncertainty on $\BF(\Lbpijpsi)$.
%The measurement of $\BF(\Lbpi)$ can be combined with the branching fraction measurement for \LbK to give constraints on the ratio of CKM matrix elements $|\frac{\Vtd}{\Vts}|$. Such a determination of $|\frac{\Vtd}{\Vts}|$  requires a theory prediction for the ratio of the relevant form factors.

%This thesis also reports the ratio of tracking efficiencies, $\epsilon_{\mathrm{rel}}$, between data and simulation for $\KS\to\pip\pim$ decays occurring within the LHCb detector acceptance. As \KS particles are long-lived, their associated tracking efficiencies are less precisely determined compared to those of shorter-lived particles. The average value of $\epsilon_{\mathrm{rel}}$ for $\KS\to\pip\pim$ decays, where the \KS has a flight distance of $\gsim 1\m$, is found to be
%\begin{equation*}
%  \epsilon_{\mathrm{rel}} = 0.70\pm0.02.
%\end{equation*}

% To perform this calibration measurement a novel technique was developed which has the potential to be used in measuring the value of $\epsilon_{\mathrm{rel}}$ for other decays involving long-lived particles.

%The value of $\epsilon_{\mathrm{rel}}$ is found to be weakly dependent on the kinematics of the events as well as the particles decay position in the detector. 
