\chapter{Efficiencies}

\textit{ To be able to translate observed signal events into branching fraction estimate,  the normalisation channel
of $B^{+} \rightarrow (J/\psi\rightarrow\mu^{+}\mu^{-})K^{+}$ is used. Both, for signal and normalisation
channel the absolute efficiencies, luminosity, the b-quark cross-section or fragmentation fractions will
cancel. There are, however, efficiencies that will not cancel and will be necessary for the final limit setting procedure. In this section, methods of obtaining efficiencies of selection for normalisation and signal channel are described as well as efficiencies themselves.}

\section{Efficiency Ratio}
\label{EfficiencyRatio}

As this measurement is performed in a particular min$q^{2}$ region, all signal efficiencies are calculated with the min$q^{2}$ selection imposed. 

Overall selection efficiency for signal, $\varepsilon_{s}$, and normalisation, $\varepsilon_{n}$, includes contributions from the detector acceptance efficiency labelled (GEN); the reconstruction selection efficiency (REC); the offline selection efficiency (OFF) comprising of trigger, $J/\psi$ and $\Psi(2S)$ veto, MVA based selection efficiency (CombiBDT and MisidBDT); fitting region selection efficiency (fitrange); the efficiency of the PID requirement (PID). 

As the full simulation (\textit{generator-level+detector}) has no particular $minq^2$ selection imposed before reconstruction, the first two efficiencies will be obtained using privately generated \textit{generator-level} simulation. The summary of method used to extract signal efficiency is shown in Table ~\ref{tab:signaleffsummary}. For normalisation channel, there is no $minq^2$ region selection and hence full simulation can be used everywhere.

\begin{table}[H]
\centering
\small
\hspace*{-0.5cm}\begin{tabular}{| l | l |}
\hline
Component & Method  \\ \hline
$\varepsilon^{GEN}$, $\varepsilon^{REC}$ & \textit{generator-level}   \\
$\varepsilon^{TRG}$, $\varepsilon^{OFF}$, $\varepsilon^{BDTs}$, $\varepsilon^{fitrange}$   & \textit{generator-level+detector} \\
$\varepsilon^{PID}$ & Using \texttt{PIDCalib} \\
\hline
 \end{tabular}
 \caption{Method of obtaining efficiencies. Most of these efficiencies are evaluated using simulation, however, TRG and PID efficiencies are evaluted using data and/or simulation techniques.}
\label{tab:signaleffsummary}
\end{table}

The first two efficiencies for signal and normalisation are obtained using privately generated simulation from \autoref{tab:MCPPass}

\begin{equation}
{\varepsilon^{GEN,minq^{2}}}\times {\varepsilon^{REC,minq^{2}}}= \frac{N^{in\_acc,minq^{2}}}{N^{generated,minq^{2}}}\times \frac{N^{REC,minq^{2}}}{N^{in\_acc,minq^{2}}}
\end{equation}

\begin{equation}
N^{in\_acc,minq^{2}} = N^{in\_acc} \times \varepsilon_{minq^{2}}
\label{eq:number}
\end{equation}

In Equation ~\ref{eq:number}, $\varepsilon_{minq^{2}}$ is obtained by dividing number of generated events in \textit{generator-level} simulation (mentioned in \autoref{tab:MCPPass}) with minq$^2$ condition imposed, $N^{generated,minq^{2}}$, to total number of generated events, $N^{generated}$. $N^{in\_acc}$ is the number of events in \textit{generator-level+detector} simulation before reconstruction, $N^{REC,minq^{2}}$ is the number of events after reconstruction with minq$^2$ condition.

Hence the relative efficiency, $R^{\{21,26\}}(\varepsilon)$, is calculated 

%\hspace*{-1.0cm}\begin{equation}
%R^{\{21,26\}}_{\{NOFCME\}}(\varepsilon)=\frac{\varepsilon_{s}}{\varepsilon_{n}}=\frac{\varepsilon^{GEN}_{s}}{\varepsilon^{GEN}_{n}} \times \frac{\varepsilon^{REC}_{s}}{\varepsilon^{REC}_{n}} \times \frac{\varepsilon^{TRG}_{s}}{\varepsilon^{TRG}_{n}} \times \frac{\varepsilon^{OFF}_{s}}{\varepsilon^{OFF}_{n}} \times \frac{\varepsilon^{CombiBDT}_{s}}{\varepsilon^{CombiBDT}_{n}} \times \frac{\varepsilon^{MisidBDT}_{s}}{\varepsilon^{MisidBDT}_{n}} \times \frac{\varepsilon^{fitrange}_{s}}{\varepsilon^{fitrange}_{n}} \times \frac{\varepsilon^{PID}_{s}}{\varepsilon^{PID}_{n}},
%\label{eq:notsplitted}
%\end{equation}

\hspace*{-1.0cm}\begin{equation}
R^{\{21,26\}}(\varepsilon)=\frac{\varepsilon_{s}}{\varepsilon_{n}}=\frac{\varepsilon^{GEN}_{s}}{\varepsilon^{GEN}_{n}} \times \frac{\varepsilon^{REC}_{s}}{\varepsilon^{REC}_{n}} \times \frac{\varepsilon^{TRG}_{s}}{\varepsilon^{TRG}_{n}} \times \frac{\varepsilon^{OFF}_{s}}{\varepsilon^{OFF}_{n}} \times \frac{\varepsilon^{CombiBDT}_{s}}{\varepsilon^{CombiBDT}_{n}} \times \frac{\varepsilon^{MisidBDT}_{s}}{\varepsilon^{MisidBDT}_{n}} \times \frac{\varepsilon^{fitrange}_{s}}{\varepsilon^{fitrange}_{n}} \times \frac{\varepsilon^{PID}_{s}}{\varepsilon^{PID}_{n}},
\label{eq:notsplitted}
\end{equation}

%\hspace*{-1.0cm}\begin{equation}
%\hspace*{-2.0cm}R^{\{21,26\}}_{\{lowFCME,highFCME\}}(\varepsilon)=\frac{\varepsilon_{s}}{\varepsilon_{n}}=\frac{\varepsilon^{GEN}_{s}}{\varepsilon^{GEN}_{n}} \times \frac{\varepsilon^{REC}_{s}}{\varepsilon^{REC}_{n}} \times \frac{\varepsilon^{TRG}_{s}}{\varepsilon^{TRG}_{n}} \times \frac{\varepsilon^{OFF}_{s}}{\varepsilon^{OFF}_{n}} \times \frac{\varepsilon^{CombiBDT}_{s}}{\varepsilon^{CombiBDT}_{n}} \times \frac{\varepsilon^{MisidBDT}_{s}}{\varepsilon^{MisidBDT}_{n}} \times \frac{\varepsilon^{fitrange}_{s}}{\varepsilon^{fitrange}_{n}} \ \frac{\varepsilon^{FCME}_{s}}{\varepsilon^{FCME}_{n}} \times \frac{\varepsilon^{PID}_{s}}{\varepsilon^{PID}_{n}},
%\label{eq:splitted}
%\end{equation}

%where NOFCME means one bin of fractional corrected mass error, lowFCME and highFCME fractional corrected mass error are two bins of fractional corrected mass error, see Section ~\ref{split} for more details. 

 where 21, 26 denote the stripping version for Run \Rn{1} and Run \Rn{2}.
%As seen in equations ~\ref{eq:notsplitted} and ~\ref{eq:splitted}, absolute selection efficiencies $\varepsilon_{s}$, $\varepsilon_{n}$ have several components.
%\newline Overall selection efficiency includes contribution from the detector acceptance efficiency labelled (GEN);
%the reconstruction selection efficiency (REC); the efficiency of the offline selection (OFF) comprising of trigger, $J/\psi$ and $\Psi(2S)$ veto, MVA based selection (CombiBDT and MisidBDT); fitting region selection (fitrange); the efficiency of the PID requirement (PID). Individual  

%Since this analysis will be perfomed in two bins of FCME, and in two Stripping version there will be 4 efficiency ratios.  Most of these efficiencies are evaluated using MC, however, TRG and PID efficiencies will be evaluted using data and/or MC techniques. Values for different efficiencies and exact method of obtaining them is described in the following subsections.

\subsection{Detector Acceptance Efficiency (GEN)}
For charged particles detector acceptance efficiency describes the fraction of decays contained in the polar angle region of [10, 400] mrad. For neutral particles, the corresponding angular region is [0, 400] mrad. 

        \begin{table}[H]
                \begin{center}
        \begin{tabular}{l c c }

%        \hline
                Channel & Year & $\varepsilon_{gen}$\\ \hline
              %  $B^{+} \rightarrow J/\psi K^{+}$ & 2011 &  Sim08, Pyth6 & N/A & N/A & \\
              %  $B^{+} \rightarrow J/\psi K^{+}$ & 2011 &  Sim08, Pyth8 & 0.16379 $\pm$ 0.00030 & 0.16366 $\pm$ 0.00030 & 0.16372$\pm$0.00021 \\
                $B^{+} \rightarrow \mu^{+} \mu^{-} \mu^{+} \nu$ & 2012 & 0.1856$\pm$0.0011 \\
%                $B^{+} \rightarrow \mu^{+} \mu^{-} \mu^{+} \nu$ & 2012 & \\
                $B^{+} \rightarrow \mu^{+} \mu^{-} \mu^{+} \nu$ & 2016 & 0.1959$\pm$0.0016 \\ \hline

                $B^{+} \rightarrow J/\psi K^{+}$ & 2012 & 0.1622$\pm$0.0002\\
                %$B^{+} \rightarrow J/\psi K^{+}$ & 2012 &  \\
                %$B^{+} \rightarrow J/\psi K^{+}$ & 2015 &  Sim09a, Pyth8 & 0.17304 $\pm$ 0.00048 & 0.17386 $\pm$ 0.00047 &0.17346$\pm$0.00034 \\
                $B^{+} \rightarrow J/\psi K^{+}$ & 2016 &  0.1739$\pm$0.0004  \\
                \hline
%               $B^{+} \rightarrow \mu^{+} \mu^{-} \mu^{+} \nu$ & 2012 &  Sim08, Pyth6 &0.1828$\pm$0.0015 & \multirow{2}{*}{0.1856$\pm$0.0011} \\
%                $B^{+} \rightarrow \mu^{+} \mu^{-} \mu^{+} \nu$ & 2012 &  Sim08, Pyth8 &0.1884$\pm$0.0015 & \\
%                $B^{+} \rightarrow \mu^{+} \mu^{-} \mu^{+} \nu$ & 2016 &  Sim09b, Pyth8 &0.1959$\pm$0.0016 & 0.1959$\pm$0.0016 \\
              %  $B^{+} \rightarrow J/\psi \pi^{+}$ & 2012 &  Sim08, Pyth6 & 0.1519 $\pm$ 0.000400 & N/A & \multirow{2}{*}{0.15816$\pm$0.00024}\\
              %  $B^{+} \rightarrow J/\psi \pi^{+}$ & 2012 &  Sim08, Pyth8 & 0.1622 $\pm$ 0.000424 & 0.1611 $\pm$ 0.000421 &  \\
%               \hline
              %  $B^{0} \rightarrow J/\psi K^{*}$ & 2012 &  Sim08, Pyth8 & 0.16141 $\pm$ 0.00043 & 0.16050 $\pm$ 0.00043 &0.16095$\pm$0.00030 \\
%                \hline
              %  PartReco & 2012 & Sim08, Pyth8 & 0.16058 $\pm$ 0.00051 & 0.16058 $\pm$ 0.00050 & 0.1606$\pm$0.0004 \\
%                \hline
                \end{tabular}
        \end{center}
        \caption{Geometrical detector acceptance efficiencies for signal and normalisation channel. For 2012 and 2016 simulation samples, the overall detector acceptance efficiency will be the average for two possible magnetic polarity conditions: down, up. For 2012 this will be also averaged with two different simulation versions: Pythia 6.4\cite{pythia6} and Pythia 8.1\cite{pythia8}.}%  Efficiencies were calculated by generating statistics tables generated using dirac-bookeeping-prod4path, looking at generator level cut efficiency: /afs/cern.ch/user/s/slstefko/cmtuser/stattables.}
        \label{tab:MCdeteff}
        \end{table}

\subsection{Reconstruction Efficiency (REC)}
The reconstruction efficiency is calculated on simulated events which have passed the detector acceptance. For signal, this efficiency consists of reconstruction, stripping. For normalisation it consists from reconstruction, stripping, \textbf{and on the top} signal stripping is applied. This is done so that selections in normalisation and signal channel are kept as similar as possible and the fact that the signal selection has tighter cuts as explained in \autoref{nchannel}.

However, it should be noted that reconstruction efficiency reflects stripping selection \textbf{without the PID cuts} for both signal and normalisation. This is because \gls{PID} is badly modelled in simulation and hence will be accounted for separately.

  \begin{table}[H]
                \begin{center}
        \begin{tabular}{l c c }

        \hline
		Channel & Year & $\varepsilon_{rec|sel}$ \\ \hline
                $B^{+} \rightarrow \mu^{+} \mu^{-} \mu^{+} \nu$ & 2012 &  0.10841$\pm$0.00030 \\
                $B^{+} \rightarrow \mu^{+} \mu^{-} \mu^{+} \nu$ & 2016 &  0.12417$\pm$0.00032 \\ \hline
                $B^{+} \rightarrow J/\psi K^{+}$ & 2012 & 0.17741$\pm$0.00013 \\ \
%$B^{+} \rightarrow J/\psi K^{+}$ & 2015 &  Sim09a, Pyth8 & 4201997 & & \\
                $B^{+} \rightarrow J/\psi K^{+}$ & 2016 &  0.20031$\pm$0.00011 \\
		
		\hline
        \end{tabular}
        \end{center}
        \caption{Reconstruction and preselection efficiencies for signal and normalisation channels.}
        \label{tab:myreco}
        \end{table}


\subsection{Trigger Efficiency (TRG)}
\label{trigef}
 The trigger efficiency is calculated on the top of (GEN) and (REC) efficiency. In order to extract the trigger efficiency, full simulation for both signal and normalisation can be used. It should be noted though that at \gls{LHCb}, full simulation is produced based on a certain trigger configuration. Trigger configuration key, TCK, represents unique code for exact conditions the data have been triggered with at \texttt{L0}, \texttt{HLT1} and \texttt{HLT2}, notably thresholds of certain quantities such as $p$, $p_{T}$. 
 
Therefore if default TCK for simulation is representative for the whole considered dataset then the efficiency can be extracted directly from the simulation produced, which is the case for the Run \Rn{1} data.

However in Run \Rn{2} the trigger thresholds have been changing often resulting in 16 different TCKs with very different $p$, $p_{T}$ thresholds, see Table ~\autoref{tab:2016MC} for full detail. In the third column, luminosity proportion for 2016 is given. It can be seen that the default simulation in 2016 (corresponding to TCK decimal key 288888335) only represents around 35\% of the data. For this reason, the trigger efficiencies for 2016 data have been obtained by emulation of the trigger on simulation for \texttt{L0} and \texttt{HLT1} level for each individual TCK, creating 16 TCK-based simulations.  This trigger emulation to extract efficiencies was tested with the default trigger configuation (TCK 288888335) to validate the emulation and the correct efficiencies have been recovered. It should be noted that small differences arise from difference between \textit{offline} and \texttt{HLT1} container for \gls{PV}s which stores the information about \gls{minipchi2} as the \gls{PV} finding-algorithm is different, but these have neglgible effect. \mybox{mentioning this jsut i case of reproducibility, but maybe not necessary}.
%Therefore the emulation will not be exact, however, the effect is very small as it seems to be only resolution that will slightly shift the distribution, see figure ~\autoref{fig:hlt1shortcoming} 

In order to obtain the average efficiency for Run\Rn{2}, the 16 TCK efficiencies are weighted by the proportion of luminosity corresponding to the integrated luminosity for a given TCK over the full 2016 integrated luminosity. The integrated luminosity per TCK was extracted by looking at API version of the LHCb rundatabase. 

The full trigger luminosity for Run\Rn{2} is calculated by averaging the luminosity-weighted efficiencies, as seen in ~\autoref{tab:L0andHLT1Calib}.


\begin{table}
	\begin{center}
\footnotesize
      \begin{tabular}{l l l l | l l l l | l l }
      \multicolumn{4}{c |}{ } & \multicolumn{4}{c|}{\texttt{HLT1TrackMuon}} & \multicolumn{2}{c}{\texttt{L0Muon}}\\ \hline
	      TCK dec & TCK hex & \%$\mathcal{L}$ & $\mathcal{L}\,\textrm{pb}^{-1}$ & \gls{pgh2} & $p_{\mu}$[\mev] & $p_T(\mu)$[\mev] & \gls{minipchi2} & $SPD_{mult}$ & $p_T(\mu)$[\mev] \\% & $sum_{ET}$\\
      \multicolumn{10}{c}{\textbf{2016 MD} $0.859656\,\textrm{fb}^{-1}$} \\
      287905280 & 0x11291600 & $0.769$ & $12.74$  & $-$ & 6.0 & 0.91 & 10        & 450  & 14 \\% &  $-$\\
      287905283 & 0x11291603 & $2.11$ & $35.01$  & $-$ & 6.0 & 0.91 & 10         & 450  & 23 \\% &  $-$ \\
      287905284 & 0x11291604 & $1.50$ &  $24.78$  & $-$ & 6.0 & 0.91 & 10        & 450  & 27 \\% & $-$ \\
      287905285 & 0x11291605 & $4.73$ &   $78.42$   & $-$ & 6.0 & 0.91 & 10      & 450 & 31 \\% & $-$ \\
      288822793 & 0x11371609 & $4.35$   &  $72.14$ & 0.2 & 6.0 & 1.1  & 35       & 450  & 27 \\% & $-$ \\
      288822798 & 0x1137160e & $1.37$  &  $22.756$ & 0.2 & 6.0 & 1.1  & 35       & 450  & 27 \\% & $-$ \\
      288888329 & 0x11381609 & $0.414$  &  $6.86$  & 0.2 & 6.0 & 1.1  & 35       & 450 & 31 \\% & $-$ \\
      288888334 & 0x1138160e & $1.912$  &  $31.70$  & 0.2 & 6.0 & 1.1  & 35      & 450 & 31 \\% & $-$ \\
      288888335 & 0x1138160f & $34.7$  &  $575.25$  & 0.2 & 6.0 & 1.1  & 35  & 450 & 37 \\% & $-$ \\
      \multicolumn{10}{c}{\textbf{2016 MU} $0.798156\,\textrm{fb}^{-1}$} \\
      288495113 & 0x11321609 & $6.45$ &  $107.00$   &  $-$ & 6.0 & 0.91 & 10 & 450 & 27 \\% & $-$   \\
      288626185 & 0x11341609 & $7.12$ &  $118.06$  &  $-$ & 6.0 & 0.91 & 10 & 450 & 27 \\% & $-$    \\
      288691721 & 0x11351609 &  $1.42$ &  $23.46$   &  0.2 & 6.0 & 1.1  & 35  & 450 & 27 \\% & $-$  \\
      288757257 & 0x11361609 & $25.0$ &  $414.62$   &  0.2 & 6.0 & 1.1  & 35 & 450 & 27  \\% & $-$   \\
      288888337 & 0x11381611 & $ 2.66$ &  $44.13$  &  0.2 & 6.0 & 1.1  & 35  & 450 & 31 \\% & $-$   \\
      288888338 & 0x11381612 & $5.41$ &  $89.75$  &  0.2 & 6.0 & 1.1  & 35  & 450 & 33 \\% & $-$    \\
      288888339 & 0x11381613 & $0.0685$ &  $1.136$  &  0.2 & 6.0 & 1.1  & 35  & 450 & 27 \\% & 1000 \\
      \multicolumn{10}{c}{\textbf{MC Default}} \\
      1362630159 & 0x5138160f & $-$  & $-$   & 0.2 & 6.0 & 1.1  & 35 & 450 & 37\\% &  157\\

   \end{tabular}
\caption{Summary of 16 different TCKs listing properties of candidates necessary to pass \texttt{L0} and \texttt{HLT1} selection in 2016. In the final row, the default configuration for 2016 is shown and it corresponds to 288888335 TCK.}
\label{tab:2016MC}
	\end{center}
\end{table}

For the \texttt{HLT2} level, there were no significant changes of thresholds and are hence efficiencies are obtained from full simulation regardless. The systematic effect of this assumption will be listed in the systematic uncertainties chapter \mybox{SALLY - ADD SECTION REFERENCE TO SYSTEMATICS}.




%As it can be seen in the list of trigger requirements in ~\ref{tab:triggersel}, \texttt{L0MuonDecision} needs to be modelled. This trigger line selects event candidates only if candidate has certain muon $p_{T}$ and $nSPD$ hits.


%\texttt{HLT1} trigger selection was also emulated offline as \texttt{HLT1}. The efficiency of \texttt{HLT1TrackMuonDecision} is then determined on the top of \texttt{L0MuonDecision} as only events which have passed either \texttt{L0MuonDecision} or \texttt{L0DimuonDecision} would be considered. \texttt{HLT1TrackMuonDecision} trigger lines requires events with only certain muon $p_{T}$,$p$, \textit{ghost probability} and $MIP\chi^{2}$. 

%There are other variables that are included in \texttt{HLT1TrackMuonDecision} such as whether the track is \gls{VELO} track or how many hits have been missed in \gls{VELO}, however, these have not changed throughout the Run \Rn{2} data and are not likely to be different between signal and normalisation channel.  Hence only efficiency for the relevant cuts are included in emulation of \texttt{HLT1TrackMuonDecision}.

%It should be noted that there is difference between \textit{offline} and \texttt{HLT1} container for PVs which stores the information about $MIP\chi^{2}$ fast Kalman fitter rather then full is used. Therefore the emulation will not be exact, however, the effect is very small as it seems to be only resolution that will slightly shift the distribution, see figure ~\autoref{fig:hlt1shortcoming}.

%This trigger emulation to extract efficiencies was tested with the default trigger configuation to validate the emulation and the correct efficiencies have been recovered. TCK dependent efficiecy breakdown for signal and normalisation channel can be seen in Table ~\autoref{tab:L0andHLT1Calib}. In order to obtain the average efficiency for Stripping 26, these efficiencies are weighted by the \% of lumi for which this luminosity was ran on. These numbers have been obtained by looking at API version of the rundatabase where one can obtain luminosity per TCK.


\begin{table}[ht]
\footnotesize
\begin{center}
\begin{tabular}{ l |  c  c  c | c  c  c }
 \multicolumn{1}{c|}{} & \multicolumn{3}{c|}{$B^{+} \rightarrow \mu^{+} \mu^{-} \mu^{+} \nu$ } & \multicolumn{3}{c}{$B^{+} \rightarrow J/\psi\ K^{+}$} \\ \hline
 TCK & $\varepsilon_{L0}$ & $\varepsilon_{HLT1}$ & $\varepsilon_{HLT2}$ & $\varepsilon_{L0}$ & $\varepsilon_{HLT1}$ & $\varepsilon_{HLT2}$ \\
\hline
287905280 & 0.921 & 0.999 & 0.831 & 0.891 & 0.997 & 0.943 \\
287905283 & 0.905 & 0.999 & 0.845 & 0.878 & 0.998 & 0.953 \\
287905284 & 0.894 & 0.999 & 0.855 & 0.867 & 0.998 & 0.962 \\
287905285 & 0.88 & 0.999 & 0.868 & 0.854 & 0.998 & 0.973 \\
288495113 & 0.894 & 0.999 & 0.855 & 0.867 & 0.998 & 0.962 \\
288626185 & 0.894 & 0.999 & 0.855 & 0.867 & 0.998 & 0.962 \\
288691721 & 0.894 & 0.957 & 0.873 & 0.867 & 0.94 & 0.965 \\
288757257 & 0.894 & 0.957 & 0.873 & 0.867 & 0.94 & 0.965 \\
288822793 & 0.894 & 0.957 & 0.873 & 0.867 & 0.94 & 0.965 \\
288822798 & 0.88 & 0.957 & 0.886 & 0.854 & 0.941 & 0.976 \\
288888329 & 0.894 & 0.957 & 0.873 & 0.867 & 0.94 & 0.965 \\
288888334 & 0.88 & 0.957 & 0.886 & 0.854 & 0.941 & 0.976 \\
288888335 & 0.848 & 0.958 & 0.911 & 0.821 & 0.941 & 0.999 \\
288888337 & 0.88 & 0.957 & 0.886 & 0.854 & 0.941 & 0.976 \\
288888338 & 0.871 & 0.957 & 0.895 & 0.844 & 0.941 & 0.984 \\
288888339 & 0.89 & 0.957 & 0.877 & 0.864 & 0.94 & 0.968 \\
\hline
Weighted efficiency & 0.876 & 0.967 & 0.884 & 0.849 & 0.953 & 0.978 \\
\hline
\end{tabular}
\end{center}
\caption{Efficiencies of 2016 trigger emulation on MC. Depending on TCK, the efficiencies vary up 10\% for \texttt{L0} level for signal MC and up to 5\% for normalisation TCK. This is important as \textit{single event sensitivity} is sensitive to the ratio of these two efficiencies. This configuration is discribing correctly only 35\% data with high $p_{T}$ threshold.}
\label{tab:L0andHLT1Calib}
\end{table}



Run \Rn{1} trigger efficiency is determined directly by looking at default TCK and is summarised in Table ~\autoref{tab:L0andHLT1Calib2012}. 

%It should be noted that in the rest of the efficiencies Stripping 21 with be representative of 21+21r1 dataset as it was noticed that for normalisation channel these efficiencies are eqvivalent. In this section following ratio will be calculated,


\begin{table}[H]
\begin{center}
\begin{tabular}{ l  l  l  l  }
 Efficiency &  $B^{+} \rightarrow \mu^{+} \mu^{-} \mu^{+} \nu$  (2012)  &  $B^{+} \rightarrow J/\psi\ K^{+}$ (2012) & $B^{+} \rightarrow J/\psi\ K^{+}$ (2011) \\
\hline
$\varepsilon_{L0}$ &0.900 & 0.873 & 0.907 \\
$\varepsilon_{HLT1}$ &0.934 & 0.908 & 0.879 \\
$\varepsilon_{HLT2}$ &0.883 & 0.981 & 0.973 \\
\hline
\end{tabular}
\end{center}
\caption{2012 default TCK efficiencies. These values will be taken to be representative of 2012 and 2011 dataset as the cummulative efficiency for 2011 and 2012 is nearly identical.}
\label{tab:L0andHLT1Calib2012}
\end{table}



