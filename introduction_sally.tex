\chapter{Introduction}
%The Standard Model is without question the most powerful and tested theory of particle physics. It describes and predicts many phenomena very well even though the theory is not including any explanation for the nature of dark matter and it doesn't make any attempt to describe gravity in a quantum field theory framework.

%Furthermore, fine-tuning of some parameters in the Standard Model such as the Higgs mass, where parameters get exactly the right value to produce required behaviour, beg questions if there is some symmetry in the model building that is missing. In this chapter, the theoretical basis of the Standard Model is discussed and is then followed by experimental and theoretical consideration of fully leptonic decays.


The field of particle physics aims is to describe the universe we see today by decomposing everything into fundamental building blocks which then follow certain behaviour according to a given set of rules. The foundation bricks of theoretical formulation of the current best established theory that describe the universe around us, the Standard Model (\Gls{SM}), were laid last century. Some achievements of the \gls{SM} do really leave us breathless with agreement between theoretical and experimental results of ten parts in a billion. 

This theory is, however, incomplete as it fails to adress several issues. The theory does not include any explanation for the nature of dark matter and it doesn't make any attempt to describe gravity in a quantum field theory framework. Furthermore, fine-tuning of some parameters in the \gls{SM} such as the Higgs mass, where parameters get exactly the right value to produce required behaviour, beg questions if there is some symmetry in the model building that is missing. Lastly, as with any model, SM operates with many free parameters that need to be plugged in so that predictions can be made. So why are there exactly so many?

This thesis describes a search for a decay which can help to shed light on some of these parameters with following organization.

In~\autoref{stheory} the \gls{SM} of particle physics is discussed together with the theoretical and experimental motivation for fully leptonic decays, especially for \Bmumumu decay. In~\autoref{chap:dec} the tool to search for \Bmumumu decays, the LHCb detector, is detailed. This is then followed by a discussion about how does trimuon signature behave in the detector which is covered in~\autoref{chap:trimuon}. The analysis of \Bmumumu, the central theme for the thesis is then described in~\autoref{chap:sel} and~\autoref{chap:masandef}.  

