\chapter{Discovering (Setting Limit for) \Bmumumu at LHCb}
\label{chap:sel}

\textit{LHCb's flagship analyses contain several muons in the final state coming from differently flavoured $B$ mesons. Despite being in this category, search for \Bmumumu is limited by the rareness of its occurrence as well as different backgrounds that can mimic its signature in the detector. Moreover, presence of invisible neutrino does induce uncertainties into reconstruction. This \autoref{chap:sel} will concentrate on characterisation of backgrounds as well as selection that is performed in order to reduce the backgrounds.}


\section{Topology of at LHCb \Bmumumu at LHCb}

Upon hadronisation of $b\bar{b}$ pair \Bpm particle will travel less than a millimetre in the laboratory frame of reference before it decays into its decay products. This allows reconstruction of a primary vertex \gls{PV} and its decay vertex, \textit{secondary vertex} \gls{SV}. By joining these vertices, direction as well as length of the \Bpm existence, also known as flight distance (\gls{FD}), can be established. In order to infer information about kinematic properties of \Bpm, the decay products are studied. All three muons are used to reconstruct the visible four-momentum. By conservation of momentum with respects to the direction of the flight of \Bpm, neutrino is assigned all missing momentum transverse to the direction of the flight of \Bpm. The schematic diagram can be seen in \autoref{fig:sigtopolog}.

\begin{figure}[!h]
	\centering
	\includegraphics[width = 0.8\textwidth]{figs/sel/DecReco_fin.eps}
	\caption{Schematic view of \Bmumumu decay. At $pp$ interaction point, or $PV$, $b\bar{b}$ pair hadronizes into \Bpm. \Bpm flies some distance before decaying into three muons and neutrino. All charged tracks (in filled-blue) seen can be combined into four-vector representing the visible part of the decay (semi filled-blue). Information about invisible neutrino (semi filled-red) are deduced from the conservation of momentum with respect to the direction of the flight of \Bpm. Neglecting momentum component parallel to the direction of flight for neutrino, transverse component of momentum is given.}
	\label{fig:sigtopolog}
\end{figure}

Altogether it allows for reconstruction of a quantity, \textit{corrected mass}, that plays similar role to invariant mass in fully reconstructed decays. Invariant mass is usually used in \gls{LHCb} for fitting distribution from which physics results are extracted as it distinguishes well signal from background and there is minimal modelling problem.

\textit{Corrected mass} is defined as

\begin{equation}
	M_{corr} = \sqrt{{M}^{2} + |p^{2}_{T}|} + |p_{T}|,
\end{equation}	
where the $M^{2}$ is the invariant visible mass squared and $p^{2}_{T}$ is the missing momentum squared transverse to the direction of $B^{+}$ flight.

%\begin{equation}
%M_{B_{corr}} = \sqrt{M_{\mu^{+} \mu^{-} \mu^{+}}^{2} + |p_{T}|} + |p_{T}|,
%\end{equation}	

$M_{corr}$ can be thought of as the minimal correction to the visible mass to account for the missing neutrino information. The resolution on the \textit{corrected mass} hence becomes a critical quantity that needs to be understood. As the method of reconstruction of corrected mass relies heavily on the knowledge of \Bpm flight direction, the resolution of \gls{PV} position and \gls{SV} vertex is crucial. Let $\vec{{x}}_{PV}=\{x_{PV},y_{PV},z_{PV}\}$, $\vec{{x}}_{SV}=\{x_{SV},y_{SV},z_{SV}\} $ be \gls{PV} and \gls{SV} vertex position and $\vec{p}=\{p_{x},p_{y},p_{z}\}$ be the visible trimuon momentum. Then the missing transverse momentum to the direction of the flight $p_{T}$ (momentum of the neutrino) is


\begin{equation}
	p^{2}_{T} = |\vec{p} - (\vec{{x}}_{SV}-\vec{{x}}_{PV})\frac{\vec{p} \cdot(\vec{{x}}_{SV}-\vec{{x}}_{PV})}{|(\vec{{x}}_{SV}-\vec{{x}}_{PV})|^{2}}|^{2}
\end{equation}

In general in order to propagate error on $f(x,y,z)$, where $x,y,z$ are independent variables, the variance of $f(x,y,z)$

\begin{equation}
\begin{aligned}
	\langle f^{2}-\langle f \rangle^{2} \rangle  &=  \langle f(x+\delta x, y+\delta y, z+\delta z)^{2} - f(\langle x \rangle, \langle y \rangle, \langle z \rangle)^{2} \rangle \\
\end{aligned}
\end{equation}

Using first order Taylor expansion of variance and rewriting into the matrix form:

\begin{equation}
\begin{aligned}
%	\langle \frac{\partial{f}}{\partial{x}}  \frac{\partial{f}}{\partial{x}} \times \delta x^{2} + \frac{\partial{f}}{\partial{y}}  \frac{\partial{f}}{\partial{y}} \times \delta y^{2} + \frac{\partial{f}}{\partial{z}}  \frac{\partial{f}}{\partial{z}} \times \delta z^{2} + \sum_{i,j=1..3, i \neq j} \partial_{d_{i}} \partial_{d_{i}}  \frac{\partial{f}}{\partial{x}} \frac{\partial{f}}{\partial{y}} \rangle \\ 
%        = 
       \begin{bmatrix}
		\frac{\partial{f}}{\partial{x}} & \frac{\partial{f}}{\partial{y}} & \frac{\partial{f}}{\partial{z}} \\
       \end{bmatrix}
       \begin{bmatrix}
	       {\delta x}^{2} & \delta x \delta y & \delta x \delta z  \\ 
	        \delta y \delta x & {\delta y}^{2} & \delta y \delta z  \\
	        \delta z \delta x & \delta z \delta y & {\delta z}^{2}  \\
       \end{bmatrix}
       \begin{bmatrix}
		\frac{\partial{f}}{\partial{x}} \\ \frac{\partial{f}}{\partial{y}} \\\frac{\partial{f}}{\partial{z}} \\
       \end{bmatrix}
\end{aligned}
\end{equation}

So now assuming that $x=\vec{{x}}_{PV}$, $y=\vec{{x}}_{SV}$ and $z=\mathbold{\vec{p}}=\{E,p_{x},p_{y},p_{z}\}$,

\begin{equation}
\begin{aligned}
       \nabla^{T}_{x_{PV}} \rm{COV_{x_{PV}}} \nabla_{x_{PV}} + \nabla^{T}_{x_{SV}} \rm{COV_{x_{SV}}} \nabla_{x_{SV}} + \nabla^{T}_{p} \rm{COV_{p}} \nabla_{p}
\end{aligned}
\end{equation}

where \rm{COV} is the covariance matrix.

In conclusion in order to calculate error on \textit{corrected mass}, $\delta_{corrm}$


\begin{equation}
\begin{aligned}
	\delta_{corrm} = \sqrt{ \langle f^{2}-\langle f \rangle^{2} \rangle} = \sqrt{\nabla^{T}_{x_{PV}} \rm{COV_{x_{PV}}} \nabla_{x_{PV}} + \nabla^{T}_{x_{SV}} \rm{COV_{x_{SV}}} \nabla_{x_{SV}} + \nabla^{T}_{p} \rm{COV_{p}} \nabla_{p}} 
\end{aligned}
\end{equation}

which can be calculated analytically (method used for all the plots) or using numerical approximation of first derivative of \textit{finite differences}.

\section{Sources of Backgrounds}
The largest background that can be will be looking similar to signal comes from \textit{cascade decays}, where semileptonic $b \rightarrow c \rightarrow s$ or ($\bar{b} \rightarrow \bar{c} \rightarrow \bar{s}$) transition occurs. A typical example of this background in hadronic terms is $B^{+} \rightarrow (\bar{D}^{0} \rightarrow (K^{+} \rightarrow \mu^{-} \nu) \mu^{+} \nu$), where $K^{+}$ is misidentified as muon. Because $K^{+}$ is misidentified as muon, this type of background is denoted as misID background.

In fact, any other particle species that is misidentified, belongs to the misID background category. If the sign of the misidentified particle agrees with the sign of the mother \Bpm, it belongs to the same sign misID background (\textit{SS misID}) background. In the event where opposite sign particle to the mother \Bpm is misidentified, this background will be referred to as (\textit{OS misID}) background. However, \textit{OS misID} background is expected to have smaller rate as the misidentified particle would have to proceed via decays with additional particles or if coming as product from other $b$ hadronization.

The presence of other \B-hadron from $b\bar{b}$ pair does create its own decay chain and hence it is possible to combine one of its muon tracks with two muons from the "signal" \B. This is denoted as combinatorial background.

Then presence of neutrino in a final state allows for certain uncertainty regarding the information of the fourth decay product. If some of the tracks of the decays are not reconstructed, either because they are neutral, or either they are charged but they are soft, it means that the missing information may be attributed to the neutrino. \textit{Missing tracks} will hence create partially reconstructed background. Some of the most dangerous are ${B^{+} \rightarrow D \mu^{+} \nu}$ type partially reconstructed backgrounds where $B^+ \rightarrow (D^0 \rightarrow K^- \pi^+ \mu^{+} \mu^{-})\mu \nu$, where $\mathcal{B}(D^0 \rightarrow K \pi^+ \mu^{+} \mu^{-}) \approx 4.17\times 10^{-6}$ and $B^{+} \rightarrow D^0 \mu \nu \approx 10 \%$. This predicts $\mathcal{B}(B^+ \rightarrow K^+ \pi^- \mu^+ \mu^{-} ) = 1\times10^{-7}$.

\section{Analysis strategy}
\label{Strategy}

The analysis of the $B^{+} \rightarrow \mu^{+} \mu^{-} \mu^{+} \nu_\mu$ decay is divided into several different parts; signal selection, optimisation, normalisation, fitting and limit setting. Throughout this document, charge conjugates of the decays are assumed unless stated otherwise. Results presented are based on the analysis of the full 3 fb$^{-1}$ Run 1 dataset as well $\approx$ 1.7 fb$^{-1}$ Run 2 data (not using 2015 dataset due to very low sensitivity (high muon trigger thresholds)). Additionally the search will be conducted in a particular min$q^{2}$ = min($q^{2}(\mu_{1}^{+},\mu^{-}), q^2(\mu^{-},\mu_{2}^{+})$) region.
\newline To perform the search for $B^{+} \rightarrow \mu^{+} \mu^{-} \mu^{+} \nu_\mu$, a specific preselection was applied to form potential signal candidates. To reconstruct the mass of the $B^{+}$ with missing information about the neutrino, a corrected mass variable $M_{B_{corr}} = \sqrt{M_{3\mu}^{2} + |p^{2}_{\perp}|} + |p_{\perp}|$, where $M_{3\mu}^{2}$ is the invariant visible mass squared and $p^{2}_{\perp}$ is the missing momentum squared transverse to the direction of flight of $B^{+}$, is introduced. A simulation sample that mimics the decay of the $B^{+} \rightarrow \mu^{+} \mu^{-} \mu^{+} \nu_\mu$ passing through preseletion was used to develop further discriminating selection. To get the selection efficiency for different types of backgrounds, different proxy samples are used. For more details about samples used see Section~\ref{Data}.
\newline Combinatorial background, which arises as random combinations of tracks passing the preselection, is taken from the upper corrected $\mu^{+} \mu^{-} \mu^{+}$ mass side band, $M_{Bcorr} > 5.5$ GeV, where very few signal candidates are expected.


\section{Samples}
\subsection{Data Samples}
Results presented in this thesis are based on the analysis of the full 3 fb$^{-1}$ Run \Rn{1} dataset at $\sqrt{s}={7},{8}$ \tev as well $\approx$ 1.7 fb$^{-1}$ Run \Rn{2} data at $\sqrt{s}=13$\tev.

\subsection{Simulation Samples}
\label{Simulation Samples}
For signal simulation, three different decay models were exploited and are summarized in \autoref{tab:MCPPass}.

\begin{table}[h!]
	\begin{center}
		\begin{tabular}{l l l l l l}

			Channel & Year & Pythia  & EVTGEN & Size & Stage \\ \hline
			 \multicolumn{6}{c}{Simulation used for fitting mass shapes} \\ \hline
			$B^{+} \rightarrow \mu^{+} \mu^{-} \mu^{+} \nu$ & 2012 & Pythia 6.4\cite{pythia6} & PHSP & 0.5$\rm{M}$ & \textit{generator-level+detector}\\
			$B^{+} \rightarrow \mu^{+} \mu^{-} \mu^{+} \nu$ & 2012 & Pythia 8.1\cite{pythia8} & PHSP & 0.5$\rm{M}$ & \textit{generator-level+detector}\\
			$B^{+} \rightarrow \mu^{+} \mu^{-} \mu^{+} \nu$ & 2012 & Pythia 6.4\cite{pythia6} & MINE & 0.5$\rm{M}$ & \textit{generator-level+detector}\\
			$B^{+} \rightarrow \mu^{+} \mu^{-} \mu^{+} \nu$ & 2012 & Pythia 8.1\cite{pythia8} & MINE & 0.5$\rm{M}$ & \textit{generator-level+detector}\\
			$B^{+} \rightarrow \mu^{+} \mu^{-} \mu^{+} \nu$ & 2016 & Pythia 8.1\cite{pythia8} & MINE & 1.0$\rm{M}$ & \textit{generator-level+detector}\\ \hline
			 \multicolumn{6}{c}{Simulation used for evaluating \textit{generator-level} efficiencies} \\ \hline
			$B^{+} \rightarrow \mu^{+} \mu^{-} \mu^{+} \nu$ & 2012 & Pythia 6.4\cite{pythia6} & PHSP & 25000 & \textit{generator-level}\\ %using only for comaprison pythia 6
			$B^{+} \rightarrow \mu^{+} \mu^{-} \mu^{+} \nu$ & 2012 & Pythia 6.4\cite{pythia6} & MINE & 25000 & \textit{generator-level}\\
			$B^{+} \rightarrow \mu^{+} \mu^{-} \mu^{+} \nu$ & 2012 & Pythia 8.1\cite{pythia8} & MINE & 25000 & \textit{generator-level}\\ \hline

                         \multicolumn{6}{c}{Simulation used for ratification of minq$^{2}$ selection} \\ \hline 

			$B^{+} \rightarrow \mu^{+} \mu^{-} \mu^{+} \nu$ & 2012 & Pythia 6.4\cite{pythia6} & NIKI & 25000 & \textit{generator-level}\\ %\hline
			%			\hline
%			$B^{+} \rightarrow J/\psi K^{+}$ & 2011 & Pythia6\cite{pythia6} & /Sim08c/Digi13/Trig0x40760037/Reco14a/Stripping20r1NoPrescalingFlagged \\
%			$B^{+} \rightarrow J/\psi K^{+}$ & 2011 & Pythia8\cite{pythia8} & /Sim08c/Digi13/Trig0x40760037/Reco14a/Stripping20r1NoPrescalingFlagged \\
%			$B^{+} \rightarrow J/\psi K^{+}$ & 2012 & Pythia6\cite{pythia6} & /Sim08h/Digi13/Trig0x409f0045/Reco14c/Stripping20NoPrescalingFlagged \\
%			$B^{+} \rightarrow J/\psi K^{+}$ & 2012 & Pythia8\cite{pythia8} & /Sim08h/Digi13/Trig0x409f0045/Reco14c/Stripping20NoPrescalingFlagged \\
%			%$B^{+} \rightarrow J/\psi K^{+}$ & 2015 & Pythia8 & /Sim09a/Trig0x411400a2/Reco15a/Turbo02/Stripping24NoPrescalingFlagged \\
%			$B^{+} \rightarrow J/\psi K^{+}$ & 2016 & Pythia8\cite{pythia8} & /Sim09b/Trig0x6138160F/Reco16/Turbo03/Stripping26NoPrescalingFlagged \\
%			\hline
%			$B^{+} \rightarrow J/\psi \pi^{+}$ & 2012 & Pythia6\cite{pythia6} & /Sim08a/Digi13/Trig0x409f0045/Reco14a/Stripping20NoPrescalingFlagged \\
%			$B^{+} \rightarrow J/\psi \pi^{+}$ & 2012 & Pythia8\cite{pythia8} & /Sim08a/Digi13/Trig0x409f0045/Reco14a/Stripping20NoPrescalingFlagged \\
%
%			\hline
%			$B^{0} \rightarrow J/\psi K^{*}$ & 2011 & Pythia8 & /Sim08f/Digi13/Trig0x40760037/Reco14a/Stripping20r1NoPrescalingFlagged\\
%			$B^{0} \rightarrow J/\psi K^{*}$ & 2012 & Pythia8 & /Sim08f/Digi13/Trig0x409f0045/Reco14a/Stripping20NoPrescalingFlagged\\
%			$B^{0} \rightarrow J/\psi K^{*}$ & 2016 & Pythia8 & /Sim09b/Trig0x6138160F/Reco16/Turbo03/Stripping26NoPrescalingFlagged\\
%			\hline
%			PartReco & 2012 & Pythia8 & /Sim08g/Digi13/Trig0x409f0045/Reco14a/Stripping20NoPrescalingFlagged  \\
			\hline
		\end{tabular}
	\end{center}
	\caption{Summary of signal simulation samples used in this analysis with different decay models. In all cases the daughters of \Bpm are required to be within \gls{LHCb} acceptance. All of this samples are mixture under magnetic polarity up and magnetic polarity down conditions. }
	\label{tab:MCPPass}
\end{table}



Full phase space model, \textit{PHSP}, only takes into account the kinematic constraints of the decay without taking into account any input from theoretical considerations as the matrix element is constant and hence angular momentum is disregarded.

In order to produce simulation with decay model which is more representative of the spin structure involved, following strategy is adapted. In this picture, the decay proceeds as follows: \Bpm decays into \Wpm and a pair of opposite sign muons and then \Wp is decayed to $\mu^{+} \nu$. \textit{BTOSLLBALL} model\cite{Ali:1999mm}, traditionally used for $\B \rightarrow (K,K^{*}) l^{+} l^{-}$ decay, with the form factor calculations can be used to simulate $\Bpm \rightarrow \Wpm l^{+} l^{-}$ decay. After that, \Wp is decayed to $\mu^{+} \nu$ using \textit{PHSP}. For semileptonic $b \rightarrow s l^{+} l^{-}$ transitions, there is a characteristic photon pole for low $q(\mu^{+},\mu^{-})$, invariant mass of the opposite muon pair, and flat distribution for $K^{*}(\mu^{+}, \nu_{\mu}) $, invariant mass of the muon and neutrino pair. In order to achieve this, a new pseudo-particle is introduced to EVTGEN with specific properties, $K^{*}(\mu^{+}, \nu_{\mu})$, and the best output can be seen to be for a particle $K^{*}(\mu^{+}, \nu_{\mu})$ with mass to be set to $0.1 \text{GeV/c}^{2}$, and width, corresponding to $\tau= 1.3\times10^{-17}$ nanoseconds as can be seen in \autoref{fig:mcgeneration}. This procedure was also applied for the charge conjugate case. This model is denoted as \textit{MINE} and is used as default in mass fits and efficiency calculations.


\begin{figure}[h!]
\centering
\includegraphics[width=0.5\linewidth]{./sel/reporttry_new}\put(-70,133){(a)}
\includegraphics[width=0.5\linewidth]{./sel/reporttrialqpres_new}\put(-50,133){(b)}
\caption{Distributions for signal MC in using Pythia 6.4 \cite{pythia6} conditions. (a) $K^{*}(\mu^{+}, \nu_{\mu})$ (b) $q(\mu^{+},\mu^{-})$ distributions under different $K^{*}$ mass hypotheses. The most flat distribution in $K^{*}(\mu^{+}, \nu_{\mu})$ is plotted in yellow.}
\label{fig:mcgeneration}
%\vspace*{-1.0cm}
\end{figure}


Finally, exclusively for this decay, a new decay model \textit{B2MuMuMuNu} was added to EVTGEN, based on work performed by theorist Nikola Nikitin (write more once theory chapter is done and refer to it.). This model denoted as \textit{NIKI}, is used mainly for validation purposes. 


\section{Preselection for \Bmumumu}

Set of initial identification for signal \Bmumumu summarized in \autoref{tab:stripcutsB}, also known as \textit{stripping} selection was develloped in order to improve signal to background ratio. 

Firstly, all three muon tracks are required to have a significant \gls{IP} with respect to the primary vertex. Minimum Impact Parameter $\chi^{2}$, (\gls{minipchi2}), gives the minimum significance of a particles's trajectory to the primary vertex. Hence by requiring \gls{minipchi2}$>9$ for muons is consistent with the hypothesis that the muon is $3\sigma$ away from the primary vertex and hence can be well differentiated. In addition, the change in the $\chisq$ if \gls{PV} and \gls{SV} vertices are fitted separately as opposed to common vertex fit, \gls{fdchi2}, suppresses prompt backgrounds. 

Each muon track is required to have good track $\chi^{2}$ per number of degrees of freedom (\texttt{ndof}), (\gls{trackchi2ndof}), of the fit as well as low \gls{pgh2}. This removes spurious tracks as well as tracks with low quality.

Each muon candidate is also identified with initial basic \gls{PID} variables. Firstly muons are chosen due to their signature in the muons stations with the binary \texttt{isMuon} decision. Secondly, muons candidates are chosen such that it is more likely that the candidate is muon than pion or kaon using global DLLmu variables defined in \autoref{muonID}. This reduces the background from misidentified muons.

In order to only select events which are compatible with the three muons originating from the same point in the space, (\gls{vertexchi2ndof}), the $\chi^{2}$ of the trimuon vertex per degree of freedom fit required to be small. This decreases the contamination from \textit{cascade decays} where the particle with the $c$ quark content from $b \rightarrow \underline{c} \rightarrow s$, such as $D$, would have non-negligible lifetime leading to higher \gls{vertexchi2ndof}. 

Requiring that \Bp direction points in the same direction as the line from \gls{PV} to \gls{SV}, (\gls{DIRA} - which measures the angle between these two vectors), is close to unity translates into a well reconstructed event, which minimizes combinatorial background, where random track makes this pointing worse. Putting bounds on mass window, whether it is visible or corrected mass, also supresses combinatorial events. This is because of on average higher momentum of combinatorial muon leading to higher mass.  


\begin{table}%[H]
\begin{center}
\begin{tabular}{l|c l }

    \hline
     Candidate & Stripping Selection \\ \hline

	muon & \gls{minipchi2} $> 9$ &  \rdelim\}{3}{1cm}[\ track] \\
	muon & $p_{T} >$ 0 \\
	muon & \gls{trackchi2ndof}$ < 3$ \\

	
	muon & $DLL_{\mu} > 0$ & \rdelim\}{3}{1cm}[\ \gls{PID}] \\
	muon & $DLL_{\mu} - DLL_{K} > 0$ \\
	muon &  \texttt{isMuon==true} \\ \hline
	
	combination & \gls{DIRA} $> 0.999$ \\
        combination & $p_{T} >$ 2000 \mev\\
	combination & \gls{fdchi2} $> 50$\\
	combination & \gls{vertexchi2ndof} $< 4$ \\
	combination & $0\ \mevcc\ <\ M_B\ <\ 7500\ \mevcc$ \\
	combination & $2500\ \mevcc\ <\ M_{B_{corr}}\ <\ 10000\ \mevcc $\\ \hline
     \end{tabular}

\end{center}
	\caption{Selection of events based on muon and the $B^{+}$ candidate requirements. \textit{Stripping selection} for the signal decay $B^{+} \rightarrow \mu^{+} \mu^{-} \mu^{+} \nu_\mu$ is the same for both Run1 and 2016 data.}
\label{tab:stripcutsB}
\end{table}

\section{Trigger Selection}
In order to obtain triggered data, $B^{+} \rightarrow \mu^{+} \mu^{-} \mu^{+} \nu_\mu$ candidates are required to pass certain set of trigger decisions at \gls{L0}, \gls{HLT1} and \gls{HLT2} levels summarized in \autoref{tab:triggersel}. It can be noted that the decision is applied at the mother \Bpm level. In particular, \texttt{Bplus$\_$L0MuonDecision$\_$TOS decision}, means that one of the muons from \Bpm in an event has triggered and made positive decision.

\begin{table}[h!]
\begin{center}
	\begin{tabular}{ l l}%l | l | }
Trigger Selection  \\ %& 2011 & 2012 & 2016 \\
\hline
		\texttt{Bplus$\_$L0MuonDecision$\_$TOS} \\ %& 0.915 & 0.895 & 0.74 \\
\hline
		\texttt{Bplus$\_$Hlt1TrackMuonDecision$\_$TOS} \\% & 0.874 & 0.929 & 0.931 \\
%Or of HLT2 lines below & 0.986 & 0.987 & 0.996  \\
\hline
		\texttt{Bplus$\_$Hlt2TopoMu2BodyBBDTDecision$\_$TOS} & \rdelim\}{4}{1cm}[\ OR]\\ % & 0.859 & 0.892 & 0.94* \\
		\texttt{Bplus$\_$Hlt2TopoMu3BodyBBDTDecision$\_$TOS} \\ % & 0.677 & 0.76 & 0.886* \\
		\texttt{Bplus$\_$Hlt2DiMuonDetachedDecision$\_$TOS} \\ % & 0.809 & 0.769 & 0.988 \\
		\texttt{Bplus$\_$Hlt2DiMuonDetachedHeavyDecision$\_$TOS} \\ % & 0.94 & 0.929 & 0.99 \\
\hline
\end{tabular}
\end{center}
	\caption{Trigger selection applied on both signal and normalisation samples}
	\label{tab:triggersel}
\end{table}

As discussed in \autoref{triggerchap} \texttt{L0MuonDecision} decides on whether an event is accepted depending on the $p_{T}$ of muon and the number of hits in the \gls{SPD}. Run \Rn{1} can be split into 2011 and 2012 conditions where, in 2011 the most used threshold for positive decision is 1.48 \gevc \cite{Aaij:2012me} and 1.76 \gevc \cite{Albrecht:2013fba}. Run \Rn{1} \gls{SPD} rate only accepts events below 600. In Run \Rn{2}, the trigger thresholds varied more but the most representative acceptance for muon $p_{T}$ was above 1.85 \gevc with \gls{SPD} multiplicity below 450.

\texttt{Hlt1TrackMuonDecision} accepts events where mother particle has certain lifetime, such as $B,D$ by requiring certain \gls{ipchi2} of the track with respect to all of its \gls{PV}s. There has to be at least one muon (\texttt{isMuon==true}) in its final state with certain kinematic thresholds on $p$ and $p_{T}$, for example in 2011, the identified muons that triggered positive decision had to have $p$ above 8 \gevc \cite{Aaij:2012me}.

At \gls{HLT2} level, the candidates are reuqired to pass through at least one of the four decisions. \texttt{Hlt2TopoMu[2,3]BodyBBDTDecision} belong to the \textit{topological triggers} category with extra requirement of muon being idendified by \texttt{isMuon} decision. \texttt{Hlt2DiMuonDetachedDecision} and \texttt{Hlt2DiMuonDetachedHeavyDecision} reconstruct decays with two muons in a final state, dimuon. The two lines differ in selection that is optimized either for heavy or light dimuon pair. For example, \texttt{Hlt2DiMuonDetachedDecision} accepts events dimuon $p_{T}$ above 1.5 \gevc and with mass above 1 \gevcc, whereas  \texttt{Hlt2DiMuonDetachedHeavyDecision} accepts dimuon pairs with any $p_{T}$ but above 2.95 \gevcc in mass. The reason why these lines are called detached are because individual muons are required to have high \gls{ipchi2}.


\section{$q^{2}$ selection}

In \Bmumumu, two pairs of opposite sign muons can be formed, namely $q^2(\mu_1,\mu_2)$ and $q^2(\mu_2,\mu_3)$ where $\mu_1=\mu^{+} , \mu_2=\mu^{-}, \mu_3=\mu^{+} $.
From the two invariant mass squared pairs one can define, $minq^2 = min[q^{2}(\mu_1,\mu_2), q^2(\mu_2,\mu_3)]$ and $maxq^{2} = max[q^{2}(\mu_1,\mu_2), q^2(\mu_2,\mu_3)]$. This measurement is made in region where ${minq<980}$ \mevcc because of two main reasons: most of the contributions to the amplitude of the decay is below this value, see \autoref{fig:qsqsel} and especially combinatorial background is greatly reduced if $minq^{2}<1$ (\gevcc)$^{2}$.

In order to remove backgrounds that proceed via resonant $J/\psi$ and $\Psi(2S)$ contributions, vetoes are placed in the corresponding regions \autoref{tab:vetoes}.

\begin{figure}[h!]
\centering
\includegraphics[width=0.5\linewidth]{./sel/scatterplotiatko_qmin.pdf}\put(-70,133){(a)}
\includegraphics[width=0.5\linewidth]{./sel/scatterplotiatko_qmin_data.pdf}\put(-50,133){(b)}
	\caption{(a) Signal simulation sample distribution in $minq$ and $maxq$ variables. Values below 980 \mevcc are accepted. (b) Data sample after \textit{stripping} selection with no other cuts shows clearly the $J/\psi$ and $\Psi(2S)$ resonances.  }
\label{fig:qsqsel}
%\vspace*{-1.0cm}
\end{figure}



\begin{table}[h!]
\begin{center}
\begin{tabular}{l c c}
	Veto & $minq$ \mevcc & $maxq$ \mevcc \\ \hline
        J/$\psi$  & !(2946.0 $<$ $|$ m($\mu^{+}$ $\mu^{-}$) $|$ $<$ 3176.0) & !(2946.0 $<$ $|$ m($\mu^{+}$ $\mu^{-}$) $|$ $<$ 3176.0) \\
	$\Psi$(2S) &  !(3586.0 $<|$ m($\mu^{+}$ $\mu^{-}$) $|$ $<$ 3766.0) & !(3586.0 $<$ $|$ m($\mu^{+}$ $\mu^{-}$) $|$ $<$ 3766.0) \\
        \hline
\end{tabular}
\end{center}
\caption{Vetoes for $J/\psi$ and $\Psi(2S)$ contributions}
\label{tab:vetoes}
\end{table}


Similarly to


