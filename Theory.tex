\chapter{Theory}
The form of the full \gls{SM} Lagrangian is dictated by the internal symmetries $SU(3)\times SU(2)\times U(1)$, where SU denotes the special unitary group with determinant one.  In the SM, Quantum Chromidynamics (\Gls{QCD}) is governed by an $SU(3)$ symmetry whereas the symmetry $SU(2)\times U(1)$ acts on the Higgs field and Electroweak sector.

This chapters will focus on the Electro-weak SU(2)$\times$U(1) sector.  There are however two brief points to be made on the nature QCD, which are those of confinement and asymptotic freedom.
\begin{description}
\item [Confinement] refers to strength of the colour force as a function of distance, which, unlike with Electromagentism,  does not diminish with increased separation between quarks. This means that a lone colour charge can never be observed as in the limit of $R\to \infty$ the total energy would also go as $E(R)\to \infty$. Instead, when two quarks separate, the most energetically favoured state is to create two new quarks from the resulting colour field. This process is referred to as hadronization.

\item[Asymptotic freedom] is a product of the energy-dependent coupling stength in QCD. More specifically, it refers to the fact that the coupling strength becomes asymptotically weaker with increasing energy (and shorter distances). %  and at high energies specifically,
  Thus, if two quarks have a high enough energy they will feel almost not colour potential and can be treated using perturbation methods. Unfortunately, many of the experimentally interesting or accessible QCD processes are not in this limit. As a result either effective theories (i.e. theories which give correct results in certain limits) or non-perturbative methods are needed to produce QCD predictions. These methods will be discussed in more detail in section~\ref{subsec:ff}.
\end{description}
The rest of this section will focus on the Electroweak force, which is responsible for the Flavour Change Neutral Currents in the decay \Lbpi, denoted as $\Delta F = 2$. There will be a brief explanation of how fermions and the Electroweak bosons acquire mass, followed by a discussion on flavour and Minimal Flavour violation.

\section{Local gauge invariance}
\label{subsec:gauge}
The Lagrangian density for a free fermion is given as
\begin{equation}
  \label{eq:L_free}
  \mathcal{L} = i(\hbar c)\overline{\psi}\gamma^{\mu}\partial_{\mu}\psi - (mc^{2})\overline{\psi}\psi
\end{equation}
which yields the equation of motion

\begin{equation}
 i\partial_{\mu}\overline{\psi}\gamma^{\mu} -  (\frac{mc}{\hbar})\psi = 0
\end{equation}
This is the Dirac equation describing a particle of spin $\frac{1}{2}$ and mass $m$, here represented by the field $\psi$. 

If the spinor $\psi$ is transformed by a global phase such that $\psi \to e^{i\theta}\psi$ Eq.~\ref{eq:L_free} remains invariant, as $\overline{\psi} \to e^{-i\theta}\overline{\psi}$ and the exponentials cancel. However if the phase $\theta$ is dependent on space-time points, $x^{\mu}$, such that $\psi \to e^{i\theta(x)}\psi$, the resulting Lagrangian transform is $\mathcal{L} \to \mathcal{L} - \hbar c(\partial_{\mu}\theta)\overline{\psi}\gamma^{\mu}\psi$ (which arises from the presence of the $\partial_{\mu}\psi$ term). Thus Eq.~\ref{eq:L_free} breaks \emph{local gauge invariance}.

Requiring that a complete Lagrangian be invariant under local gauge transformations requires the addition of new gauge fields to Eq.~\ref{eq:L_free}. In this example such a field $A^{\mu}$ would be required to transform as $A^{\mu} \to A^{\mu} + \hbar c(\partial_{\mu}\theta)\overline{\psi}\gamma^{\mu}\psi$ in order to preserve local gauge invariance.

In the case of Quantum Eletrodynamics (QED), $A^{\mu}$ is the electromagnetic potential.

Introducing the $A^{\mu}$ field to Eq.~\ref{eq:L_free} also requires an additional new term describing a free photon. The free photon term consists of a kinematic part, $F^{\mu\nu}F_{\mu\nu}$, and a mass term, $-m^{2}A^{\mu}A_{\mu}$. Whereas the kinematic term for the free photon is locally gauge invariant the mass term is not, except in the special case when m = 0.

Thus the construction of a locally gauge invariant U(1) QED Lagraigian is fairly straightforward for two reasons:
\begin{enumerate}
\item
  The mass term of the boson, -$m^{2}A^{\mu}A_{\mu}$, is locally gauge invariant because m=0.
\item
  QED is a U(1) symmetry which does not differentiate between the left and right-handed parts of $\psi$, thus the fermion mass term, -$mc^{2}\psi\overline{\psi}$, in Eq.~\ref{eq:L_free} is also locally gauge invariant.
\end{enumerate}
\section{Generation of fermion masses}
Both points made above in section~\ref{subsec:gauge} do not apply to the case of the full SU(2) $\times$ U(1) Electroweak interaction. Firstly the $W^{\pm}$ and $Z^{0}$ bosons are observed to be massive unlike the photon. Secondly the weak force only interacts with left-handed fermions and right handed anti-fermions. The problem of this second point can be shown as follows. Using the left and right hand projection operators
\begin{equation}
  P_{R} = \frac{1+\gamma^{5}}{2}
\end{equation}
and
\begin{equation}
  P_{L} = \frac{1-\gamma^{5}}{2}
\end{equation}
the fermion spinor can be written as
\begin{equation}
\psi = \frac{1-\gamma^{5}}{2}\psi + \frac{1+\gamma^{5}}{2}\psi = \psi_{L}+\psi_{R}
\end{equation}
Thus $\psi_{L}\overline{\psi}_{L}$ = $\psi_{R}\overline{\psi}_{R}$ = 0 (as $P_{i}^{2} = P_{i}$ and $P_{R} + P_{L} = 1$). The gives the remaining mass term as

\begin{equation}
  -m\psi\overline{\psi} = -m[\overline{\psi}_{R}\psi_{L} + \overline{\psi}_{L}\psi_{R}]
\end{equation}
meaning that the weak currents can be written in terms of $\gamma^{\mu}$ matrices and $\gamma^{5}$. This gives the charged weak currents as
\begin{equation}
    J_{W}^{\mu^{+}} = \frac{1}{\sqrt{2}}(\overline{\nu}_{L}\gamma^{\mu}e_{L} + \overline{u}_{L}\gamma^{\mu}d_{L}) $$and$$ J_{W}^{\mu^{-}} = \frac{1}{\sqrt{2}}(\overline{e}_{L}\gamma^{\mu}\nu_{L} + \overline{d}_{L}\gamma^{\mu}u_{L})
    \label{eq:chargedcurr}
\end{equation}


Since $\psi_{L}$ is a weak isospin doublet ($\bf{I}=\frac{1}{2}$) but $\psi_{R}$ is an isospin singlet ($\bf{I}=0$) they will behave differently under rotations and thus this combination of left and right terms is not gauge invariant.
The solution is to introduce a very specific potential that keeps the full Lagrangian invariant under $SU(2)\times U(1)$ but will break the symmetry of the vacuum.
This potential is given by
\begin{equation}
  V(\phi) = -\mu^{2}\phi^{*}\phi +\lambda|\phi^{*}\phi|^{2}
\end{equation}

This is the Higgs potential  where $\mu$ is the Higgs mass parameters and $\lambda$ is the Higgs self-coupling. By using the Higgs field the fermions can acquire their mass through the Higgs vacuum expectation value, whilst still preserving gauge invariance.

The Higgs field is represented as a complex spinor
\begin{equation}
  \phi = \binom{\phi^{+}}{\phi^{0}}
  \label{eq:higgsspin_1}
\end{equation}

\begin{equation}
  \tilde{\phi} = \binom{\phi^{0^{*}}}{-\phi^{-}}
  \label{eq:higgsspin_2}
\end{equation}
and is an SU(2) doublet (isospin component $I_{3}$ =  $\pm\frac{1}{2}$) where the electric charges of the upper and lower components are chosen to ensure that hypercharge Y  = $\pm 1$ through the relation 
\begin{equation}
  Q = I_{3}+\frac{Y}{2}.
\end{equation}
A stable minima is found when $\mu^{2}<0$ which yields an infinite number of degenerate minima states which satisfy

\begin{equation}
  \phi\phi^{\dagger} = \sqrt{\frac{\mu^{2}}{2\lambda}} = \frac{\nu}{\sqrt{2}}.
\end{equation}
Here $\nu$ is a real constant measured to be 246 $\gev$~\cite{vev} and is the only parameter in the Standard Model to have units. In summary, by choosing $\mu^{2}<0$ the vacuum expectation value (i.e the value of $|\phi|$ at the minima) is now non-zero and the symmetry is spontaneously broken. %%Transfering to what is referred to as the unitary gauge, where by the scalar degrees of freedom are used to give the W^{\pm} and Z bosons their longitudinal degrees of freedom and there is a single real scalar particle, the Higgs, gives
 

The left-handed part of fermion wave-functions forms isospin doublets and the right-handed part forms singlets. Thus the left-hand part of the fermion field is given as
\begin{equation}
  \begin{split}
    &q_{L_{u,d}} = \binom{u_{L}}{d_{L}}, q_{l_{e,\nu_{e}}} = \binom{e_{L}}{\nu_{L}}\\
    \end{split}
\end{equation}
and the right-handed part
\begin{equation}
  u_{R}, u_{d}, e_{R}, \nu_{R}
\end{equation}


This combination of the doublet and singlet coupling of left and right-handed parts lead to the breaking of gauge invariance. However the combination , $L\phi\overline{R}$, of the left and right-handed part with the single Higgs isospin doulet gives SU(2) singlets and is invariant. Choosing
\begin{equation}
  \phi = \binom{0}{\nu} %$\implies$ $L\phi\overline{R}$ = 
\end{equation}

\begin{equation}
  \tilde{\phi} = \binom{\nu}{0}
\end{equation}


the Yukawa Lagrangian can be written as %%containg the quark terms is given as

\begin{equation}
\begin{split}
    &\mathcal{L}_{Yuk} = \Gamma_{mn}^{u}\overline{q}_{m,L}\tilde{\phi}u_{n,R} + \Gamma_{mn}^{d}\overline{q}_{m,L}\phi d_{n,R} + \Gamma_{mn}^{e}\overline{l}_{m,L}\phi e_{n,R} + \Gamma_{mn}^{\nu}\overline{l}_{m,L}\tilde{\phi}\nu_{n,R} +h.c = \\
 &f_{e}\overline{l}_{L_{e,\nu_{e}}}\phi e_{R} + f_{u}\overline{q}_{L_{u,d}}\tilde{\phi}u_{R} + f_{d}\overline{q}_{L_{u,d}}\phi d_{R} + h.c. \\
 & f_{\mu}\overline{l}_{L_{\mu,\nu_{\mu}}}\phi \mu_{R} + f_{c}\overline{q}_{L_{c,s}}\tilde{\phi}c_{R} + f_{s}\overline{q}_{L_{c,s}}\phi s_{R} + h.c. +\\
    & f_{\tau}\overline{l}_{L_{\tau,\nu_{\tau}}}\phi \tau_{R} + f_{t}\overline{q}_{L_{t,b}}\tilde{\phi}t_{R} + f_{b}\overline{q}_{L_{t,b}}\phi b_{R} + h.c. = \\
    & \frac{f_{e}\nu}{\sqrt{2}} (\overline{e}_{L}{e}_{R}+\overline{e}_{R}{e}_{L}) + \frac{f_{u}\nu}{\sqrt{2}}(\overline{u}_{L}{u}_{R}+\overline{u}_{R}{u}_{L}) +\frac{f_{d}\nu}{\sqrt{2}}(\overline{d}_{L}{d}_{R}+\overline{d}_{R}{d}_{L}) +  ...  \\
%%    & \frac{f_{\mu}\nu}{\sqrt{2}}(\overline{\mu}_{L}{\mu}_{R}+\overline{\mu}_{R}{\mu}_{L}) + \frac{f_{\u}\nu}{\sqrt{2}}(\overline{c}_{L}{c}_{R}+\overline{c}_{R}{c}_{L}) + \\
\end{split}
  \label{eq:yuk}
\end{equation}
where the indices on the top line imply summation over all families.

The fermions masses can be read off Eq.~\ref{eq:yuk} as
\begin{equation}
  m_{i} = \frac{f_{i}\nu}{\sqrt{2}}
\end{equation}

The weak-interaction eigenstates for the quarks are not the same however as the mass eigenstates. Let
\begin{equation}
  M_{weak}^{u} = \begin{pmatrix}u^{\prime}\\c^{\prime} \\t^{\prime}\end{pmatrix} = U_{u}\begin{pmatrix}u\\c\\t\end{pmatrix}  = U_{u}M_{mass}^{u}$$and$$ M_{weak}^{d} = \begin{pmatrix}d^{\prime}\\s^{\prime} \\b^{\prime}\end{pmatrix} = U_{d}\begin{pmatrix}d\\s\\b\end{pmatrix} = U_{d}M_{mass}^{d}
      \label{eq:weakeigen}
\end{equation}
where $U_{d}$ and $U_{u}$ are unitary matrices.
In the case of neutral currents the mass and weak-interaction eigenstates are the same and thus no flavour changing neutral currents are induced. Substituting Eq.~\ref{eq:weakeigen}{21} into Eq.~\ref{eq:chargedcurr}{9} (ignoring the leptonic part of the weak current) gives
\begin{equation}
  \begin{split}
    J_{W}^{\mu^{+}} =
    &\frac{1}{\sqrt{2}}\overline{M}_{weak_{L}}^{u}\gamma^{\mu}M_{weak_{L}}^{d} = \frac{1}{\sqrt{2}}\overline{M}_{mass_{L}}^{u}U_{u}^{\dagger}\gamma^{\mu}U_{d}M_{mass_{L}}^{d} = \\
    & \frac{1}{\sqrt{2}}\overline{M}_{mass_{L}}^{u}\gamma^{\mu}(U_{u}^{\dagger}U_{d})M_{mass_{L}}^{d} = \frac{1}{\sqrt{2}}\overline{M}_{mass_{L}}^{u}\gamma^{\mu}(V_{CKM})M_{mass_{L}}^{d}  = \\
    &\begin{pmatrix}\overline{u}&\overline{c}&\overline{t}\end{pmatrix}\gamma_{\mu}V_{CKM}\begin{pmatrix}d\\s\\b\end{pmatrix} = \\
      &\begin{pmatrix}\overline{u}&\overline{c}&\overline{t}\end{pmatrix}\gamma_{\mu}\begin{pmatrix}V_{ud}&V_{us}&V_{ub}\\V_{cd}&V_{cs}&V_{cb}\\V_{td}&V_{ts}&V_{tb}\\\end{pmatrix} \begin{pmatrix}  d\\s\\b\end{pmatrix} = \\
  \end{split}
\end{equation}
The elements of the CKM matrix have a hierarchical structure where the most off-diagonal terms (i.e. $V_{td}$ and $V_{ub}$) are the smallest and the diagonal terms are the largest ($\sim$ unity). As it is a $3\times3$ complex matrix there are initially 18 free parameters. The requirement of unitary reduces this to 9 and the addition of 5 relative unobserved phases between the elements reduces the total number of free parameters to 4. These 4 parameters can be expressed in the Wolfenstein parametrisation, which to third order in the parameter $\lambda$ gives ~\cite{wolf}
\begin{equation}
  \begin{pmatrix}1 - \frac{\lambda^{2}}{2} &\lambda& A\lambda^{3}(\rho - i\eta)\\-\lambda&1-\frac{\lambda^{2}}{2} & A\lambda^{2}\\ A\lambda^{3}(1-\rho - i\eta)&-A\lambda^{2}&1\\\end{pmatrix}
\end{equation}
where 
\begin{equation}
  A = 0.808\substack{+0.022\\-0.015}, \lambda = 0.2253\pm 0.0007,  \rho =+ 0.135\substack{+0.031\\-0.016},  \eta = 0.349\pm0.017
\end{equation}
%%as taken from Ref.~\cite{wolf}
%%http://arxiv.org/pdf/1002.0900v2.pdfwhere $\overline{\rho} = \rho(1-\lambda^{2}/2 + ...)$ and $\overline{\eta} = \eta(1-\lambda^{2}/2 + ...)$  as taken from Ref.~\cite{wolf} %\url{http://www.sciencedirect.com/science/article/pii/S0370269311009956}
%%%%%%%%%%%%%%%%%%%%%%%%%%%%%%%%%%%%%%%%%%%%%%%%%%%%%%%%%%%%%%%%%%%%%%%%%%%%%%%%%%%%%%%%%%%%%%%%%%%%%%%%%%%%%%%%%%%%%%%%%%%%%%%%%%%%%%%%%%%%%%
\section{Generation of gauge boson masses}
As discussed previously experiment shows that the Electroweak force has massive propagators. To illustrate how these gauge bosons $W^{\pm}$ and $Z^{0}$ acquire their mass it is necessary to introduce the full Higgs potential
\begin{equation}
\mathcal{L}_{scale} = (D^{\mu}\phi)^{\dagger}(D_{\mu}\phi) - V(\phi)
\end{equation}
where $D^{\mu}$ is the covariant derivative and is chosen as %such the scalar fields transfrom the same way as the gauge bosons.
\begin{equation}
  D_{\mu} = \partial_{\mu}+ig\frac{1}{2}\vec{\tau}.\vec{W_{\mu}} + ig^{'}\frac{1}{2}YB_{\mu}
  \label{eq:higgscodev}
\end{equation}
where $W^{\mu}$ and $B^{\mu}$ are gauge bosons and $\tau_{i = 1,2,3}$ are the group generators of SU(2), the Pauli matrices. As done previously 3 components of the $\phi$ field are can be set to zero to leave just the real part of the lower component of the spinor, $\binom{0}{\nu}$.
\section{Goldstone's theorem and the unitary gauge}
The choice of $\mu^{2} < 0$ results in a non-zero vacuum expectation value and thus spontaneously breaks the symmetry. % Goldstones therom states that in potenitals where the ground state is degenerate (as in this case) massless bscalar bosons will appear. These bosons however are not observed in nature. 
It is natural to write $\phi$ in terms of the fields ($\xi$,$\eta$) which are shifted to the vacuum minima, as sketched in Fig~\ref{fig:higgs}
\begin{figure}[ht!]
  \centering
  \subfloat[]{
    \includegraphics[scale = 0.3]{/home/es708/work_home/lambda/AnalysisNeat/EventSelection/screenshots/higgspotential}\label{fig:potential}}
  \\
  \subfloat[]
           {\includegraphics[scale = 0.3, trim  = 5mm 0mm 0mm 0mm, clip]{/home/es708/work_home/lambda/AnalysisNeat/EventSelection/screenshots/vacuumshift}\label{fig:fields}}
           \caption{Showing \protect\subref{fig:potential} the resulting $V(\phi)$ potential for the case of $\mu^{2}<0$ and \protect\subref{fig:fields} the shift of fields to the minima}
           \label{fig:higgs}
\end{figure}



This gives
\begin{equation}
  \phi_{0} = \frac{1}{\sqrt{2}}[(v+\eta) + i\xi]
  \end{equation}


To make the resulting Lagrangian easier to interpret a gauge is chosen such that the $\xi$ terms vanish.  This  requires $\phi$ to rotated by -$\frac{\xi}{\nu}$. Assuming such a rotation is infinitesimal such that terms $\mathcal{O}(\xi^{2}\eta^{2}\eta\xi)$ can be dropped leaves
\begin{equation}
  \phi^{'} \to e^{-i\xi/\nu}\phi  = e^{-i\xi/\nu}\frac{1}{\sqrt{2}}[(v+\eta) + i\xi] = e^{-i\xi/\nu}\frac{1}{\sqrt{2}}[v+\eta]e^{+i\xi/\nu} = \frac{1}{\sqrt{2}}[v+\eta] =\frac{1}{\sqrt{2}}[v+h]
\end{equation}
The absorption of $\xi$ (referred to as a Goldstone Boson) gives an extra degree of freedom to the $h$ field which is used to generate its mass. This choice of gauge is referred to as the unitary gauge and $h$ is the Higgs field of the real massive scalar particular, the Higgs. Goldstone's theorem states that for each broken generator of an original symmetry group a massless spin-zero particle will appear. By transforming the gauge this spin-zero particle can be eaten to provide the longitudinal components of other bosons $\implies$ \emph{if the symmetries associated with a gauge boson are broken then said gauge boson will acquire a mass via the Higgs mechanism.}
\section{Giving mass to the $\bf{W^{\mu}}$ and Z bosons}
Invariance of the potential $\phi$ under the a symmetry with a generator $Z$ implies that $e^{i\alpha Z}\phi_{0} = \phi_{0}$.  Again dropping higher orders implies that for such $\phi$ to remain invariant under such a transformation requires Z$\phi_{0} = 0$. Thus it can be seen that all SU(2) $W^{\mu}$  bosons and the $U(1)_{Y}$ $B^{\mu}$ gauge boson will acquire a mass as:
\begin{equation}
  \begin{split}
    SU(2):
    &\tau_{1}\phi_{0} =\begin{psmallmatrix}0&&1\\1&&0\end{psmallmatrix}\frac{1}{\sqrt{2}} \begin{psmallmatrix}0\\v+h\end{psmallmatrix}= +\frac{1}{\sqrt{2}}\begin{psmallmatrix}v+h\\0\end{psmallmatrix}\neq 0 \to broken\\
    &\tau_{2}\phi_{0} =\begin{psmallmatrix}0&&-i\\i&&0\end{psmallmatrix}\frac{1}{\sqrt{2}}\begin{psmallmatrix}0\\v+h\end{psmallmatrix}= -\frac{i}{\sqrt{2}}\begin{psmallmatrix}v+h\\0\end{psmallmatrix}\neq 0 \to broken\\
    &\tau_{3}\phi_{0} =\begin{psmallmatrix}1&&0\\0&&-1\end{psmallmatrix}\frac{1}{\sqrt{2}}\begin{psmallmatrix}0\\v+h\end{psmallmatrix}= -\frac{1}{\sqrt{2}}\begin{psmallmatrix}0\\v+h\end{psmallmatrix}\neq 0 \to broken\\
    U(1)_{Y}:
    &Y\phi_{0} =         Y_{\phi_{0}}\frac{1}{\sqrt{2}} \begin{psmallmatrix}0\\v+h\end{psmallmatrix}= +\frac{1}{\sqrt{2}}\begin{psmallmatrix}v+h\\0\end{psmallmatrix}\neq 0 \to broken\\
  \end{split}
\end{equation}
Thus all 4 gauge bosons acquire mass through the Higgs mechanism. However by mixing the third component of SU(2), $\tau_{3}$ with $U(1)_{Y}$ there is a combination  that leaves the symmetry unbroken namely
\begin{equation}
  U(1)_{EM}: Q\phi_{0} = (\tau_{3} + \frac{Y}{2})\phi_{0} = \begin{psmallmatrix}1&&0\\0&&0\end{psmallmatrix}\frac{1}{\sqrt{2}} \begin{psmallmatrix}0\\v+h\end{psmallmatrix}= 0 \to unbroken\\
\end{equation}

This combination corresponds to the massless photon. The physical gauge bosons $W^{\pm}$ are just combinations of $W^{1}$ and $W^{2}$ namely
\begin{equation}
  \tau_{+} = \frac{1}{2}(\tau_{1}+i\tau_{2})
\end{equation}
and
\begin{equation}
  \tau_{-} = \frac{1}{2}(\tau_{1}-i\tau_{2})
\end{equation}
The $Z$ and $\gamma$ and combinations of $B$ and $W^{3}$, namely
\begin{equation}
  Z_{\mu} = \frac{1}{\sqrt{g^{2}+g^{\prime 2}}} (gW_{3} - g^{\prime}B_{\mu})
\end{equation}
 and
 \begin{equation}
  A_{\mu} = \frac{1}{\sqrt{g^{2}+g^{\prime 2}}} (gW_{3} + g^{\prime}B_{\mu})
\end{equation}
 where $g$ and $g^{\prime}$ are free parameters as in Eq.~\ref{eq:higgscodev}. Expanding Eq.~\ref{eq:higgscodev} and making these substitutions leaves
\begin{equation}
  (D^{\mu}\phi)^\dagger(D_{\mu}\phi) = \frac{1}{8}\nu^{2}[g^{2}(W^{+})^{2} + g^{2}(W^{-})^{2} + (g^{2}+g^{\prime 2})Z_{\mu}^{2} + 0.A_{\mu}^{2}] 
  \end{equation}
from which the masses of the gauge bosons can be read off as $M^{2}V_{\mu}^{2}$ giving
\begin{equation}
  \begin{split}
    &M_{\gamma} = 0\\
    &M_{W_{-}} = M_{W_{+}} = \frac{1}{2}\nu g\\
    &M_{Z} =  \frac{1}{2}\nu\sqrt{g^{2}+g^{\prime 2}} \\
  \end{split}
\end{equation}
This leads to the expression
\begin{equation}
  \frac{M_{W}^{2}}{M_{Z}^{2}\cos^{2}(\theta_{W})}  = 1 $$where$$
  \cos(\theta_{W}) =  \frac{M_{W}}{M_{Z}}
\end{equation}
Thus the degrees of freedom given by Goldstone bosons that were generated when the symmetry was broken have been used to provide the longitudinal components of the $W^{\pm}$ and $Z$ bosons, allowing them to acquire mass whilst still preserving local gauge invariance. The Standard Model does not predict the values of $g$ and $g^{\prime}$. Experimental measurements of the masses gives

\begin{equation}
  \begin{split}
  &M_{W} = 80.385 \pm 0.015 \gevcc\\
    &M_{Z} = 91.1875 \pm 0.0021 \gevcc\\
    \end{split}
\end{equation}
The fact that these propagators are massive explains why the weak force is comparatively weak compared to the Electromagnetic force.
\section{The flavour problem}
\label{subsec:mfv}
Flavour changing neutral currents (FCNC), as in the case of \Lbpi, contain both QCD and weak contributions. Due to the large separation in distance and time scales of these two forces the total Lagrangian can be written as an effective theory whereby the physics can be separated at a certain energy $\mu$ such that the effective Hamiltonian is factorised into the short distance (energy $>\mu$) and long distance (energy $<\mu$) contributions. Perturbation theory can then be used to calculate the short distance physics using the effective Hamiltonian
\begin{equation}
  \mathcal{H}_{eff} = \sum_{i}\mathcal{C}_{i}\mathcal{O}_{i}
  \end{equation}
where $\mathcal{O}_{i}$ is an operator, called a Wilson operator, representing an effective interaction, and $\mathcal{C}_{i}$ is referred to as a Wilson Coefficient and is the coupling strength of this effective interaction. Here $i$ runs over the different Wilson operators. These Wilson coefficients can be calculated precisely in SM and New Physics (NP) models. Some of the tightest limits on a NP model with generic flavour structure come from meson oscillation such as $\overline{K}^{0}-K^{0}$ and $\overline{B}^{0}-B^{0}$ oscillations.  Such mixing occurs via FCNC where $\Delta$F = 2 and hence there is no contribution from tree-level diagrams. The effective Hamiltonian can be written as
\begin{equation}
  \mathcal{H}_{eff} = \sum_{i}\mathcal{C}_{ij}(\mathcal{O}_{Li}\gamma^{\mu}\mathcal{O}_{Lj})^{2}
\end{equation}
In the case of new physics the constant $\mathcal{C}_{ij}$ can be written as 
\begin{equation}
\mathcal{C}_{ij} = \frac{c_{ij}}{\Lambda^{2}}
\end{equation}
where $\Lambda$ is the mass scale of any NP contributions. Given that experimental measurements of such meson mixing suggest no presence of NP requires that $|\mathcal{A}_{NP}^{\Delta F = 2}|<|\mathcal{A}_{SM}^{\Delta F = 2}|$. This  sets the mass scale $\Lambda$ at~\cite{kaonmix}
\begin{equation}
  \Lambda>\frac{4.4\tev}{|V_{ti}^{*}V_{tj}|/|c_{ij}|^{1/2}} \sim
  \begin{cases}
    1.3\times 10^{4}\tev \times |c_{sd}|^{1/2}\\
    5.1\times 10^{2}\tev \times |c_{bd}|^{1/2}\\
    1.1\times 10^{2}\tev \times |c_{bs}|^{1/2}\\
  \end{cases}
  %%\right
\end{equation}
Assuming a generic structure where $c_{ij} = 1$ the analysis in Ref determines that $\Lambda \gtrsim$ 100\tev. Alternatively if $\Lambda \sim 1\tev$ then $c_{ij}\lesssim 10^{-5}$ ~\cite{flavourlimit} . This can be interpreted as either NP is found at $\Lambda \sim 1\tev$  but the coupling constants for NP contributions to $\Delta F=2$ operators are highly non-generic (and these particles have somehow evaded detection so far) or the coupling constants are generic but NP lies at a much higher energy scale. As NP at the $\sim \tev$ is well-motivated the Minimal Flavour Hypothesis (MFV) is adapted whereby it is assumed that physics does lie at the $\sim$ \tev scale but that the flavor structure has a hierarchy similar to that of the CKM matrix.
%%%%%%%%%%%%%%%%%%%%%%%%%%%%%%%%%%%%%%%%%%%%%%%%%
%%%%%%%%%%%%%%%%%%%%%%%%%%%%%%%%%%%%%%%%%%%%%%%5
%%%%%%%%%%%%%%%%%%%%%%%%%%%%%%%%%%%%%%%%%%%%%%%%
\section{Form factors for hadronic transition}
\label{subsec:ff}
Electroweak decays such as $\Lbpi$ also have contributions from non-perturbation QCD contributions which are difficult to calculate. These non-perturbative QCD contributions are expressed as form factors,  which describe the non-perturbuative QCD effects during hadronisation as a generic functional form, and are dependent on the final and initial hadron states. As these are non-perturbative they are difficult to calculate and there are various models and approximations used to do so. The most relevant for $b\to dll$ decays are Heavy Quark Effective Theory and lattice QCD. Heavy Quark Effective Theory can be effective when the transitioning quarks in both the initial and final state hadron are much heavier than the rest. In this scenario, instead of the light quark interacting directly with the heavy quark, the light quark can be treated as interacting with a colour potential whose source is the effectively stationary heavy quark. It is assumed that this potential is unchanged when the quark transition occurs. This works very well for $b\to c$ transitions but less well for $b\to s $ and $b\to d$ transitions, although it is still used, combined along with lattice QCD, in the calculation of $\Lb\to\Lambda_{0}\mun\mup$ decays~\cite{Meinel}, which will be discussed later.

Lattice QCD can be used in the case when there is no appropriate effective theory or perturbative alternative. The idea of lattice QCD is to express the matrix elements of interest as correlation functions, and then effectively use numerical integration to solve for the path integral corresponding to these correlation functions. This numerical integration is carried out on a grid or lattice of points in space time. At each lattice site a field, representing a quark, is defined and the link between each site represents the gluon. Monte Carlo Methods are then used to evaluate the path integrals, or gauge links, for different lattice configurations. The fact there is a finite spacing, given by the lattice spacing, $a$, means that discreet, as opposed to continuous, symmetries are involved and the theory remains renormalizable. It has show far had relative success, the proton mass for explain has been theoretically calculated to within 2\% of its actual value~\cite{proton}.


%%%%%%%%%%%%%%%%%%%%%%%%%%%%%%
%%%%%%%%%%%%%%%%%%%%%%%%%%%%%
%%%%%%%%%%%%%%%%%%%%%%%%%%%
\section{Using B\to hll to test minimal flavour violation}
The CKM elements $V_{ts}$ and $V_{td}$ are not likely to be measured with much precision with tree-level decays, as top quarks decay before they hadronize,  so decays mediated via loop or box diagrams are used to measure them. For absolute values for $V_{td}$ and $V_{ts}$  the most precise measurements come from $\B-\Bbar$ and $\Bs-\overline{\Bs}$ mixing respectively, the Feynmann diagrams for which can be seen in Fig~\ref{fig:Bdmix}.
\begin{figure}[!h]\def\nh{0.5\textwidth}
  \centering
  \includegraphics[clip=true,   trim =0mm 190mm 0mm 0mm, scale = 0.7]{feynmann/Bd_mix}
  \caption{An example of a feynmann diagram for $\B$ and $\Bs$ mixing}
  \label{fig:Bdmix}
\end{figure}
Of all of the $u$-type quarks appearing in Fig.~\ref{fig:Bdmix} the top will dominate as it is heaviest. This gives
\begin{equation}
  \begin{split}
    & \Delta m_{\B_{d}} \sim m_{t}^{2}|\Vtb\Vtd|^{2} \sim m_{t}^{2}.\mathcal{O}(\lambda^{6}) \\
    & \Delta m_{\B_{s}} \sim m_{t}^{2}|\Vtb\Vts|^{2} \sim m_{t}^{2}.\mathcal{O}(\lambda^{4})
  \end{split}
\end{equation}

Using these mixing measurements, for which $\Delta m_{B_{d}} = (0.510\pm0.003)\ps^{-1}$ \cite{pdg},  $\Delta m_{\B_{s}} = 17.761 ± 0.022\ps^{-1}$ \cite{bslhcb}, leads to
\begin{equation}
  \begin{split}
    & |\Vtd| = (8.4\pm0.6)\times 10^{-3}\\
    & |\Vts| = (40.0\pm2.7)\times 10^{-3}
  \end{split}
\end{equation}
where the uncertainties are theoretically dominated~\cite{pdg}. Many of these uncertainties cancel when the ratio of the two is taken which gives
\begin{equation}
  \frac{|V_{ts}|}{|V_{td}|} = 0.216\pm0.001\pm 0.011
  \label{eq:vtsvtd}
\end{equation}
As discussed in section~\ref{subsec:mfv} measuring such CKM elements can be used as a probe for potential new physics effects. Assuming the effects of any new physics are small more suppressed decays in the SM may be more sensitive to new physics, thus there is a motivation to measure $\frac{|V_{ts}|}{|V_{td}|}$ in rarer decays.
In Ref~\cite{pimumunew}, which supersedes the work in Ref~\cite{pimumuold} the value of $\BF(\B\to\pip\mun\mup)$ is measured to be $1.83\pm0.24\pm0.05$ where the first error is statistical and second is systematic. There has been much theoretical work around this decay channel (~\cite{bpipi_th_1}, ~\cite{bpipi_th_2}, ~\cite{bpipi_th_3} ~\cite{bpipi_th_4}) and the measured branching fraction was found to be consistent with theoretical prediction.  This channel is the equivalent process of $\Lbpi$ but with one less spectator quark, as demonstrated in Fig.~\ref{fig:boxpeng}.

\begin{figure}[!h]\def\nh{0.5\textwidth}
  \centering
  \hspace*{-1cm}
  \subfloat[]{\includegraphics[clip=true,   trim =0mm 150mm 0mm 30mm, scale = 0.4]{feynmann/Lb_peng_ds}\label{FD:1}}
  \subfloat[]{\includegraphics[clip=true, trim =0mm 150mm 0mm 30mm, scale = 0.4]{feynmann/B_peng_ds}\label{FD:3}}\\
  \hspace*{-1cm}
  \subfloat[]{\includegraphics[clip =true, trim = 0mm 150mm 0mm 30mm, scale = 0.4]{feynmann/Lb_box_ds}\label{FD:2}}%%\hskip 0.04\textwidth
  \subfloat[]{\includegraphics[clip =true, trim = 0mm 150mm 0mm 30mm, scale = 0.4]{feynmann/B_box_ds}\label{FD:4}}%%\hskip 0.04\textwidth
  \caption{Feynman diagrams for \protect\subref{FD:1} \Lb\to\proton\pim/\Km\mup\mun via a loop,  \protect\subref{FD:3} $B^{-}\to\pim/\kaon^{-}\mup\mun$ via a loop,  \protect\subref{FD:2}, \Lb\to\proton\pim/\Km\mup\mun via a box diagram,
   \protect\subref{FD:4}, $B^{-}\to\pim/\kaon^{-}\mup\mun$ via a box diagram.
  }
  \label{fig:boxpeng}
\end{figure}
The combination of $\BF(\B\to\pip\mun\mup)$ with $\BF(\B\to\Kp\mun\mup)$ ~\cite{bKmumu},  gives
\begin{equation}
  \frac{|V_{ts}|}{|V_{td}|} = \frac{\BF(\B\to\Kp\mun\mup)}{\BF(\B\to\pip\mun\mup)} \times \frac{\int F_{K}d q^{2}}{\int F_{\pi}\,dq^{2}} = 0.24^{+0.05}_{-0.04}
  \end{equation}
where $F_{K/\pi}d q^{2}$ is a combination of Wilson coefficients, phase space factors and form factors. The form factors for $\B\to\pi$ are taken from Ref~\cite{bpimumuff1} and Ref~\cite{bpimumuff2} and the form factors for $\B\to K$ are taken from Ref~\cite{bKmumuff1}.

This provides the most accurate determination of $|V_{td}/V_{ts}|$ from a decay a that includes both penguin and box diagrams and it is consistent with previous measurements as per Eq.~\ref{eq:vtsvtd}.

The value of $V_{ts}/\Vtd$ has never been measured before in the baryonic sector however, and there has thus far been no theoretical work done on this sector either. The study of baryonic $b\to dll$ decays via \Lbpi is important as the spin of \Lb baryon differs from that of the \B meson (1/2 as opposed 0) which can in principle prove an additional handle on the fundamental interaction~\cite{Meinel}. An angular analysis of \Lbpi, when more data in available, would also allow measurements of different Wilson Coefficients. In the case of $b\to sll$ decays measurements of the same process,e.g. the $b\to sll$ electroweak loop, done via either mesonic or baryonic decays, show different results with respect to the SM. The electroweak decays $\B\to\kaon^{*}\mup\mun$ and $\Lb\to\Lambda_{0}\mup\mun$ are both $b\to sl^{+}l^{-}$ type decays and in the former the branching fraction is lower than the SM prediction at high $q^{2}$ whist in the latter it is higher at high $q^{2}$, see Fig.~\ref{fig:bfq2}. Here $q^{2}$ refers to the total 4-momentum of the dimuon system. This could easily be statistics or poor theoretical understanding, as high $q^{2}$ means softer hadrons, however the argument remains that there is an interest in performing the same measurements via the baryonic sector as have been performed in the meson sector.
\begin{figure}[!h]\def\nh{0.5\textwidth}
  \centering
  \hspace*{-2cm}  
  \subfloat[]{\includegraphics [clip =true, trim = 50mm 50mm 10mm 0mm, scale = 0.35]{figs/lbtolmumuq2.png}\label{FD:1}}
   \subfloat[]{\includegraphics [clip =true, trim = 50mm 50mm 10mm 0mm, scale = 0.35]{figs/btokstmumuq2.png}\label{FD:2}}
   %%\subfloat[]{\includegraphicss[scale = 0.4]{figs/btokstmumuq2.png}\label{FD:3}}
  \caption{Branching fraction as a function of $q^{2}$ for \protect\subref{FD:1} $\B\to\kaon^{*}\mun\mup$~\cite{LHCB-PAPER-2015-051} and \protect\subref{FD:3} $\Lb\to\Lambda^{0}\mun\mup$~\cite{LHCB-PAPER-2015-009}}
  \label{fig:bfq2}
\end{figure}

\section{Effective Hamiltonian for b\to dll decays} %and $\boldmath{N_{\Lbpi}}$ for 3\invfb of data}
As mentioned there are no theoretical predictions for \BF(\Lbpi) and for a value of $|\Vts/\Vtd|$ to be obtained from a measurement the relevant form factors would need to be calculated. The decay rate would be calculated by treating the electroweak part of the diagram with an effective Hamiltonian, using Wilson co-efficient, as shown by the blob in Fig.~\ref{fig:wilson}. The theoretical difficulty is predicting the form factors for $\Lb\to\proton\pim$, which is made more complicated due to the fact that it involes baryons and is 3 body.
\begin{figure}[h!]
  \centering
  \includegraphics[clip=true, trim =0mm 150mm 0mm 30mm, scale = 0.7]{feynmann/Lb_peng_wilson}
  \caption{Showing the feynmann diagram for \Lbpi with an effective Hamiltonian for the electroweak part of the decay~\cite{lowrecoil}}
  \label{fig:wilson}
    \end{figure}
The effective Hamiltonian for a $b\to d$ decay is given as
\begin{equation}
H^{b\to d}_{eff} = \frac{4G_{F}}{\sqrt{2}}(\lambda_{u}\sum^{2}_{i=1}\mathcal{C}_{i}\mathcal{O}^{u}_{i} + \lambda_{c}\sum^{2}_{i=1}\mathcal{C}_{i}\mathcal{O}^{c}_{i} - \lambda_{t}\sum^{10}_{i=3}\mathcal{C}_{i}\mathcal{O}^{t}_{i}) + h.c.
\end{equation}

where as indicated $\mathcal{O}_{1,2}$ represent diagrams with internal $u$ and $c$ quarks and $\mathcal{O}_{3,10}$ represent diagrams with internal $t$ quarks and $\lambda_{p} = V_{pb}V^{*}_{pd}$, ($p = u,c,t$) are the products of CKM matrix elements. Unlike in the $b\to s$ case all three terms in the unitary relation have the same order of suppression. This is effectively because in each product $V_{pb}V^{*}_{pd}$ there is the same total number of changes across quark family pairs, each product being doubly Cabibbo suppressed in total, giving $\lambda_{u} \sim \lambda_{c} \sim \lambda_{t} \sim \lambda^{3}$, $\lambda$ being the Wolfenstein parameter. As mentioned the operators $\mathcal{O}_{1-2}$ represent diagrams with $u$ and $c$ quarks, moreover the operators $\mathcal{O}_{3-6}$ represent $b\to dq\overline{q}$ transitions and $\mathcal{O}_{8}$ represents a chromomagentic operator. Thus in the case of the electroweak process the operators $\mathcal{O}_{7,9,10}$ are of most interest as these represent either vector or axial Z current or a photon as shown in Fig.~\ref{fig:wilson7910}, meaning that it is the top quark that dominants.
\begin{figure}[ht!]
    \centering
  \hspace*{-2cm}
  \subfloat[]{
    \includegraphics[clip=true, trim =0mm 150mm 0mm 30mm, scale = 0.45]{feynmann/Lb_peng_wilson_coeff7}\label{coeff7}}
  \subfloat[]
           {\includegraphics[scale = 0.45, trim  = 0mm 150mm 10mm 0mm, clip]{feynmann/Lb_peng_wilson_coeff910}\label{coeff910}} 
           \caption{Showing \protect\subref{coeff7} the diagram represented by the $\mathcal{O}_{7}$ operator and\protect\subref{coeff910} the diagrams represented by the $\mathcal{O}_{9,10}$ operators~\cite{lowrecoil}}
           \label{fig:wilson7910}
\end{figure}

%% \begin{figure}[h!]
%%   \centering
%%   \includegraphics[clip=true, trim =0mm 150mm 0mm 30mm, scale = 0.7]{feynmann/Lb_peng_wilson}
%%   \caption{Showing the feynmann diagram for \Lbpi with an effective Hamiltonian for the electroweak part of the decay http://arxiv.org/pdf/1506.07760v3.pdf}
%%   \label{fig:wilson7910}
%%     \end{figure}

Thus the most important operators are given as 
\begin{equation}
\begin{split}
&\mathcal{O}_{7} = \frac{e}{g^{2}}m_{b}(\overline{q}_{L}\sigma^{\mu\nu}b_{R}F_{\mu\nu})\\
&\mathcal{O}_{9} = \frac{e}{g^{2}}m_{b}(\overline{q}_{L}\gamma_{\mu}b_{L}\lepton\gamma^{\mu}\lepton)\\
&\mathcal{O}_{10} = \frac{e}{g^{2}}m_{b}(\overline{q}_{L}\gamma_{\mu}b_{L}\lepton\gamma^{\mu}\gamma_{5}\lepton)\\
\end{split}
\end{equation}
where $F^{\nu\mu}$ is the electromagnetic field tensor and $\sigma_{\nu\mu}$ are the Pauli spin matrices. Using these operator definitions the effective Hamiltonian for a $b\to dll$ transition can be written as
\begin{equation}
\hspace*{-1cm}
H^{b\to dll}_{eff} = \frac{G_{F}\alpha_{em}\Vtb\Vts^{*}}{2\sqrt{2}}(C_{9}^{eff}\overline{d}\gamma_{\mu}(1 - \gamma_{5})b\overline{\lepton}\gamma^{\mu}\lepton + C_{10}\overline{d}\gamma_{\mu}(1-\gamma_{5})b\overline{\lepton}\gamma^{\mu}\gamma_{5}\lepton - 2m_{b}C_{7}\frac{1}{q^{2}}\overline{d}i\sigma_{\mu\nu}q^{\nu}(1+\gamma_{5})b\overline{\lepton}\gamma^{\mu}\lepton)
\end{equation}
where $q^{\nu}$ is the four-momenta of the dimuon system and the terms $(1-\gamma_{5})$ and $(1+\gamma_{5})$ project out either the left or right part, respectively,  of the $b$ quark.
To get the final amplitude it is necessary to sandwhich this effective Hamiltonian between the initial and final baryon states. As little work has been done on the decay of $\Lb$ into four-body final states, i.e. $\Lb\to h h \mu\mu$, it is useful to consider instead the case when the proton and the pion form a resonant $N^{*}$ state (which can loosely be thought of as an excited neutron) such that $\Lb\to N^{*}\mun\mup$. This gives the final amplitude as $\left<\Lb(p+q)|\mathcal{H}_{eff}|N^{*}(p)\right>$ where $p$ in the momentum of the $N^{*}$ state and $q$ is the momentum of dimuon system. Splitting into left and right handed parts gives two matrices elements to calculate, namely $\left<\Lb(p+q)|\overline{b}\gamma_{\mu}(1-\gamma_{5})d|N^{*}(p)\right>$ and $\left<\Lb(p+q)|\overline{b}i\sigma_{\mu\nu}q^{\nu}(1+\gamma_{5})d|N^{*}(p)\right>$  Using the theory work done on $\Lb\to\Lambda_{0}\mup\mun$ in Ref~\cite{lbtolmumu} gives %%$\left<\Lb(p+q)|\mathcal{H}_{eff}|N^{*}(p)\right>$
\begin{equation}
\hspace*{-0.5cm}
\begin{split}
&\left<\Lb(p+q)|\overline{b}i\sigma_{\mu\nu}q^{\nu}(1+\gamma_{5})d|N^{*}(p)\right> = \\&\\&\overline{u}_{N^{*}}(p)[\gamma_{\mu}f^{T}_{1}(Q^{2}) + i\sigma_{\mu\nu}q^{\nu}f^{T}_{2}(Q^{2}) + q^{\mu}f^{T}_{3}(Q^{2}) + \\
&\gamma_{\mu}\gamma_{5}g^{T}_{1}(Q^{2}) + i\sigma_{\mu\nu}\gamma_{5}q^{\nu}g^{T}_{2}(Q^{2}) + q^{\mu}\gamma_{5}g^{T}_{3}(Q^{2})]u_{\Lb}(p+q)
\end{split}
\end{equation}
\begin{equation}
\hspace*{-0.5cm}
\begin{split}
&\left<\Lb(p+q)|\overline{b}(1-\gamma_{5})d|N^{*}(p)\right> = \\&\\&\overline{u}_{N^{*}}(p)[\gamma_{\mu}f_{1}(Q^{2}) + i\sigma_{\mu\nu}q^{\nu}f_{2}(Q^{2}) + q^{\mu}f_{3}(Q^{2}) + \\
&\gamma_{\mu}\gamma_{5}g_{1}(Q^{2}) + i\sigma_{\mu\nu}\gamma_{5}q^{\nu}g_{2}(Q^{2}) + q^{\mu}\gamma_{5}g_{3}(Q^{2})]u_{\Lb}(p+q)\\
\end{split}
\end{equation}
where $g_{i}$,$g^{T}_{i}$ and $f_{i}$,$f^{T}_{i}$ are all form factors. Thus using these operators there are 12 form factors to calculate. There are however ways of reducing the number of form factors needed to describe a process, such as Heavy Quark Effective Theory, which is used, along with lattice QCD in Ref~\cite{Meinel} to predict the differential branching fraction of $\Lb\to\Lambda_{0}\mun\mup$.

\subsection{Leptonic $B^{+}\rightarrow \l^{+} \nu_{l}$}
This text is based on a summary provided by PDG on Leptonic Decays of Charged Pseudoscalar Mesons.


Purely leptonic decays, that proceed by annihilation-type diagrams of pseudoscalar mesons, have been of interest to the the CKM presicion community around



%% \subsection{Form factors for hadronic transtition}
%% \label{subsec:ff}
%% Electroweak decays such as $\Lbpi$ also have contributions from non-perturbative QCD contributions which are difficult to calculate. These non-perturbative QCD contributions are expressed as form factors,  which describe the non-perturbuatuve QCD effects during hadronisation as a generic functional form, and are dependent on the final and intial hadron states. As these are non-perturbative they are difficult to calculate and there are various models and approxiamtions used to do so. The most relevant for $b\to dll$ decays are Heavy Quark Effective Theory and lattice QCD. Heavy Quark Effective Theory can be effective when the transitiiong quarks in both the intial and final state hadron are much heavier than the rest. In this scenario, instead of the light quark interacting directly with the heavy quark, the light quark can be treated as interacting with a color potential whose source is the effectively stationary heavy quark. It assumed that this potential is unchanged when the quark tansition occurs. This works very well for $b\to c$ transitions but less well for $b\to s $ and $b\to d$ transistions, although it is still used, combined along with lattice QCD, in the calculation of $\Lb\to\Lambda_{0}\mun\mup$ decays.
%% Lattice QCD can be used in the case when no exact 


%% \subsection{Effective hamilitioan for b\to dll decays and $\boldmath{N_{\Lbpi}}$ for 3\invfb of data}
%% As mentioned there are no theoritical predictions for \BF(\Lbpi) and for a value of $|\Vts/\Vtd|$ to be obtained from a measuremnt the relevant form factors would need to be calculated. The decay rate woudl be calculated by treating the electroweak part of the diagram with an effective hamilitonion, using Wilson co-efficients, as shown by the blob in Fig.~\ref{fig:wilson}  The theoritical difficulty is predicting the form factors for $\Lb\to\proton\pim$, which is made more complicated due to the fact that it involed baryons and is 3 body.
%% \begin{figure}[h!]
%%   \centering
%%   \includegraphics[clip=true, trim =0mm 150mm 0mm 30mm, scale = 0.7]{feynmann/Lb_peng_wilson}
%%   \caption{Showing the feynmann diagram for \Lbpi with an effective Hamiltonian for the electroweak part of the decay http://arxiv.org/pdf/1506.07760v3.pdf}
%%   \label{fig:wilson}
%%     \end{figure}
%% The effective hamiltionan for a $b\to dll$ decay is given as
%% \begin{equation}
%% H^{b\to d}_{eff} = \frac{4G_{F}}{\sqrt{2}}(\lambda_{u}\sum^{2}_{i=1}\mathcal{C}_{i}\mathcal{O}^{u}_{i} + \lambda_{c}\sum^{2}_{i=1}\mathcal{C}_{i}\mathcal{O}^{c}_{i} - \lambda_{t}\sum^{10}_{i=3}\mathcal{C}_{i}\mathcal{O}^{t}_{i}) + h.c.
%% \end{equation}

%% where as indicated $\mathcal{0}_{1,2}$ represent diagrams with internal $u$ and $c$ quarks and $\mathcal{0}_{3,10}$ represent diagrams with internal $t$ quarks and $\lambda_{p} = V_{pb}V^{*}_{pd}$, ($p = u,c,t$) are the products of CKM matrix elements. Unlike in the $b\to sll$ case all three terms in the unitary relation have the same order of supression. This is effectvely because in each product $V_{pb}V^{*}_{pd}$ there is the same total number of changes across quark family pairs, each product being doubly Cabibbo supressed in total, giving $\lambda_{u} \sim \lambda_{c} \sim \lambda_{t} \sim \lambda^{3}$, $\lambda$ being the Wolfenstein parameter. A list of all .In the case of the electroweak process however only the last 3 operators are imporant as these represent either vector or aixial Z current or a photon http://arxiv.org/pdf/1506.07760v3.pdf http://lhc.fuw.edu.pl/misiak.pdf  
